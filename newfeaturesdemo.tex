% !TEX program = lualatexmk
% !TEX encoding = UTF-8 Unicode

\documentclass{article}
\usepackage{mandi}
\usepackage{mandistudent}
\usepackage{mandiexp}
\usepackage{newfeatures}
\usepackage{geometry}
\usepackage[titles]{tocloft}

\newcommand{\listequationsname}{List of Equations}
\newlistof{myequations}{equ}{\listequationsname}
\newcommand{\myequations}[1]{%
  \addcontentsline{equ}{myequations}{\protect\numberline{\theequation}#1}\par}
\renewcommand\cftmyequationsindent{\cftfigindent}
\renewcommand\cftmyequationsnumwidth{\cftfignumwidth}

\begin{document}
\tableofcontents
\newpage

\phantomsection
\tcblistof[\section*]{deriv}{List of Derivations}

\phantomsection
\listofmyequations
\newpage

\section{Nice Differentials And Evaluations In Integrals}
\begin{gather*}
  \int_{x=0}^{x=3} x^2 \,\diff{x} = \evaluatedat{\frac{1}{3}x^3}_0^3 = 9 \\
  \int_{x=0}^{x=3} x^2 \,\diff*{x} = \evaluatedat*{\frac{1}{3}x^3}_0^3 = 9 \\
  \int_{x=0}^{x=3} \diff{x} = \evaluatedat{x}_0^3 = 3  \\
  \int_{x=0}^{x=3} \diff*{x} = \evaluatedat*{x}_0^3 = 3 \\
  \int_{r=0}^{r=R} r\,\diff{r} \int_{\theta=0}^{\theta=\pi}\sin\theta\,\diff{\theta }
    \int_{\phi=0}^{\phi=2\pi} \diff{\phi} = 
    \evaluatedat{\frac{1}{2}r^2}_0^R \cdot
    \evaluatedat{-\cos\theta}_0^{\pi}\cdot
    \evaluatedat{\phi}_0^{2\pi} = 4\pi R^2 \\
  \int_{r=0}^{r=R} r\,\diff*{r} \int_{\theta=0}^{\theta=\pi}\sin\theta\,\diff*{\theta}
    \int_{\phi=0}^{\phi=2\pi} \diff*{\phi} = 
    \evaluatedat*{\frac{1}{2}r^2}_0^R  \cdot
    \evaluatedat*{-\cos\theta}_0^{\pi} \cdot
    \evaluatedat*{\phi}_0^{2\pi} = 4\pi R^2
\end{gather*}

\section{Virtual Parentheses}
Virtual Parentheses
\[
  \virtualparens[\Big]{-G\frac{m_1 m_2}{r}} \qquad -G\frac{\virtualparens{m_1 m_2}}{r}
    \qquad -G\virtualparens[\Big]{\frac{m_1 m_2}{r}}
\] 

\section{Derivation Environment}
New derivation environment

\begin{equation}
E =\gamma mc^2
\end{equation}
\myequations{Particle Energy}

\begin{derivation}[derivpeach]
  x + y &= z     \reason{given} \\
      y &= z - x \reason{solve for \(y\)} \\
      \begin{split}
        a &= b + c + d + e + f + g + k  \\
          &\quad + l + m + n + o + p + q + r
      \end{split} \reason{a very long expression that came from nowhere} \label{eqnnumd0}
\end{derivation}

\begin{equation}
E=h\nu
\end{equation}
\myequations{Photon Energy}

\begin{derivation}[derivorange]
  \gamma &= \frac{1}{\sqrt{1-v^2}}\reason{definition}\label{eqnumd1} \\
  \gamma^2 &= \frac{1}{1-v^2}\reason{square each side}\label{eqnumd2} \\
  \frac{1}{\gamma^2} &= 1-v^2\reason{take reciprocal of each side}\label{eqnumd3} \\
  v &= \sqrt{1-\frac{1}{\gamma^2}}\reason{rearrange and solve for \(v\) to get the 
                                          final answer} \label{eqnumd4}
\end{derivation}

Going from Eq.~\eqref{eqnumd1} to Eq.~\eqref{eqnumd4} isn't trivial, but it's quite simple.

\begin{equation}
a^2+b^2=c^2
\end{equation}
\myequations{Pythagorean Theorem}

\begin{derivation}[derivcyan]
  \gamma &= \frac{1}{\sqrt{1-v^2}}\reason{definition}\label{eqnumd5} \\
  \gamma^2 &= \frac{1}{1-v^2}\reason{square each side}\label{eqnumd6} \\
  \frac{1}{\gamma^2} &= 1-v^2\reason{take reciprocal of each side}\label{eqnumd7} \\
  v &= \sqrt{1-\frac{1}{\gamma^2}}\reason{rearrange and solve for \(v\) to get the 
                                          final answer} \label{eqnumd8}
\end{derivation}

\begin{derivation}[derivviolet]
  \gamma &= \frac{1}{\sqrt{1-v^2}}\reason{definition}\label{eqnumd9} \\
  \gamma^2 &= \frac{1}{1-v^2}\reason{square each side}\label{eqnumd10} \\
  \frac{1}{\gamma^2} &= 1-v^2\reason{take reciprocal of each side}\label{eqnumd11} \\
  v &= \sqrt{1-\frac{1}{\gamma^2}}\reason{rearrange and solve for \(v\) to get the 
                                          final answer} \label{eqnumd12}
\end{derivation}

\begin{derivation}
  \gamma &= \frac{1}{\sqrt{1-v^2}}\reason{definition}\label{eqnumd13} \\
  \gamma^2 &= \frac{1}{1-v^2}\reason{square each side}\label{eqnumd14} \\
  \frac{1}{\gamma^2} &= 1-v^2\reason{take reciprocal of each side}\label{eqnumd15} \\
  v &= \sqrt{1-\frac{1}{\gamma^2}}\reason{rearrange and solve for \(v\) to get the 
                                          final answer} \label{eqnumd16}
\end{derivation}

\begin{derivation}[derivgreen]
  \gamma &= \frac{1}{\sqrt{1-v^2}}\reason{definition}\label{eqnumd17} \\
  \gamma^2 &= \frac{1}{1-v^2}\reason{square each side}\label{eqnumd18} \\
  \frac{1}{\gamma^2} &= 1-v^2\reason{take reciprocal of each side}\label{eqnumd19} \\
  v &= \sqrt{1-\frac{1}{\gamma^2}}\reason{rearrange and solve for \(v\) to get the 
                                          final answer} \label{eqnumd20}
\end{derivation}

\begin{derivation}[derivmagenta]
  \gamma &= \frac{1}{\sqrt{1-v^2}}\reason{definition}\label{eqnumd21} \\
  \gamma^2 &= \frac{1}{1-v^2}\reason{square each side}\label{eqnumd22} \\
  \frac{1}{\gamma^2} &= 1-v^2\reason{take reciprocal of each side}\label{eqnumd23} \\
  v &= \sqrt{1-\frac{1}{\gamma^2}}\reason{rearrange and solve for \(v\) to get the 
                                          final answer} \label{eqnumd24}
\end{derivation}

\begin{derivation}[derivbrown]
  \gamma &= \frac{1}{\sqrt{1-v^2}}\reason{definition}\label{eqnumd25} \\
  \gamma^2 &= \frac{1}{1-v^2}\reason{square each side}\label{eqnumd26} \\
  \frac{1}{\gamma^2} &= 1-v^2\reason{take reciprocal of each side}\label{eqnumd27} \\
  v &= \sqrt{1-\frac{1}{\gamma^2}}\reason{rearrange and solve for \(v\) to get the 
                                          final answer} \label{eqnumd28}
\end{derivation}

\begin{derivation}[derivwhite]
  \gamma &= \frac{1}{\sqrt{1-v^2}}\reason{definition}\label{eqnumd29} \\
  \gamma^2 &= \frac{1}{1-v^2}\reason{square each side}\label{eqnumd30} \\
  \frac{1}{\gamma^2} &= 1-v^2\reason{take reciprocal of each side}\label{eqnumd31} \\
  v &= \sqrt{1-\frac{1}{\gamma^2}}\reason{rearrange and solve for \(v\) to get the 
                                          final answer} \label{eqnumd32}
\end{derivation}

\begin{derivation}[derivgray]
  \gamma &= \frac{1}{\sqrt{1-v^2}}\reason{definition}\label{eqnumd33} \\
  \gamma^2 &= \frac{1}{1-v^2}\reason{square each side}\label{eqnumd34} \\
  \frac{1}{\gamma^2} &= 1-v^2\reason{take reciprocal of each side}\label{eqnumd35} \\
  v &= \sqrt{1-\frac{1}{\gamma^2}}\reason{rearrange and solve for \(v\) to get the 
                                          final answer} \label{eqnumd36}
\end{derivation}

\begin{equation}
E_K = \frac{\left\lVert \vec{p} \right\rVert^2}{(\gamma + 1)m}
\end{equation}
\myequations{Relativistic Kinetic Energy}

This equation won't be listed.
\begin{equation}
E^2 = \left\lVert \vec{p} \right\rVert^2 c^2 + (mc^2)^2
\end{equation}

\newgeometry{left=0.5in,right=0.5in}
\section{Current Math Fonts}
\demomathfonts
\restoregeometry

\end{document}
