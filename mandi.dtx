% \iffalse meta-comment
% !TEX program = lualatexmk
%
% Copyright (C) 2021 by Paul J. Heafner <heafnerj@gmail.com>
% ---------------------------------------------------------------------------
% This  work may be  distributed and/or modified  under the conditions of the 
% LaTeX Project Public  License, either  version 1.3  of this  license or (at 
% your option) any later version. The  latest  version  of this license is in
%            http://www.latex-project.org/lppl.txt
% and  version 1.3 or  later is  part of  all distributions of  LaTeX version 
% 2005/12/01 or later.
%
% This work has the LPPL maintenance status `maintained'.
%
% The Current Maintainer of this work is Paul J. Heafner.
%
% This work consists of the files mandi.dtx
%                                 mandi.ins
%                                 mandi.pdf
%                                 README.md
%
% and includes the derived files  mandi.sty
%                                 mandiexp.sty
%                                 vdemo.py
% ---------------------------------------------------------------------------
%
% \fi
%
% \iffalse
%
%<*internal>
\iffalse
%</internal>
%
%<*readme>
mandi provides commands for introductory physics. To install, open a command 
line  and  type  the  following,  repeating 2-4 until there are no warnings:
 
  1. lualatex mandi.ins  (can also use latex)
  2. lualatex mandi.dtx  (lualatex is required)
  3. makeindex -s gind.ist -o mandi.ind mandi.idx
  4. makeindex -s gglo.ist -o mandi.gls mandi.glo  
Move the *.sty files into a directory searched by TeX.
%</readme>
%
%<*mandiexp>
\def\mandiexp@Version{3.0.0h}
\def\mandiexp@Date{2021-03-17}
\NeedsTeXFormat{LaTeX2e}[1999/12/01]
\providecommand\DeclareRelease[3]{}
\providecommand\DeclareCurrentRelease[2]{}
\DeclareRelease{v3.0.0h}{2021-03-17}{mandiexp.sty}
\DeclareCurrentRelease{v\mandiexp@Version}{\mandiexp@Date}
\ProvidesPackage{mandiexp}
 [%
 \mandiexp@Date\space v\mandiexp@Version\space Macros for Matter & Interactions
 ]
%
\newcommand*{\mandiexpversion}{v\mandiexp@Version\space dated \mandiexp@Date}
%
\typeout{}%
\typeout{mandiexp: You are using mandiexp \mandiexpversion.}
\typeout{}%
%
% Commands specific to Matter & Interactions
% The momentum principle
\NewDocumentCommand{\lhsmomentumprinciple}{ s }{%
  \Delta
  \IfBooleanTF{#1}%
    {\vec*{p}}%
    {\vec{p}}%
  _{\symup{sys}}%
}%
\NewDocumentCommand{\rhsmomentumprinciple}{ s }{%
  \IfBooleanTF{#1}%
    {\vec*{F}}%
    {\vec{F}}%
  _{\symup{sys,net}}\,\Delta t%
}%
\NewDocumentCommand{\lhsmomentumprincipleupdate}{ s }{%
  \IfBooleanTF{#1}%
    {\vec*{p}}%
    {\vec{p}}%
  _{\symup{sys,final}}%
}%
\NewDocumentCommand{\rhsmomentumprincipleupdate}{ s }{%
  \IfBooleanTF{#1}%
    {\vec*{p}}%
    {\vec{p}}%
  _{\symup{sys,initial}}+%
  \IfBooleanTF{#1}%
    {\vec*{F}}%
    {\vec{F}}%
  _{\symup{sys,net}}\,\Delta t%
}%
\NewDocumentCommand{\momentumprinciple}{ s }{%
  \IfBooleanTF{#1}%
    {\lhsmomentumprinciple* = \rhsmomentumprinciple*}%
    {\lhsmomentumprinciple = \rhsmomentumprinciple}%
}%
\NewDocumentCommand{\momentumprincipleupdate}{ s }{%
  \IfBooleanTF{#1}%
    {\lhsmomentumprincipleupdate* = \rhsmomentumprincipleupdate*}%
    {\lhsmomentumprincipleupdate = \rhsmomentumprincipleupdate}%
}%
% The momentum principle
\NewDocumentCommand{\lhsenergyprinciple}{}{%
  \Delta E_{\symup{sys}}%
}%
\NewDocumentCommand{\rhsenergyprinciple}{ O{} }{%
  W_{\symup{ext}}#1%
}%
\NewDocumentCommand{\lhsenergyprincipleupdate}{}{%
  E_{\symup{sys,final}}%
}%
\NewDocumentCommand{\rhsenergyprincipleupdate}{ O{} }{%
  E_{\symup{sys,initial}}+%
  W_{\symup{ext}}#1%
}%
\NewDocumentCommand{\energyprinciple}{ O{} }{%
  \lhsenergyprinciple = \rhsenergyprinciple[#1]%
}%
\NewDocumentCommand{\energyprincipleupdate}{ O{} }{%
  \lhsenergyprincipleupdate = \rhsenergyprincipleupdate[#1]%
}%
% The angular momentum principle
\NewDocumentCommand{\lhsangularmomentumprinciple}{ s }{%
  \Delta
  \IfBooleanTF{#1}%
    {\vec*{L}}%
    {\vec{L}}%
  _{A\symup{,sys,net}}%
}%
\NewDocumentCommand{\rhsangularmomentumprinciple}{ s }{%
  \IfBooleanTF{#1}%
    {\vec*{\tau}}%
    {\vec{\tau}}%
  _{A\symup{,sys,net}}\,\Delta t%
}%
\NewDocumentCommand{\lhsangularmomentumprincipleupdate}{ s }{%
  \IfBooleanTF{#1}%
    {\vec*{L}}%
    {\vec{L}}%
  _{A,\symup{sys,final}}%
}%
\NewDocumentCommand{\rhsangularmomentumprincipleupdate}{ s }{%
  \IfBooleanTF{#1}%
    {\vec*{L}}%
    {\vec{L}}%
  _{A\symup{,sys,initial}}+%
  \IfBooleanTF{#1}%
    {\vec*{\tau}}%
    {\vec{\tau}}%
  _{A\symup{,sys,net}}\,\Delta t%
}%
\NewDocumentCommand{\angularmomentumprinciple}{ s }{%
  \IfBooleanTF{#1}%
    {\lhsangularmomentumprinciple* = \rhsangularmomentumprinciple*}%
    {\lhsangularmomentumprinciple = \rhsangularmomentumprinciple}%
}%
\NewDocumentCommand{\angularmomentumprincipleupdate}{ s }{%
  \IfBooleanTF{#1}%
    {\lhsangularmomentumprincipleupdate* = \rhsangularmomentumprincipleupdate*}%
    {\lhsangularmomentumprincipleupdate = \rhsangularmomentumprincipleupdate}%
}%
\NewDocumentCommand{\energyof}{ m o }{%
  E_{#1\IfValueT{#2}{,#2}}%
}%
\NewDocumentCommand{\systemenergy}{ o }{%
  E_{\symup{sys}\IfValueT{#1}{,#1}}%
}%
\NewDocumentCommand{\particleenergy}{ o }{%
  E_{\symup{particle}\IfValueT{#1}{,#1}}%
}%
\NewDocumentCommand{\restenergy}{ o }{%
  E_{\symup{rest}\IfValueT{#1}{,#1}}%
}%
\NewDocumentCommand{\internalenergy}{ o }{%
  E_{\symup{internal}\IfValueT{#1}{,#1}}%
}%
\NewDocumentCommand{\chemicalenergy}{ o }{%
  E_{\symup{chem}\IfValueT{#1}{,#1}}%
}%
\NewDocumentCommand{\thermalenergy}{ o }{%
  E_{\symup{therm}\IfValueT{#1}{,#1}}%
}%
\NewDocumentCommand{\photonenergy}{ o }{%
  E_{\symup{photon}\IfValueT{#1}{,#1}}%
}%
\NewDocumentCommand{\translationalkineticenergy}{ s d[] }{%
% d[] must be used because of the way consecutive optional
%  arguments are handled. See xparse docs for details.
%  See https://tex.stackexchange.com/a/569011/218142
  \IfBooleanTF{#1}%
  {E_\bgroup \symup{K}}%
  {K_\bgroup\symup{trans}}%
       \IfValueT{#2}{,#2}%
     \egroup%
}%
\NewDocumentCommand{\rotationalkineticenergy}{ s d[] }{%
% d[] must be used because of the way consecutive optional
%  arguments are handled. See xparse docs for details.
%  See https://tex.stackexchange.com/a/569011/218142
  \IfBooleanTF{#1}%
  {E_\bgroup}%
  {K_\bgroup}%
       \symup{rot}\IfValueT{#2}{,#2}%
     \egroup%
}%
\NewDocumentCommand{\vibrationalkineticenergy}{ s d[] }{%
% d[] must be used because of the way consecutive optional
%  arguments are handled. See xparse docs for details.
%  See https://tex.stackexchange.com/a/569011/218142
  \IfBooleanTF{#1}%
  {E_\bgroup}%
  {K_\bgroup}%
       \symup{vib}\IfValueT{#2}{,#2}%
     \egroup%
}%
\NewDocumentCommand{\gravitationalpotentialenergy}{ o }{%
  U_{\symup{g}\IfValueT{#1}{,#1}}%
}%
\NewDocumentCommand{\electricpotentialenergy}{ o }{%
  U_{\symup{e}\IfValueT{#1}{,#1}}%
}%
\NewDocumentCommand{\springpotentialenergy}{ o }{%
  U_{\symup{s}\IfValueT{#1}{,#1}}%
}%
%</mandiexp>
%
%<*vdemo>
from vpython import *

scene.width = 400
scene.height = 760
# constants and data
g = 9.8       # m/s^2
mball = 0.03  # kg
Lo = 0.26     # m
ks = 1.8      # N/m
deltat = 0.01 # s 

# objects (origin is at ceiling)
ceiling = box(pos=vector(0,0,0), length=0.2, height=0.01, 
              width=0.2)
ball = sphere(pos=vector(0,-0.3,0),radius=0.025,
              color=color.orange)
spring = helix(pos=ceiling.pos, axis=ball.pos-ceiling.pos,
               color=color.cyan,thickness=0.003,coils=40,
               radius=0.010)

# initial values
pball = mball * vector(0,0,0)      # kg m/s
Fgrav = mball * g * vector(0,-1,0) # N
t = 0

# improve the display
scene.autoscale = False        # turn off automatic camera zoom
scene.center = vector(0,-Lo,0) # move camera down
scene.waitfor('click')         # wait for a mouse click

# initial calculation loop
# calculation loop
while t < 10:
    rate(100)
    # we need the stretch
    s = mag(ball.pos) - Lo
    # we need the spring force
    Fspring = ks * s * -norm(spring.axis)
    Fnet = Fgrav + Fspring
    pball = pball + Fnet * deltat
    ball.pos = ball.pos + (pball / mball) * deltat
    spring.axis = ball.pos - ceiling.pos
    t = t + deltat
%</vdemo>
%
%<*internal>
\fi
\def\nameofplainTeX{plain}
\ifx\fmtname\nameofplainTeX\else
  \expandafter\begingroup
\fi
%</internal>
%
%<*install>
\input docstrip.tex
\keepsilent
\askforoverwritefalse
\usedir{tex/latex/mandi}
\preamble

 Copyright (C) 2021 by Paul J. Heafner <heafnerj@gmail.com>
 ---------------------------------------------------------------------------
 This  work may be  distributed and/or modified  under the conditions of the 
 LaTeX Project Public  License, either  version 1.3  of this  license or (at 
 your option) any later version. The  latest  version  of this license is in
            http://www.latex-project.org/lppl.txt
 and  version 1.3 or  later is  part of  all distributions of  LaTeX version 
 2005/12/01 or later.

 This work has the LPPL maintenance status `maintained'.

 The Current Maintainer of this work is Paul J. Heafner.

 This work consists of the files mandi.dtx
                                 mandi.ins
                                 mandi.pdf
                                 README.md

 and includes the derived files  mandi.sty
                                 mandiexp.sty
                                 vdemo.py
 ---------------------------------------------------------------------------

\endpreamble

\generate{\usepreamble\empty\usepostamble\empty
          \file{README.md}{\from{\jobname.dtx}{readme}}}
\generate{\file{\jobname.ins}{\from{\jobname.dtx}{install}}}
\generate{\file{\jobname.sty}{\from{\jobname.dtx}{package}}}
\generate{\file{mandiexp.sty}{\from{\jobname.dtx}{mandiexp}}}
\generate{\usepreamble\empty\usepostamble\empty
          \file{vdemo.py}{\from{\jobname.dtx}{vdemo}}}

\obeyspaces
\Msg{*************************************************************}
\Msg{*                                                           *}
\Msg{* To finish the  installation, open a command line and      *}
\Msg{* type the following, repeating 2-4 until there are no      *}
\Msg{* warnings:                                                 *}
\Msg{*   2. lualatex mandi.dtx  (lualatex is required)           *}
\Msg{*   3. makeindex -s gind.ist -o mandi.ind mandi.idx         *}
\Msg{*   4. makeindex -s gglo.ist -o mandi.gls mandi.glo         *}
\Msg{* Move the *.sty files into a directory searched by TeX.    *}
\Msg{*                                                           *}
\Msg{*************************************************************}
%</install>
%<install>\endbatchfile
%
%<*internal>
\usedir{tex/latex/mandi}
\generate{\usepreamble\empty\usepostamble\empty
          \file{README.md}{\from{\jobname.dtx}{readme}}}
\generate{\file{\jobname.ins}{\from{\jobname.dtx}{install}}}
\generate{\file{\jobname.sty}{\from{\jobname.dtx}{package}}}
\generate{\file{mandiexp.sty}{\from{\jobname.dtx}{mandiexp}}}
\generate{\usepreamble\empty\usepostamble\empty
          \file{vdemo.py}{\from{\jobname.dtx}{vdemo}}}
\ifx\fmtname\nameofplainTeX
  \expandafter\endbatchfile
\else
  \expandafter\endgroup
\fi
%</internal>
%
%<*driver>
\ProvidesFile{mandi.dtx}
%</driver>
%
%<*driver>
\documentclass[10pt]{ltxdoc}
\PassOptionsToPackage{listings,documentation}{tcolorbox} % prevent option clash
\usepackage{\jobname}                                    % load mandi.sty
\usepackage{\jobname exp}                                % load mandiexp.sty
\usepackage{mwe}                                         % provides test images
\usepackage[left = 1.00in,%                              %
            right = 1.00in,%                             %
            marginparwidth = 0.70in]{geometry}           % main documentation
\usepackage[listings,documentation]{tcolorbox}           % workhorse package
\tcbset{%                                                % tcolorbox options
  index german settings,%
  index colorize = false,%
  lefthand ratio = 0.50,%
  color hyperlink = blue,%
  color command = purple,%
  color environment = purple!65!black,%
  doc left = 0.5in,%
  doc marginnote = {colframe = blue!50!white,colback = blue!5!white},%
  doc head command = {interior style = {fill,left color = blue!15!white}},%
  doc head environment = {interior style = {fill,left color = blue!15!white}},%
  doc head key = {interior style = {fill,left color = blue!15!white}},%
  docexample/.style = {%
      colback = gray!10!white,sidebyside,lefthand ratio = 0.5,center},%
  listing style = vpython,%
}%
% Redefine tcolorbox's \tcbdocnew and \tcbdocupdated defaults.
\renewcommand*{\tcbdocnew}[1]
  {\textcolor{green!50!black}{\sffamily\bfseries N} #1}
\renewcommand*{\tcbdocupdated}[1]
  {\textcolor{blue!75!black}{\sffamily\bfseries U} #1}
\hypersetup{colorlinks=true}                      % colored links; no borders

%  See https://tex.stackexchange.com/q/156383/218142
\newcommand*{\pkg}[1]{\textsf{#1}}                  % typeset package names
\newcommand*{\mandi}{\textsf{mandi}}                % typeset mandi
\newcommand*{\mandiexp}{\textsf{mandiexp}}          % typeset mandi
\newcommand*{\GlowScript}{\texttt{GlowScript}}      % typeset GlowScript
\newcommand*{\GlowScriptorg}{\texttt{GlowScript.org}} % typeset GlowScript.org
\newcommand*{\VPython}{\texttt{VPython}}            % typeset VPython
\newcommand*{\VPythonorg}{\texttt{VPython.org}}     % typeset VPython.org
\newcommand*{\gsurl}{glowscript.org}                % GlowScript URL
\newcommand*{\vpurl}{vpython.org}                   % VPython URL
\newcommand*{\lualatex}{Lua\LaTeX}                  % typeset LuaLaTeX

% A customized internal hyperref tool to
% mimic that in tcolorbox.
\NewDocumentCommand{\setplace}{ s m }{%
  \IfBooleanTF {#1}%
    {\phantomsection}%
    {}%
  \label{#2}%    
}%
\NewDocumentCommand{\linktoplace}{ m m }{%
  \hyperref[#1]{\texttt{#2}%
    \ifnum\getpagerefnumber{#1}=\thepage\relax%
    \else%
      \textsuperscript{\ding{213}\,{P.}\,\pageref*{#1}}%
    \fi%
  }%
}%

% We need a new command for in-line listings to prevent overfull boxes.
% Anything in |...| will be in small plain text.
% Previously used !...! but that conflicts with colors.
\lstMakeShortInline[basicstyle=\normalfont\ttfamily\small]|

\DisableCrossrefs         % index descriptions only
\PageIndex                % index refers to page numbers
\CodelineNumbered         % number source lines
\RecordChanges            % record changes
\begin{document}          % main document
  \DocInput{\jobname.dtx} %
  \PrintIndex             %
\end{document}            % end main document
%</driver>
% \fi
%
% \IndexPrologue{\section{Index}Page numbers refer to page where the 
%   corresponding entry is documented and/or referenced.}
% 
% \CheckSum{2488}
%
% \CharacterTable
%  {Upper-case    \A\B\C\D\E\F\G\H\I\J\K\L\M\N\O\P\Q\R\S\T\U\V\W\X\Y\Z
%   Lower-case    \a\b\c\d\e\f\g\h\i\j\k\l\m\n\o\p\q\r\s\t\u\v\w\x\y\z
%   Digits        \0\1\2\3\4\5\6\7\8\9
%   Exclamation   \!     Double quote  \"     Hash (number) \#
%   Dollar        \$     Percent       \%     Ampersand     \&
%   Acute accent  \'     Left paren    \(     Right paren   \)
%   Asterisk      \*     Plus          \+     Comma         \,
%   Minus         \-     Point         \.     Solidus       \/
%   Colon         \:     Semicolon     \;     Less than     \<
%   Equals        \=     Greater than  \>     Question mark \?
%   Commercial at \@     Left bracket  \[     Backslash     \\
%   Right bracket \]     Circumflex    \^     Underscore    \_
%   Grave accent  \`     Left brace    \{     Vertical bar  \|
%   Right brace   \}     Tilde         \~}
%
% \title{The \href{https://ctan.org/pkg/mandi}{\mandi} Package}
% \author{^^A
%    Paul J. Heafner\thanks{^^A
%      Email: \href{mailto:heafnerj@gmail.com?subject=[Heafner]\%20mandi}
%      {heafnerj@gmail.com}^^A
%    }^^A
% }^^A
% \date{\today}
%
% \newgeometry{left=1.0in,right=1.0in,top=4.0in}
% \maketitle
% \thispagestyle{empty}
% \centerline{Version \mandiversion}
% \centerline{\textbf{PLEASE DO NOT DISTRIBUTE THIS VERSION.}}
% \restoregeometry
%
% \newgeometry{left=1.0in,right=1.0in,top=0.5in,bottom=1.0in}
%   \tableofcontents
%   \newpage
%   \phantomsection
%   \addcontentsline{toc}{section}{Acknowledgements}
%   \section*{Acknowledgements}
%   To all of the students who have learned \LaTeX\ in my introductory
%   physics courses over the years, I say a heartfelt thank you. You
%   have contributed directly to the state of this software and to its
%   use in introductory physics courses and to innovating how physics
%   is taught. 
%   \newpage
%   \phantomsection
%   \addcontentsline{toc}{section}{Change History}
%   \PrintChanges
%   \newpage
%   \phantomsection
%   \addcontentsline{toc}{section}{List of \texttt{GlowScript} Programs}
%   \listofglowscriptprograms
%   \phantomsection
%   \addcontentsline{toc}{section}{List of \texttt{VPython} Programs}
%   \listofvpythonprograms
%   \phantomsection
%   \addcontentsline{toc}{section}{List of Figures}
%   \listoffigures
% \restoregeometry
%
% \changes{v3.0.0h}{2021-03-17}{Initial release.} 
%
% \section{Introduction}
% This is the documentation for the \mandi,\footnote{The package name can 
% be pronounced either with two syllables, to rhyme with \emph{candy}, or 
% with three syllables, as \emph{M and I}.} which is designed primarily 
% for students in introductory physics courses. This document serves to 
% document what commands \mandi\ provides and does not necessarily fully 
% demonstrate how students would use them. There is a separate document 
% that serves that purpose.
% 
% \subsection{Loading the Package}
% Load \mandi\ as you would any package in your preamble. 
%
%\iffalse
%<*example>
%\fi
\begin{dispListing*}{sidebyside=false,listing only}
  \usepackage[options]{mandi}
\end{dispListing*}
%\iffalse
%</example>
%\fi
%
%\iffalse
%<*example>
%\fi
\begin{docCommand}{mandiversion}{}
  Typesets the current version and build date.
\end{docCommand}
\begin{dispExample*}{sidebyside=false}
  The version is \mandiversion\ and is a stable build.
\end{dispExample*}
%\iffalse
%</example>
%\fi
%
% \subsection{Package Options}
%
%\iffalse
%<*example>
%\fi
\begin{docKeys}[%
    doc new = 2021-01-30,%
    doc keypath = {},%
  ]%
  {%
    {%
      doc name = units,%
      doc parameter = {=\meta{type of unit}},%
      doc description = {initially unspecified, set to \docValue{alternate}},%
    },%
    {%
      doc name = preciseconstants,%
      doc parameter = {=\meta{boolean}},%
      doc description = {initially unspecified, set to \docValue{false}},%
    },%
  }%
  Now \mandi\ uses a key-value interface for options.
  The \refKey{units} key can be set to \docValue{base}, \docValue{derived}, 
  or \docValue{alternate}. The \refKey{preciseconstants} key is always 
  either \docValue{true} or \docValue{false}.
\end{docKeys}
%\iffalse
%</example>
%\fi
%
% \subsection{The \texttt{mandisetup} Command}
% 
%\iffalse
%<*example>
%\fi
\begin{docCommand}[doc new=2021-02-17]{mandisetup}{\marg{options}}
  Command to set package options on the fly after loadtime. This 
  can be done in the preamble or inside the 
  |\begin{document}...\end{document}| environment.
\end{docCommand}
\begin{dispListing*}{sidebyside=false,listing only}
  \mandisetup{units=base}
\end{dispListing*}
%\iffalse
%</example>
%\fi
%
% \newpage
% \section{Student/Instructor Quick Guide}
% Use \refCom{vec} to typeset the symbol for a vector. Use \refCom{magnitude}
% to typeset the symbol for a vector's magnitude. Use \refCom{dirvec} to
% typeset the symbol for a vector's direction. Use \refCom{changein} to
% typeset the symbol for the change in a vector or scalar. Use 
% \refCom{zerovec} to typeset the zero vector. Use \refCom{timestento} to
% typeset scientific notation.
%
%\iffalse
%<*example>
%\fi
\begin{dispExample*}{lefthand ratio=0.80}
  \( \vec{p} \) or \( \vec*{p} \) \\
  \( \vec{p}_{\symup{final}} \) or \( \vec*{p}_{\symup{final}} \) \\
  \( \magnitude{\vec{p}} \) or \( \magnitude*{\vec{p}_{\symup{final}}} \) \\
  \( \dirvec{p} \) or \( \dirvec*{p} \) \\
  \( \changein \vec{p} \) or \( \changein t \) \\
  \( \zerovec \) or \( \zerovec* \) \\
  \( 6.02\timestento{-19} \)
\end{dispExample*}
%\iffalse
%</example>
%\fi
%
% Use a \linktoplace{ssec:physquants}{physical quantity's} name to typeset 
% a magnitude and that quantity's units. If the quantity is a vector, you 
% can add |vector| either to the beginning or the end of the quantity's 
% name. For example, if you want momentum, use \refCom{momentum} and 
% its variants.
%
%\iffalse
%<*example>
%\fi
\begin{dispExample}
  \( \momentum{7.071} \) \\
  \( \vectormomentum{3,-4,5} \) \\
  \( \momentumvector{3,-4,5} \)
\end{dispExample}
%\iffalse
%</example>
%\fi
%
% Use a \linktoplace{ssec:physconsts}{physical constant's} name 
% to typeset its numerical value and units. Append |mathsymbol| 
% to the constant's name to get its mathematical symbol. For 
% example, if you want to typeset the vacuum permittivity, use 
% \refCom{vacuumpermittivity} and its variant.
%
%\iffalse
%<*example>
%\fi
\begin{dispExample*}{lefthand ratio=0.70}
  \( \vacuumpermittivitymathsymbol = \vacuumpermittivity \)
\end{dispExample*}
%\iffalse
%</example>
%\fi
%
% Use \refCom{mivector} to typeset symbolic vectors with components.
% Use the aliases \refCom{direction} or \refCom{unitvector} to 
% typeset a direction or unit vector.
%\iffalse
%<*example>
%\fi
\begin{dispExample*}{sidebyside=false}
  \( \mivector{\slot,\slot,\slot} \) or \( \mivector{p_x,p_y,p_z} \) \\
  \( \direction{\frac{1}{\sqrt{3}},\frac{1}{\sqrt{3}},\frac{1}{\sqrt{3}}} \) or
  \( \unitvector{\frac{1}{\sqrt{3}},\frac{1}{\sqrt{3}},\frac{1}{\sqrt{3}}} \) 
\end{dispExample*}
%\iffalse
%</example>
%\fi
%
% Use \refEnv{physicsproblem} and \refEnv{parts} and \refCom{problempart} to 
% typeset problems. Use \refEnv{physicssolution} to typeset step-by-step 
% mathematical solutions. Use \refEnv{glowscriptblock} to typeset  
% \href{https://\gsurl}{\GlowScript} programs. Use \refCom{vpythonfile} to 
% typeset \href{https://\vpurl}{VPython} program files. 
%
% \newpage
% \section{Intelligent Commands for Physical Quantities and Constants}
% \subsection{Physical Quantities}
% \subsubsection{Typesetting Physical Quantities}\setplace{ssec:physquants}
% Typesetting physical quantities and constants using semantically appropriate 
% names, along with the correct 
% \href{https://en.wikipedia.org/wiki/International_System_of_Units}{SI units}, 
% is the core function of \mandi. Take momentum as the prototypical physical 
% quantity in an introductory physics course.
%
%\iffalse
%<*example>
%\fi
\begin{docCommands}
  {%
    {%
      doc name = momentum,%
      doc parameter = \marg{magnitude},%
    },%
    {%
      doc new = 2021-02-24,%
      doc name = momentumvector,%
      doc parameter = \marg{\ensuremath{c_1,\dots,c_n}},%
    },%
    {%
      doc name = vectormomentum,%
      doc parameter = \marg{\ensuremath{c_1,\dots,c_n}},%
    },%
  }%
  Command for momentum and its vector variant. The default units will depend 
  on the options passed to \mandi\ at load time. Alternate units are the 
  default. Other units can be forced as demonstrated. The vector variant can 
  take more than three components. Note the other variants for the quantity's 
  value and units.
\end{docCommands}
\begin{dispExample}
  \momentum{5}                \\
  \momentumvalue{5}           \\
  \momentumbaseunits{5}       \\
  \momentumderivedunits{5}    \\
  \momentumalternateunits{5}  \\
  \momentumonlybaseunits      \\
  \momentumonlyderivedunits   \\
  \momentumonlyalternateunits \\
  \vectormomentum{2,3,4}      \\
  \momentumvector{2,3,4}      \\
  \momentum{\mivector{2,3,4}} 
\end{dispExample}
%\iffalse
%</example>
%\fi
%
% Commands that include the name of a physical quantity typeset units, so 
% they shouldn't be used for algebraic or symbolic values of components.
% For example, one shouldn't use |\vectormomentum{mv_x,mv_y,mv_z}| but
% instead the generic |\mivector{mv_x,mv_y,mv_z}| instead.
%
% \subsubsection{Checking Physical Quantities}
%
%\iffalse
%<*example>
%\fi
\begin{docCommand}[doc new=2021-02-16]{checkquantity}{\marg{name}}
  Command to check and typeset the command, base units, 
  derived units, and alternate units of a defined physical 
  quantity.
\end{docCommand}
%\iffalse
%</example>
%\fi
%
% \subsubsection{Commands For Predefined Physical Quantities}
% Every other defined physical quantity can be treated similarly. Just replace 
% |momentum| with the quantity's name. Obviously, the variants that begin with 
% |\vector| will not be defined for scalar quantities. Here are all the 
% physical quantities, with all their units, defined in \mandi. Rememeber that 
% units are not present with symbolic (algebraic) quantities, so do not use 
% the |\vector| variants of these commands for symbolic components. 
% Use \refCom{mivector} instead.
%
%\iffalse
%<*example>
%\fi
\begin{docCommands}
  {%
    {%
      doc name = acceleration,%
      doc parameter = \marg{magnitude},%
    },%
    {%
      doc new = 2021-02-24,%
      doc name = accelerationvector,%
      doc parameter = \marg{\ensuremath{c_1,\dots,c_n}},%
    },%
    {%
      doc name = vectoracceleration,%
      doc parameter = \marg{\ensuremath{c_1,\dots,c_n}},%
    },%
  }%
\end{docCommands}
\checkquantity{acceleration}
\begin{docCommand}{amount}{\marg{magnitude}}
\end{docCommand}
\checkquantity{amount}
\begin{docCommands}
  {%
    {%
      doc name = angularacceleration,%
      doc parameter = \marg{magnitude},%
    },%
    {%
      doc new = 2021-02-24,%
      doc name = angularaccelerationvector,%
      doc parameter = \marg{\ensuremath{c_1,\dots,c_n}},%
    },%
    {%
      doc name = vectorangularacceleration,%
      doc parameter = \marg{\ensuremath{c_1,\dots,c_n}},%
    },%
  }%
\end{docCommands}
\checkquantity{angularacceleration}
\begin{docCommand}{angularfrequency}{\marg{magnitude}}
\end{docCommand}
\checkquantity{angularfrequency}
\begin{docCommands}
  {%
    {%
      doc name = angularimpulse,%
      doc parameter = \marg{magnitude},%
    },%
    {%
      doc new = 2021-02-24,%
      doc name = angularimpulsevector,%
      doc parameter = \marg{\ensuremath{c_1,\dots,c_n}},%
    },%
    {% 
      doc name = vectorangularimpulse,%
      doc parameter = \marg{\ensuremath{c_1,\dots,c_n}},%
    },%
  }%
\end{docCommands}
\checkquantity{angularimpulse}
\begin{docCommands}
  {%
    {%
      doc name = angularmomentum,%
      doc parameter = \marg{magnitude},%
    },%
    {%
      doc new = 2021-02-24,%
      doc name = angularmomentumvector,%
      doc parameter = \marg{\ensuremath{c_1,\dots,c_n}},%
    },%
    {%
      doc name = vectorangularmomentum,%
      doc parameter = \marg{\ensuremath{c_1,\dots,c_n}},%
    },%
  }%
\end{docCommands}
\checkquantity{angularmomentum}
\begin{docCommands}
  {%
    {%
      doc name = angularvelocity,%
      doc parameter = \marg{magnitude},%
    },%
    {%
      doc new = 2021-02-24,%
      doc name = angularvelocityvector,%
      doc parameter = \marg{\ensuremath{c_1,\dots,c_n}},%
    },%
    {%
      doc name = vectorangularvelocity,%
      doc parameter = \marg{\ensuremath{c_1,\dots,c_n}},%
    },%
  }%
\end{docCommands}
\checkquantity{angularvelocity}
\begin{docCommand}{area}{\marg{magnitude}}
\end{docCommand}
\checkquantity{area}
\begin{docCommand}{areachargedensity}{\marg{magnitude}}
\end{docCommand}
\checkquantity{areachargedensity}
\begin{docCommand}{areamassdensity}{\marg{magnitude}}
\end{docCommand}
\checkquantity{areamassdensity}
\begin{docCommand}{capacitance}{\marg{magnitude}}
\end{docCommand}
\checkquantity{capacitance}
\begin{docCommand}{charge}{\marg{magnitude}}
\end{docCommand}
\checkquantity{charge}
\begin{docCommands}
  {%
    {%
      doc name = cmagneticfield,%
      doc parameter = \marg{magnitude},%
    },%
    {%
      doc new = 2021-02-24,%
      doc name = cmagneticfieldvector,%
      doc parameter = \marg{\ensuremath{c_1,\dots,c_n}},%
    },%
    {%
      doc name = vectorcmagneticfield,%
      doc parameter = \marg{\ensuremath{c_1,\dots,c_n}},%
    },%
  }%
\end{docCommands}
\checkquantity{cmagneticfield}
\begin{docCommand}{conductance}{\marg{magnitude}}
\end{docCommand}
\checkquantity{conductance}
\begin{docCommand}{conductivity}{\marg{magnitude}}
\end{docCommand}
\checkquantity{conductivity}
\begin{docCommand}{conventionalcurrent}{\marg{magnitude}}
\end{docCommand}
\checkquantity{conventionalcurrent}
\begin{docCommand}{current}{\marg{magnitude}}
\end{docCommand}
\checkquantity{current}
\begin{docCommands}
  {%
    {%
      doc name = currentdensity,%
      doc parameter = \marg{magnitude},%
    },%
    {%
      doc new = 2021-02-24,%
      doc name = currentdensityvector,%
      doc parameter = \marg{\ensuremath{c_1,\dots,c_n}},%
    },%
    {%
      doc name = vectorcurrentdensity,%
      doc parameter = \marg{\ensuremath{c_1,\dots,c_n}},%
    },%
  }%
\end{docCommands}
\checkquantity{currentdensity}
\begin{docCommand}{dielectricconstant}{\marg{magnitude}}
\end{docCommand}
\checkquantity{dielectricconstant}
\begin{docCommands}
  {%
    {%
      doc name = displacement,%
      doc parameter = \marg{magnitude},%
    },%
    {%
      doc new = 2021-02-24,%
      doc name = displacementvector,%
      doc parameter = \marg{\ensuremath{c_1,\dots,c_n}},%
    },%
    {%
      doc name = vectordisplacement,%
      doc parameter = \marg{\ensuremath{c_1,\dots,c_n}},%
    },%
  }%
\end{docCommands}
\checkquantity{displacement}
\begin{docCommand}{duration}{\marg{magnitude}}
\end{docCommand}
\checkquantity{duration}
\begin{docCommands}
  {%
    {%
      doc name = electricdipolemoment,%
      doc parameter = \marg{magnitude},%
    },%
    {%
      doc new = 2021-02-24,%
      doc name = electricdipolemomentvector,%
      doc parameter = \marg{\ensuremath{c_1,\dots,c_n}},%
    },%
    { doc name = vectorelectricdipolemoment,%
      doc parameter = \marg{\ensuremath{c_1,\dots,c_n}},%
    },%
  }%
\end{docCommands}
\checkquantity{electricdipolemoment}
\begin{docCommands}
  {%
    {%
      doc name = electricfield,%
      doc parameter = \marg{magnitude},%
    },%
    {%
      doc new = 2021-02-24,%
      doc name = electricfieldvector,%
      doc parameter = \marg{\ensuremath{c_1,\dots,c_n}},%
    },%
    {%
      doc name = vectorelectricfield,%
      doc parameter = \marg{\ensuremath{c_1,\dots,c_n}},%
    },%
  }%
\end{docCommands}
\checkquantity{electricfield}
\begin{docCommand}{electricflux}{\marg{magnitude}}
\end{docCommand}
\checkquantity{electricflux}
\begin{docCommand}{electricpotential}{\marg{magnitude}}
\end{docCommand}
\checkquantity{electricpotential}
\begin{docCommand}{electroncurrent}{\marg{magnitude}}
\end{docCommand}
\checkquantity{electroncurrent}
\begin{docCommand}{emf}{\marg{magnitude}}
\end{docCommand}
\checkquantity{emf}
\begin{docCommand}{energy}{\marg{magnitude}}
\end{docCommand}
\checkquantity{energy}
\begin{docCommand}{energydensity}{\marg{magnitude}}
\end{docCommand}
\checkquantity{energydensity}
\begin{docCommands}
  {%
    {%
      doc name = energyflux,%
      doc parameter = \marg{magnitude},%
    },%
    {%
      doc new = 2021-02-24,%
      doc name = energyfluxvector,%
      doc parameter = \marg{\ensuremath{c_1,\dots,c_n}},%
    },%
    {%
      doc name = vectorenergyflux,% 
      doc parameter = \marg{\ensuremath{c_1,\dots,c_n}},%
    },%
  }%
\end{docCommands}
\checkquantity{energyflux}
\begin{docCommand}{entropy}{\marg{magnitude}}
\end{docCommand}
\checkquantity{entropy}
\begin{docCommands}
  {%
    {%
      doc name = force,%
      doc parameter = \marg{magnitude},%
    },%
    {%
      doc new = 2021-02-24,%
      doc name = forcevector,%
      doc parameter = \marg{\ensuremath{c_1,\dots,c_n}},%
    },%
    {%
      doc name = vectorforce,%
      doc parameter = \marg{\ensuremath{c_1,\dots,c_n}},%
    },%
  }%
\end{docCommands}
\checkquantity{force}
\begin{docCommand}{frequency}{\marg{magnitude}}
\end{docCommand}
\checkquantity{frequency}
\begin{docCommands}
  {%
    {%
      doc name = gravitationalfield,%
      doc parameter = \marg{magnitude},%
    },%
    {%
      doc new = 2021-02-24,%
      doc name = gravitationalfieldvector,%
      doc parameter = \marg{\ensuremath{c_1,\dots,c_n}},%
    },%
    {%
      doc name = vectorgravitationalfield,%
      doc parameter = \marg{\ensuremath{c_1,\dots,c_n}},%
    },%
  }%
\end{docCommands}
\checkquantity{gravitationalfield}
\begin{docCommand}{gravitationalpotential}{\marg{magnitude}}
\end{docCommand}
\checkquantity{gravitationalpotential}
\begin{docCommands}
  {%
    {%
      doc name = impulse,%
      doc parameter = \marg{magnitude},%
    },%
    {%
      doc new = 2021-02-24,%
      doc name = impulsevector,%
      doc parameter = \marg{\ensuremath{c_1,\dots,c_n}},%
    },%
    {%
      doc name = vectorimpulse,%
      doc parameter = \marg{\ensuremath{c_1,\dots,c_n}},%
    },%
  }%
\end{docCommands}
\checkquantity{impulse}
\begin{docCommand}{indexofrefraction}{\marg{magnitude}}
\end{docCommand}
\checkquantity{indexofrefraction}
\begin{docCommand}{inductance}{\marg{magnitude}}
\end{docCommand}
\checkquantity{inductance}
\begin{docCommand}{linearchargedensity}{\marg{magnitude}}
\end{docCommand}
\checkquantity{linearchargedensity}
\begin{docCommand}{linearmassdensity}{\marg{magnitude}}
\end{docCommand}
\checkquantity{linearmassdensity}
\begin{docCommand}{luminous}{\marg{magnitude}}
\end{docCommand}
\checkquantity{luminous}
\begin{docCommand}{magneticcharge}{\marg{magnitude}}
\end{docCommand}
\checkquantity{magneticcharge}
\begin{docCommands}
  {%
    {%
      doc name = magneticdipolemoment,%
      doc parameter = \marg{magnitude},%
    },%
    {%
      doc new = 2021-02-24,%
      doc name = magneticdipolemomentvector,%
      doc parameter = \marg{\ensuremath{c_1,\dots,c_n}},%
    },%
    {%
      doc name = vectormagneticdipolemoment,%
      doc parameter = \marg{\ensuremath{c_1,\dots,c_n}},%
    },%
  }%
\end{docCommands}
\checkquantity{magneticdipolemoment}
\begin{docCommands}
  {%
    {%
      doc name = magneticfield,%
      doc parameter = \marg{magnitude},%
    },%
    {%
      doc new = 2021-02-24,%
      doc name = magneticfieldvector,%
      doc parameter = \marg{\ensuremath{c_1,\dots,c_n}},%
    },%
    {%
      doc name = vectormagneticfield,%
      doc parameter = \marg{\ensuremath{c_1,\dots,c_n}},%
    },%
  }%
\end{docCommands}
\checkquantity{magneticfield}
\begin{docCommand}{magneticflux}{\marg{magnitude}}
\end{docCommand}
\checkquantity{magneticflux}
\begin{docCommand}{mass}{\marg{magnitude}}
\end{docCommand}
\checkquantity{mass}
\begin{docCommand}{mobility}{\marg{magnitude}}
\end{docCommand}
\checkquantity{mobility}
\begin{docCommand}{momentofinertia}{\marg{magnitude}}
\end{docCommand}
\checkquantity{momentofinertia}
\begin{docCommands}
  {%
    {%
      doc name = momentum,%
      doc label = momentumdemo,%
      doc parameter = \marg{magnitude},%
    },%
    {%
      doc new = 2021-02-24,%
      doc name = momentumvector,%
      doc name = momentumvectordemo,%
      doc parameter = \marg{\ensuremath{c_1,\dots,c_n}},%
    },%
    {%
      doc name = vectormomentum,%
      doc label = vectormomentumdemo,%
      doc parameter = \marg{\ensuremath{c_1,\dots,c_n}} },%
  }%
\end{docCommands}
\checkquantity{momentum}
\begin{docCommands}
  {%
    {%
      doc name = momentumflux,%
      doc parameter = \marg{magnitude},%
    },%
    {%
      doc new = 2021-02-24,%
      doc name = momentumfluxvector,%
      doc parameter = \marg{\ensuremath{c_1,\dots,c_n}},%
    },%
    {% 
      doc name = vectormomentumflux,% 
      doc parameter = \marg{\ensuremath{c_1,\dots,c_n}},%
    },%
  }%
\end{docCommands}
\checkquantity{momentumflux}
\begin{docCommand}{numberdensity}{\marg{magnitude}}
\end{docCommand}
\checkquantity{numberdensity}
\begin{docCommand}{permeability}{\marg{magnitude}}
\end{docCommand}
\checkquantity{permeability}
\begin{docCommand}{permittivity}{\marg{magnitude}}
\end{docCommand}
\checkquantity{permittivity}
\begin{docCommand}{planeangle}{\marg{magnitude}}
\end{docCommand}
\checkquantity{planeangle}
\begin{docCommand}{polarizability}{\marg{magnitude}}
\end{docCommand}
\checkquantity{polarizability}
\begin{docCommand}{power}{\marg{magnitude}}
\end{docCommand}
\checkquantity{power}
\begin{docCommands}
  {%
    {%
      doc name = poynting,%
      doc parameter = \marg{magnitude},%
    },%
    {%
      doc new = 2021-02-24,%
      doc name = poyntingvector,%
      doc parameter = \marg{\ensuremath{c_1,\dots,c_n}},%
    },%
    {%
      doc name = vectorpoynting,% 
      doc parameter = \marg{\ensuremath{c_1,\dots,c_n}},%
    },%
  }%
\end{docCommands}
\checkquantity{poynting}
\begin{docCommand}{pressure}{\marg{magnitude}}
\end{docCommand}
\checkquantity{pressure}
\begin{docCommand}{relativepermeability}{\marg{magnitude}}
\end{docCommand}
\checkquantity{relativepermeability}
\begin{docCommand}{relativepermittivity}{\marg{magnitude}}
\end{docCommand}
\checkquantity{relativepermittivity}
\begin{docCommand}{resistance}{\marg{magnitude}}
\end{docCommand}
\checkquantity{resistance}
\begin{docCommand}{resistivity}{\marg{magnitude}}
\end{docCommand}
\checkquantity{resistivity}
\begin{docCommand}{solidangle}{\marg{magnitude}}
\end{docCommand}
\checkquantity{solidangle}
\begin{docCommand}{specificheatcapacity}{\marg{magnitude}}
\end{docCommand}
\checkquantity{specificheatcapacity}
\begin{docCommand}{springstiffness}{\marg{magnitude}}
\end{docCommand}
\checkquantity{springstiffness}
\begin{docCommand}{springstretch}{\marg{magnitude}}
\end{docCommand}
\checkquantity{springstretch}
\begin{docCommand}{stress}{\marg{magnitude}}
\end{docCommand}
\checkquantity{stress}
\begin{docCommand}{strain}{\marg{magnitude}}
\end{docCommand}
\checkquantity{strain}
\begin{docCommand}{temperature}{\marg{magnitude}}
\end{docCommand}
\checkquantity{temperature}
\begin{docCommands}
  {%
    {%
      doc name = torque,%
      doc parameter = \marg{magnitude},%
    },%
    {%
      doc new = 2021-02-24,%
      doc name = torquevector,%
      doc parameter = \marg{\ensuremath{c_1,\dots,c_n}},%
    },%
    {%
      doc name = vectortorque,% 
      doc parameter = \marg{\ensuremath{c_1,\dots,c_n}},%
    },%
  }%
\end{docCommands}
\checkquantity{torque}
\begin{docCommands}
  {%
    {%
      doc name = velocity,%
      doc parameter = \marg{magnitude},%
    },%
    {%
      doc new = 2021-02-24,%
      doc name = velocityvector,%
      doc parameter = \marg{\ensuremath{c_1,\dots,c_n}},%
    },%
    {%
      doc name = vectorvelocity,%
      doc parameter = \marg{\ensuremath{c_1,\dots,c_n}},%
    },%
    {%
      doc name = velocityc,%
      doc parameter = \marg{magnitude},%
    },%
    {%
      doc new = 2021-02-24,%
      doc name = velocitycvector,%
      doc parameter = \marg{\ensuremath{c_1,\dots,c_n}},%
    },%
    {%
      doc name = vectorvelocityc,%
      doc parameter = \marg{\ensuremath{c_1,\dots,c_n}},%
    },%
  }%
\end{docCommands}
\checkquantity{velocity}
\checkquantity{velocityc}
\begin{docCommand}{volume}{\marg{magnitude}}
\end{docCommand}
\checkquantity{volume}
\begin{docCommand}{volumechargedensity}{\marg{magnitude}}
\end{docCommand}
\checkquantity{volumechargedensity}
\begin{docCommand}{volumemassdensity}{\marg{magnitude}}
\end{docCommand}
\checkquantity{volumemassdensity}
\begin{docCommand}{wavelength}{\marg{magnitude}}
\end{docCommand}
\checkquantity{wavelength}
\begin{docCommands}
  {%
    {%
      doc name = wavenumber,%
      doc parameter = \marg{magnitude},%
    },%
    {%
      doc new = 2021-02-24,%
      doc name = wavenumbervector,%
      doc parameter = \marg{\ensuremath{c_1,\dots,c_n}},%
    },%
    {%
      doc name = vectorwavenumber,% 
      doc parameter = \marg{\ensuremath{c_1,\dots,c_n}},%
    },%
  }%
\end{docCommands}
\checkquantity{wavenumber}
\begin{docCommand}{work}{\marg{magnitude}}
\end{docCommand}
\checkquantity{work}
\begin{docCommand}{youngsmodulus}{\marg{magnitude}}
\end{docCommand}
\checkquantity{youngsmodulus}
%\iffalse
%</example>
%\fi
%
% \subsubsection{Defining and Redefining Your Own Physical Quantities}
%
%\iffalse
%<*example>
%\fi
\begin{docCommands}[%
    doc parameter = \marg{name}\marg{base units}\oarg{derived units}\oarg{alternate units},%
  ]%
  {%
    {%
      doc new = 2021-02-16,%
      doc name = newscalarquantity,%
    },%
    {%
      doc new=2021-02-21,%
      doc name = renewscalarquantity,%
    },%
  }%
  Command to define/redefine a new/existing scalar quantity. 
  If the derived or alternate units are omitted, they are 
  defined to be the same as the base units. Do not use both 
  this command and 
  \refCom{newvectorquantity} or \refCom{renewvectorquantity} 
  to define/redefine a quantity.
\end{docCommands}
%\iffalse
%</example>
%\fi
%
%\iffalse
%<*example>
%\fi
\begin{docCommands}[%
    doc parameter = \marg{name}\marg{base units}\oarg{derived units}\oarg{alternate units},%
  ]%
  {%
    {%
      doc new = 2021-02-16,%
      doc name = newvectorquantity,%
    },%
    {%
      doc new=2021-02-21,%
      doc name = renewvectorquantity,%
    },%
  }%
  Command to define/redefine a new/existing vector quantity. 
  If the derived or alternate units are omitted, they are 
  defined to be the same as the base units. Do not use both 
  this command and 
  \refCom{newscalarquantity} or \refCom{renewscalarquantity} 
  to define/redefine a quantity.
\end{docCommands}
%\iffalse
%</example>
%\fi
%
% \subsubsection{Predefined Units and Constructs}
%
%\iffalse
%<*example>
%\fi
\begin{docCommands}[%
    doc parameter = {},%
  ]%
  {%
    {%
      doc name = per,%
    },%
    {%
      doc name = usk,%
    },%
    {%
      doc parameter = \marg{magnitude}\marg{unit},%
      doc name = unit,%
    },%
    {%
      doc name = emptyunit,%
    },%
    {%
      doc name = ampere,%
    },%
    {%
      doc name = atomicmassunit,%
    },%
    {%
      doc name = candela,%
    },%
    {%
      doc name = coulomb,%
    },%
    {%
      doc name = degree,%
    },%
    {%
      doc name = electronvolt,%
    },%
    {%
      doc name = farad,%
    },%
    {%
      doc name = henry,%
    },%
    {%
      doc name = hertz,%
    },%
    {%
      doc name = joule,%
    },%
    {%
      doc name = kelvin,%
    },%
    {%
      doc name = kilogram,%
    },%
    {%
      doc name = lightspeed,%
    },%
    {%
      doc name = meter,%
    },%
    {%
      doc name = metre,%
    },%
    {%
      doc name = mole,%
    },%
    {%
      doc name = newton,%
    },%
    {%
      doc name = ohm,%
    },%
    {%
      doc name = pascal,%
    },%
    {%
      doc name = radian,%
    },%
    {%
      doc name = second,%
    },%
    {%
      doc name = siemens,%
    },%
    {%
      doc name = steradian,%
    },%
    {%
      doc name = tesla,%
    },%
    {%
      doc name = volt,%
    },%
    {%
      doc name = watt,%
    },%
    {%
      doc name = weber,%
    },%
    {%
      doc name = tothetwo,%
      doc description = postfix,%
    },%
    {%
      doc name = tothethree,%
      doc description = postfix,%
    },%
    {%
      doc name = tothefour,%
      doc description = postfix,%
    },%
    {%
      doc name = inverse,%
      doc description = postfix,%
    },%
    {%
      doc name = totheinversetwo,%
      doc description = postfix,%
    },%
    {%
      doc name = totheinversethree,%
      doc description = postfix,%
    },%
    {%
      doc name = totheinversefour,%
      doc description = postfix,%
    },%
  }%
\end{docCommands}
\begin{dispExample}
  \per                         \\
  \usk                         \\
  \unit{3}{\meter\per\second}  \\
  \emptyunit                   \\
  \ampere                      \\
  \atomicmassunit              \\
  \candela                     \\
  \coulomb                     \\
  \degree                      \\
  \electronvolt                \\
  \farad                       \\
  \henry                       \\
  \hertz                       \\
  \joule                       \\
  \kelvin                      \\
  \kilogram                    \\
  \lightspeed                  \\
  \meter                       \\
  \metre                       \\
  \mole                        \\
  \newton                      \\
  \ohm                         \\
  \pascal                      \\
  \radian                      \\
  \second                      \\
  \siemens                     \\
  \steradian                   \\
  \tesla                       \\
  \volt                        \\
  \watt                        \\
  \weber                       \\
  \emptyunit\tothetwo          \\
  \emptyunit\tothethree        \\
  \emptyunit\tothefour         \\
  \emptyunit\inverse           \\
  \emptyunit\totheinversetwo   \\
  \emptyunit\totheinversethree \\
  \emptyunit\totheinversefour 
\end{dispExample}
%\iffalse
%</example>
%\fi
%
% \subsubsection{Changing Units}
%
%\iffalse
%<*example>
%\fi
\begin{docCommands}[%
    doc updated = 2021-02-26,%
    doc parameter = {},%
  ]%
  {%
    {%
      doc name = alwaysusebaseunits,%
    },%
    {%
      doc name = alwaysusederivedunits,%
    },%
    {%
      doc name = alwaysusealternateunits,%
    },%
  }%
  Modal commands (switches) for setting the default unit form for the entire 
  document. When \mandi\ is loaded, one of these three commands is executed 
  depending on whether the optional |units| key is provided. See the section 
  on loading the package for details. Alternate units are the default because 
  they are the most likely ones to be seen in introductory physics textbooks.
\end{docCommands}
%\iffalse
%</example>
%\fi
%
%\iffalse
%<*example>
%\fi
\begin{docCommands}[%
    doc updated = 2021-02-26,%
    doc parameter = \marg{content},%
  ]%
  {%
    {%
      doc name = hereusebaseunits,%
    },%
    {%
      doc name = hereusederivedunits,%
    },%
    {%
      doc name = hereusedalternateunits,%
    },%
  }%
  Commands for setting the unit form on the fly for a single instance. The 
  example uses momentum and the Coulomb constant, but they work for any 
  defined quantity and constant.
\end{docCommands}
\begin{dispExample}
  \hereusebaseunits{\momentum{5}}      \\
  \hereusederivedunits{\momentum{5}}   \\
  \hereusealternateunits{\momentum{5}} \\
  \hereusebaseunits{\oofpez}           \\
  \hereusederivedunits{\oofpez}        \\
  \hereusealternateunits{\oofpez}
\end{dispExample}
%\iffalse
%</example>
%\fi
%
%\iffalse
%<*example>
%\fi
\begin{docEnvironments}[%
    doc updated = 2021-02-26,%
    doc parameter = {},%
  ]%
  {%
    {% 
      doc name = usebaseunits,%
      doc description = use base units,% 
    },%
    {% 
      doc name = usederivedunits,%
      doc description = use derived units,%
    },%
    {%
      doc name = usealternateunits,% 
      doc description = use alternate units,%
    },%
  }%
  Inside these environments units are changed for the duration 
  of the environment regardless of the global default setting.
\end{docEnvironments}
\begin{dispExample}
  \momentum{5}   \\
  \oofpez        \\
  \begin{usebaseunits}
    \momentum{5} \\
    \oofpez      \\
  \end{usebaseunits}
  \begin{usederivedunits}
    \momentum{5} \\
    \oofpez      \\
  \end{usederivedunits}
  \begin{usealternateunits}
    \momentum{5} \\
    \oofpez     
  \end{usealternateunits}
\end{dispExample}
%\iffalse
%</example>
%\fi
%
% \subsection{Physical Constants}
% \subsubsection{Typesetting Physical Constants}\setplace{ssec:physconsts}
% Take the quantity \oofpezmathsymbol, sometimes called the 
% \href{https://en.wikipedia.org/wiki/Coulomb_constant}{Coulomb constant}, 
% as the prototypical 
% \href{https://en.wikipedia.org/wiki/Physical_constant}{physical constant} 
% in an introductory physics course. Here are all the ways to access this 
% quantity in \mandi. As you can see, these commands are almost identical 
% to the corresponding commands for physical quantities.
%
%\iffalse
%<*example>
%\fi
\begin{docCommand}[doc label = oofpezdemo]{oofpez}{}
  Command for the Coulomb constant. The constant's numerical precision and 
  default units will depend on the options passed to \mandi\ at load time. 
  Alternate units and approximate numerical values are the defaults. Other 
  units can be forced as demonstrated.
\end{docCommand}
\begin{dispExample}
  \oofpez                   \\
  \oofpezapproximatevalue   \\
  \oofpezprecisevalue       \\
  \oofpezmathsymbol         \\
  \oofpezbaseunits          \\
  \oofpezderivedunits       \\
  \oofpezalternateunits     \\
  \oofpezonlybaseunits      \\
  \oofpezonlyderivedunits   \\
  \oofpezonlyalternateunits
\end{dispExample}
%\iffalse
%</example>
%\fi
%
% \subsubsection{Checking Physical Constants}
%
%\iffalse
%<*example>
%\fi
\begin{docCommand}[doc updated = 2021-02-26]{checkconstant}{\marg{name}}
  Command to check and typeset the constant's name, base units, derived 
  units, alternate units, mathematical symbol, approximate value, and 
  precise value.
\end{docCommand}
%\iffalse
%</example>
%\fi
%
% \subsubsection{Commands For Predefined Physical Constants}
% Every other defined physical constant can be treated similarly. Just 
% replace |oofpez| with the constant's name. Unfortunately, there is no 
% universal agreement on the names of every constant so consult the next 
% section for the names that have been used. Here are all the physical 
% constants, with all their units, defined in \mandi. 
% The constants \refCom{coulombconstant} and \refCom{biotsavartconstant} are 
% defined as semantic aliases for, respectively, \refCom{oofpez} and 
% \refCom{mzofp}.
%
%\iffalse
%<*example>
%\fi
\begin{docCommand}{avogadro}{}
\end{docCommand}
\checkconstant{avogadro}
\begin{docCommand}[doc new = 2021-02-02]{biotsavartconstant}{}
\end{docCommand}
\checkconstant{biotsavartconstant}
\begin{docCommand}{bohrradius}{}
\end{docCommand}
\checkconstant{bohrradius}
\begin{docCommand}{boltzmann}{}
\end{docCommand}
\checkconstant{boltzmann}
\begin{docCommand}[doc new = 2021-02-02]{coulombconstant}{}
\end{docCommand}
\checkconstant{coulombconstant}
\begin{docCommand}{earthmass}{}
\end{docCommand}
\checkconstant{earthmass}
\begin{docCommand}{earthmoondistance}{}
\end{docCommand}
\checkconstant{earthmoondistance}
\begin{docCommand}{earthradius}{}
\end{docCommand}
\checkconstant{earthradius}
\begin{docCommand}{earthsundistance}{}
\end{docCommand}
\checkconstant{earthsundistance}
\begin{docCommand}{electroncharge}{}
\end{docCommand}
\checkconstant{electroncharge}
\begin{docCommand}{electronCharge}{}
\end{docCommand}
\checkconstant{electronCharge}
\begin{docCommand}{electronmass}{}
\end{docCommand}
\checkconstant{electronmass}
\begin{docCommand}{elementarycharge}{}
\end{docCommand}
\checkconstant{elementarycharge}
\begin{docCommand}{finestructure}{}
\end{docCommand}
\checkconstant{finestructure}
\begin{docCommand}{hydrogenmass}{}
\end{docCommand}
\checkconstant{hydrogenmass}
\begin{docCommand}{moonearthdistance}{}
\end{docCommand}
\checkconstant{moonearthdistance}
\begin{docCommand}{moonmass}{}
\end{docCommand}
\checkconstant{moonmass}
\begin{docCommand}{moonradius}{}
\end{docCommand}
\checkconstant{moonradius}
\begin{docCommand}{mzofp}{}
\end{docCommand}
\checkconstant{mzofp}
\begin{docCommand}{neutronmass}{}
\end{docCommand}
\checkconstant{neutronmass}
\begin{docCommand}{oofpez}{}
\end{docCommand}
\checkconstant{oofpez}
\begin{docCommand}{oofpezcs}{}
\end{docCommand}
\checkconstant{oofpezcs}
\begin{docCommand}{planck}{}
\end{docCommand}
\checkconstant{planck}
\begin{docCommand}{planckbar}{}
\end{docCommand}
\checkconstant{planckbar}
\begin{docCommand}{planckc}{}
\end{docCommand}
\checkconstant{planckc}
\begin{docCommand}{protoncharge}{}
\end{docCommand}
\checkconstant{protoncharge}
\begin{docCommand}{protonCharge}{}
\end{docCommand}
\checkconstant{protonCharge}
\begin{docCommand}{protonmass}{}
\end{docCommand}
\checkconstant{protonmass}
\begin{docCommand}{rydberg}{}
\end{docCommand}
\checkconstant{rydberg}
\begin{docCommand}{speedoflight}{}
\end{docCommand}
\checkconstant{speedoflight}
\begin{docCommand}{stefanboltzmann}{}
\end{docCommand}
\checkconstant{stefanboltzmann}
\begin{docCommand}{sunearthdistance}{}
\end{docCommand}
\checkconstant{sunearthdistance}
\begin{docCommand}{sunradius}{}
\end{docCommand}
\checkconstant{sunradius}
\begin{docCommand}{surfacegravfield}{}
\end{docCommand}
\checkconstant{surfacegravfield}
\begin{docCommand}{universalgrav}{}
\end{docCommand}
\checkconstant{universalgrav}
\begin{docCommand}{vacuumpermeability}{}
\end{docCommand}
\checkconstant{vacuumpermeability}
\begin{docCommand}{vacuumpermittivity}{}
\end{docCommand}
\checkconstant{vacuumpermittivity}
%\iffalse
%</example>
%\fi
%
% \subsubsection{Defining and Redefining Your Own Physical Constants}
%
%\iffalse
%<*example>
%\fi
\begin{docCommands}[%
    doc parameter = {%
    \marg{name}\marg{symbol}\marg{approximate value}\marg{precise value}\marg{base units}\\
    \oarg{derived units}\oarg{alternate units}%
    },%
  ]%
  {%
    {%
      doc new = 2021-02-16,%
      doc name = newphysicalconstant,%
    },%
    {%
      doc new=2021-02-21,%
      doc name = renewphysicalconstant,%
    },%
  }%
  Command to define/redefine a new/existing physical constant. 
  If the derived or alternate units are omitted, they are 
  defined to be the same as the base units.
\end{docCommands}
%\iffalse
%</example>
%\fi
%
% \subsubsection{Changing Precision}
%
%\iffalse
%<*example>
%\fi
\begin{docCommands}[%
    doc new = 2021-02-16,%
    doc parameter = {},%
  ]%
  {%
    {%
      doc name = alwaysuseapproximateconstants,%
    },%
    {%
      doc name = alwaysusepreciseconstants,%
    },%
  }%
  Modal commands (switches) for setting the default precision for the entire 
  document. The default with the package is loaded is set by the presence or
  absence of the \refKey{preciseconstants} key. 
\end{docCommands}
%\iffalse
%</example>
%\fi
%
%\iffalse
%<*example>
%\fi
\begin{docCommands}[%
    doc new = 2021-02-16,%
    doc parameter = \marg{content},%
  ]%
  {%
    {%
      doc name = hereuseapproximateconstants,%
    },%
    {%
      doc name = hereusepreciseconstants,%
    },%
  }%
  Commands for setting the precision on the fly for a single instance.
\end{docCommands}
\begin{dispExample}
  \hereuseapproximateconstants{\oofpez} \\
  \hereusepreciseconstants{\oofpez}
\end{dispExample}
%\iffalse
%</example>
%\fi
%
%\iffalse
%<*example>
%\fi
\begin{docEnvironments}[%
    doc new = 2021-02-16,%
    doc parameter = {},%
  ]%
  {%
    {% 
      doc name = useapproximateconstants,%
      doc description = use approximate constants,% 
    },%
    {% 
      doc name = usepreciseconstants,%
      doc description = use precise constants,%
    },%
  }%
  Inside these environments precision is changed for the duration 
  of the environment regardless of the global default setting.
\end{docEnvironments}
\begin{dispExample}
  \oofpez        \\
  \begin{useapproximateconstants}
    \oofpez      \\
  \end{useapproximateconstants}
  \begin{usepreciseconstants}
    \oofpez      \\
  \end{usepreciseconstants}
  \oofpez
\end{dispExample}
%\iffalse
%</example>
%\fi
%
% \newpage
% \section{\GlowScript\ and \VPython\ Program Listings}
% \subsection{The \texttt{\small glowscriptblock} Environment}
%
%\iffalse
%<*example>
%\fi
\begin{docEnvironment}[%
      doc updated = 2021-02-26,doclang/environment content=GlowScript code%
    ]%
  {glowscriptblock}{\oarg{options}(\meta{link})\marg{caption}}
  Code placed here is nicely formatted and optionally linked to its source on 
  \href{https://\gsurl}{\GlowScriptorg}. Clicking anywhere in the code window 
  will open the link in the default browser. A caption is mandatory, and a 
  label is internally generated. The listing always begins on a new page. A 
  URL shortening utility is recommended to keep the URL from getting unruly. 
  For convenience, |https://| is automatically prepended to the URL and can 
  thus be omitted.
\end{docEnvironment}
\begin{dispListing*}{sidebyside=false}
\begin{glowscriptblock}(tinyurl.com/y3lnqyn3){A \texttt{GlowScript} Program}
GlowScript 3.0 vpython

scene.width = 400
scene.height = 760
# constants and data
g = 9.8       # m/s^2
mball = 0.03  # kg
Lo = 0.26     # m
ks = 1.8      # N/m
deltat = 0.01 # s

# objects (origin is at ceiling)
ceiling = box(pos=vector(0,0,0), length=0.2, height=0.01, 
              width=0.2)
ball = sphere(pos=vector(0,-0.3,0),radius=0.025,
              color=color.orange)
spring = helix(pos=ceiling.pos, axis=ball.pos-ceiling.pos,
               color=color.cyan,thickness=0.003,coils=40,
               radius=0.010)

# initial values
pball = mball * vector(0,0,0)      # kg m/s
Fgrav = mball * g * vector(0,-1,0) # N
t = 0

# improve the display
scene.autoscale = False        # turn off automatic camera zoom
scene.center = vector(0,-Lo,0) # move camera down
scene.waitfor('click')         # wait for a mouse click

# initial calculation loop
# calculation loop
while t < 10:
    rate(100)
    # we need the stretch
    s = mag(ball.pos) - Lo
    # we need the spring force
    Fspring = ks * s * -norm(spring.axis)
    Fnet = Fgrav + Fspring
    pball = pball + Fnet * deltat
    ball.pos = ball.pos + (pball / mball) * deltat
    spring.axis = ball.pos - ceiling.pos
    t = t + deltat
\end{glowscriptblock}
\end{dispListing*}
\begin{glowscriptblock}(tinyurl.com/y3lnqyn3){A \texttt{GlowScript} program}
GlowScript 3.0 vpython

scene.width = 400
scene.height = 760
# constants and data
g = 9.8       # m/s^2
mball = 0.03  # kg
Lo = 0.26     # m
ks = 1.8      # N/m
deltat = 0.01 # s

# objects (origin is at ceiling)
ceiling = box(pos=vector(0,0,0), length=0.2, height=0.01, 
              width=0.2)
ball = sphere(pos=vector(0,-0.3,0),radius=0.025,
              color=color.orange)
spring = helix(pos=ceiling.pos, axis=ball.pos-ceiling.pos,
               color=color.cyan,thickness=0.003,coils=40,
               radius=0.010)

# initial values
pball = mball * vector(0,0,0)      # kg m/s
Fgrav = mball * g * vector(0,-1,0) # N
t = 0

# improve the display
scene.autoscale = False        # turn off automatic camera zoom
scene.center = vector(0,-Lo,0) # move camera down
scene.waitfor('click')         # wait for a mouse click

# initial calculation loop
# calculation loop
while t < 10:
    rate(100)
    # we need the stretch
    s = mag(ball.pos) - Lo
    # we need the spring force
    Fspring = ks * s * -norm(spring.axis)
    Fnet = Fgrav + Fspring
    pball = pball + Fnet * deltat
    ball.pos = ball.pos + (pball / mball) * deltat
    spring.axis = ball.pos - ceiling.pos
    t = t + deltat
\end{glowscriptblock}
%\iffalse
%</example>
%\fi
%
%\iffalse
%<*example>
%\fi
\begin{dispExample*}{sidebyside=false}
  \GlowScript\ program \ref{gs:1} is nice. 
  It's called \nameref{gs:1} and is on page \pageref{gs:1}.
\end{dispExample*}
%  
%\iffalse
%</example>
%\fi
%
% \subsection{The \texttt{\small vpythonfile} Command}
%
%\iffalse
%<*example>
%\fi
\begin{docCommand}[doc updated = 2021-02-26]{vpythonfile}
  {\oarg{options}\marg{file}\marg{caption}}
  Command to load and typeset a \VPython\ program. The file is read from 
  \marg{file}. Clicking anywhere in the code window can optionally open 
  a link, passed as an option, in the default browser. A caption is mandatory, 
  and a label is internally generated. The listing always begins on a new page. 
  A URL shortening utility is recommended to keep the URL from getting unruly. 
  For convenience, |https://| is automatically prepended to the URL and can 
  thus be omitted.
\end{docCommand}
\begin{dispListing*}{sidebyside=false}
\vpythonfile[hyperurl interior = https://vpython.org]{vdemo.py}
  {A \VPython\ program}
\end{dispListing*}
\vpythonfile[hyperurl interior = https://vpython.org]{vdemo.py}
  {A \VPython\ program}
%\iffalse
%</example>
%\fi
%
%\iffalse
%<*example>
%\fi
\begin{dispExample*}{sidebyside=false}
  \VPython\ program \ref{vp:1} is nice. 
  It's called \nameref{vp:1} and is on page \pageref{vp:1}.
\end{dispExample*}
%  
%\iffalse
%</example>
%\fi
%
% \subsection{The \texttt{\small glowscriptinline} and 
%   \texttt{\small vpythoninline} Commands}
%
%\iffalse
%<*example>
%\fi
\begin{docCommands}[%
    doc updated = 2021-02-26,%
  ]%
  {%
    {%
      doc name = glowscriptinline,%
      doc parameter = \marg{GlowScript code},%
    },%
    {%
      doc name = vpythoninline,%
      doc parameter = \marg{VPython code},%
    },%
  }%
  Typesets a small, in-line snippet of code. The snippet should be
  less than one line long.
\end{docCommands}
\begin{dispExample*}{sidebyside=false}
  \GlowScript\ programs begin with \glowscriptinline{GlowScript 3.0 VPython}
  and \VPython\ programs begin with \vpythoninline{from vpython import *}. 
\end{dispExample*}
%\iffalse
%</example>
%\fi
%
% \newpage
% \section{Commands for Writing Physics Problem Solutions}
% In addition to the \refEnv{glowscriptblock} environment and the
% \refCom{vpythonfile} command, the \refCom{glowscriptinline} 
% command, and \refCom{vpythoninline} command \mandi\ provides a 
% collection of commands physics students can use for writing problem 
% solutions. This new version focuses on the most frequently needed 
% tools. These commands should always be used in math mode.
% \subsection{Traditional Vector Notation}
%
%\iffalse
%<*example>
%\fi
\begin{docCommands}[%
    doc parameter = \marg{symbol}\oarg{labels},%
  ]%
  {%
    {%
      doc name = vec,%
      doc description = use this variant for boldface notation,%
    },%
    {%
      doc name = vec*,%
      doc description = use this variant for arrow notation,%
    }%
  }%
  Powerful and intelligent command for symbolic vector notation. The 
  mandatory argument is the symbol for the vector quantity. The optional 
  label(s) consists of superscripts and/or subscripts and can be 
  mathematical or textual in nature. If textual, be sure to wrap them in 
  |\symup{...}| for proper typesetting. The starred variant gives arrow 
  notation whereas without the star you get boldface notation. Subscript 
  and superscript labels can be arbitrarily mixed, and order doesn't matter.
\end{docCommands}
\begin{dispExample*}{lefthand ratio=0.6}
  \( \vec{p} \)                                \\
  \( \vec{p}_{2} \)                            \\
  \( \vec{p}^{\symup{ball}} \)                 \\
  \( \vec{p}_{\symup{final}} \)                \\
  \( \vec{p}^{\symup{ball}}_{\symup{final}} \) \\
  \( \vec{p}^{\symup{final}}_{\symup{ball}} \) \\
  \( \vec*{p} \)
\end{dispExample*}
%\iffalse
%</example>
%\fi
%
%\iffalse
%<*example>
%\fi
\begin{docCommands}[%
    doc parameter = \marg{symbol}\oarg{labels},%
  ]%
  {%
    {%
      doc name = dirvec,%
      doc description = use this variant for boldface notation,%
    },%
    {%
      doc name = dirvec*,%
      doc description = use this variant for arrow notation,%
    }%
  }%
  Powerful and intelligent command for typesetting the direction of 
  a vector. The options are the same as those for \refCom{vec}.
\end{docCommands}
\begin{dispExample*}{lefthand ratio=0.65}
  \( \dirvec{p} \)                                \\
  \( \dirvec{p}_{2} \)                            \\
  \( \dirvec{p}^{\symup{ball}} \)                 \\
  \( \dirvec{p}_{\symup{final}} \)                \\
  \( \dirvec{p}^{\symup{ball}}_{\symup{final}} \) \\
  \( \dirvec{p}^{\symup{final}}_{\symup{ball}} \) \\
  \( \dirvec*{p} \)
\end{dispExample*}
%\iffalse
%</example>
%\fi
%
%\iffalse
%<*example>
%\fi
\begin{docCommands}
  {%
    {%
      doc name = zerovec,%
      doc description = use this variant for boldface notation,%
    },%
    {%
      doc name = zerovec*,%
      doc description = use this variant for arrow notation,%
    },%
  }%
  Command for typesetting the zero vector. The starred variant gives 
  arrow notation. Without the star you get boldface notation.
\end{docCommands}
\begin{dispExample}
  \( \zerovec  \) \\
  \( \zerovec* \)
\end{dispExample}
%\iffalse
%</example>
%\fi
%
%\iffalse
%<*example>
%\fi
\begin{docCommand}{mivector}{%
    \oarg{delimiter}\marg{\ensuremath{c_1,\dots,c_n}}\oarg{units}
  }%
  Typesets a vector as either numeric or symbolic components with an 
  optional unit (for numerical components only). There can be more 
  than three components. The delimiter used in the list of components 
  can be specified; the default is a comma. The notation mirrors that of 
  \emph{Matter \& Interactions}.
\end{docCommand}
\begin{dispExample*}{lefthand ratio=0.6}
  \mivector{p_0,p_1,p_2,p_3}                        \\
  \mivector{\gamma m v_x,\gamma m v_y,\gamma m v_z} \\
  \mivector{\frac{Q_1Q_2}{x^2},0,0}                 \\
  \mivector{-1,0,0}                                 \\
  \mivector{-1,0,0}[\velocityonlyderivedunits]      \\
  \mivector{-1,0,0}[\meter\per\second]              \\
  \velocity{\mivector{-1,0,0}}
\end{dispExample*}
%\iffalse
%</example>
%\fi
%
%\iffalse
%<*example>
%\fi
\begin{docCommands}[%
    doc new = 2021-02-21,%
    doc parameter = \oarg{delimiter}\marg{\ensuremath{c_1,\dots,c_n}},%
  ]%
  {%
    {%
      doc name = direction,%
    },%
    {%
      doc name = unitvector
    },%
  }%
  Semantic aliases for \refCom{mivector}.
\end{docCommands}
\begin{dispExample*}{lefthand ratio=0.80}
  \direction{\frac{1}{\sqrt{3}},\frac{1}{\sqrt{3}},\frac{1}{\sqrt{3}}} \\
  \unitvector{\frac{1}{\sqrt{3}},\frac{1}{\sqrt{3}},\frac{1}{\sqrt{3}}}
\end{dispExample*}
%\iffalse
%</example>
%\fi
%
%\iffalse
%<*example>
%\fi
\begin{docCommand}{changein}{}%
  Semantic alias for |\Delta|.
\end{docCommand}
\begin{dispExample}
 \( \changein t \) \\
 \( \changein \vec{p} \)
\end{dispExample}
%\iffalse
%</example>
%\fi
%
%\iffalse
%<*example>
%\fi
\begin{docCommands}[%
    doc new = 2021-02-21,%
    doc parameter = \oarg{size}\marg{quantity},%
  ]%
  {%
    {%
      doc name = doublebars,%
      doc description = double bars,%
    },%
    {%
      doc name = doublebars*,%
      doc description = double bars for fractions,%
    },%
    {%
      doc name = singlebars,%
      doc description = single bars,%
    },%
    {%
      doc name = singlebars*,%
      doc description = single bars for fractions,%
    },%
    {%
      doc name = anglebrackets,%
      doc description = angle brackets,%
    },%
    {%
      doc name = anglebrackets*,%
      doc description = angle brackets for fractions,%
    },%
    {%
      doc name = parentheses,%
      doc description = parentheses,%
    },%
    {%
      doc name = parentheses*,%
      doc description = parentheses for fractions,%
    },%
    {%
      doc name = squarebrackets,%
      doc description = square brackets,%
    },%
    {%
      doc name = squarebrackets*,%
      doc description = square brackets for fractions,%
    },%
    {%
      doc name = curlybraces,%
      doc description = curly braces,%
    },%
    {%
      doc name = curlybraces*,%
      doc description = curly braces for fractions,%
    },%
  }%
  If no argument is given, a placeholder is provided. 
  Sizers like |\big|,|\Big|,|\bigg|, and |\Bigg| can 
  be optionally specified. Beginners are encouraged 
  not to use them. See the 
  \href{https://www.ctan.org/pkg/mathtools}{\pkg{mathtools}} package 
  documentation for details. 
\end{docCommands}
\begin{dispExample}
  \[ \doublebars{} \]
  \[ \doublebars{\vec{a}} \]
  \[ \doublebars*{\frac{\vec{a}}{3}} \]
  \[ \doublebars[\Bigg]{\frac{\vec{a}}{3}} \]
\end{dispExample}
\begin{dispExample}
  \[ \singlebars{} \]
  \[ \singlebars{x} \]
  \[ \singlebars*{\frac{x}{3}} \]
  \[ \singlebars[\Bigg]{\frac{x}{3}} \]
\end{dispExample}
\begin{dispExample}
  \[ \anglebrackets{} \]
  \[ \anglebrackets{\vec{a}} \]
  \[ \anglebrackets*{\frac{\vec{a}}{3}} \]
  \[ \anglebrackets[\Bigg]{\frac{\vec{a}}{3}} \]
\end{dispExample}
\begin{dispExample}
  \[ \parentheses{} \]
  \[ \parentheses{x} \]
  \[ \parentheses*{\frac{x}{3}} \]
  \[ \parentheses[\Bigg]{\frac{x}{3}} \]
\end{dispExample}
\begin{dispExample}
  \[ \squarebrackets{} \]
  \[ \squarebrackets{x} \]
  \[ \squarebrackets*{\frac{x}{3}} \]
  \[ \squarebrackets[\Bigg]{\frac{x}{3}} \]
\end{dispExample}
\begin{dispExample}
  \[ \curlybraces{} \]
  \[ \curlybraces{x} \]
  \[ \curlybraces*{\frac{x}{3}} \]
  \[ \curlybraces[\Bigg]{\frac{x}{3}} \]
\end{dispExample}
%\iffalse
%</example>
%\fi
%
%\iffalse
%<*example>
%\fi
\begin{docCommands}[%
    doc new = 2021-02-21,%
    doc parameter = \oarg{size}\marg{quantity},%
  ]%
  {%
    {%
      doc name = magnitude,%
      doc description = alias for double bars,%
    },%
    {%
      doc name = magnitude*,%
      doc description = alias for double bars for fractions,%
    },%
    {%
      doc name = norm,%
      doc description = alias for double bars,%
    },%
    {%
      doc name = norm*,%
      doc description = alias for double bars for fractions,%
    },%
    {%
      doc name = absolutevalue,%
      doc description = alias for single bars,%
    },%
    {%
      doc name = absolutevalue*,%
      doc description = alias for single bars for fractions,%
    },%
  }%
  Semantic aliases. Use \refCom{magnitude} or \refCom{magnitude*} to
  typeset the magnitude of a vector.
\end{docCommands}
\begin{dispExample}
  \[ \magnitude{\vec{p}} \]
  \[ \magnitude{\vec*{p}} \]
  \[ \magnitude*{\vec{p}_{\symup{final}}} \]
  \[ \magnitude*{\vec*{p}_{\symup{final}}} \]
\end{dispExample}
%\iffalse
%</example>
%\fi
%
%\iffalse
%<*example>
%\fi
\begin{docCommands}[%
    doc parameter = \oarg{delimiter}\marg{\ensuremath{c_1,\dots,c_n}},%
  ]%
  {%
    {%
      doc name = colvec,%
    },%
    {%
      doc name = rowvec,%
    },%
  }%
  Typesets column vectors and row vectors as numeric or symbolic components. 
  There can be more than three components. The delimiter used in the list of 
  components can be specified; the default is a comma. Units are not 
  supported, so these are mainly for symbolic work.
\end{docCommands}
\begin{dispExample}
  \[ \colvec{1,2,3} \]
  \[ \rowvec{1,2,3} \]
  \[ \colvec{x_0,x_1,x_2,x_3} \]
  \[ \rowvec{x^0,x^1,x^2,x^3} \]
\end{dispExample}
%\iffalse
%</example>
%\fi
%
%\iffalse
%<*example>
%\fi
\begin{docCommands}[%
    doc parameter = \marg{number},%
  ]%
  {%
    {%
      doc name = tento,%
    },%
    {%
      doc name = timestento,%
    },%
    {%
      doc name = xtento,%
    },%
  }%
  Commands for powers of ten and scientific notation.
\end{docCommands}
\begin{dispExample}
 \( \tento{-4} \)      \\
 \( 3\timestento{8} \) \\
 \( 3\xtento{8} \)
\end{dispExample}
%\iffalse
%</example>
%\fi
%
% \subsection{Coordinate-Free and Index Notation}
% Beyond the current level of introductory physics, we need intelligent 
% commands for typesetting vector and tensor symbols and components 
% suitable for both coordinate-free and index notations.
%
%\iffalse
%<*example>
%\fi
\begin{docCommands}[%
    doc parameter = \marg{symbol},%
  ]%
  {%
    {%
      doc name = veccomp,%
      doc description = use this variant for coordinate-free vector notation,%
    },%
    {%
      doc name = veccomp*,%
      doc description = use this variant for index vector notation,%
    },%
    {%
      doc name = tencomp,%
      doc description = use this variant for coordinate-free tensor notation,%
    },%
    {%
      doc name = tencomp*,%
      doc description = use this variant for index tensor notation,%
    },%
  }%
  Conforms to ISO 80000-2 notation.
\end{docCommands}
\begin{dispExample}
  \( \veccomp{r}  \) \\
  \( \veccomp*{r} \) \\
  \( \tencomp{r}  \) \\
  \( \tencomp*{r} \)
\end{dispExample}
%\iffalse
%</example>
%\fi
%
%\iffalse
%<*example>
%\fi
\begin{docCommands}[%
  doc parameter = \marg{index}\marg{index},%
  ]%
  {%
    {%
      doc name = valence,%
    },%
    {%
      doc name = valence*,%
    },%
  }%
  Typesets tensor valence. The starred variant typesets it horizontally.
\end{docCommands}
\begin{dispExample}
  A vector is a \( \valence{1}{0}  \) tensor. \\
  A vector is a \( \valence*{1}{0} \) tensor.
\end{dispExample}
%\iffalse
%</example>
%\fi
%
%\iffalse
%<*example>
%\fi
\begin{docCommands}[%
  doc parameter = \marg{slot,slot},%
  ]%
  {%
    {%
      doc name = contraction,%
    },%
    {%
      doc name = contraction*,%
    },%
  }%
  Typesets tensor contraction in coordinate-free notation. There
  is no standard on this so we assert one here.
\end{docCommands}
\begin{dispExample}
  \( \contraction{1,2}  \) \\
  \( \contraction*{1,2} \)
\end{dispExample}
%\iffalse
%</example>
%\fi
%
%\iffalse
%<*example>
%\fi
\begin{docCommands}%
  {%
    {%
      doc name = slot,%
      doc parameter = \oarg{vector},%
    },%
    {%
      doc name = slot*,%
      doc parameter = \oarg{vector},%
    },%
  }%
  An intelligent slot command for coordinate-free vector
  and tensor notation. The starred variants suppress the 
  underscore.
\end{docCommands}
\begin{dispExample}
  \( (\slot)  \)          \\
  \( (\slot[\vec{a}])  \) \\
  \( (\slot*) \)          \\
  \( (\slot*[\vec{a}]) \)
\end{dispExample}
%\iffalse
%</example>
%\fi
%
% \subsection{Problems and Annotated Problem Solutions}
%
%\iffalse
%<*example>
%\fi
\begin{docEnvironments}[%
    doc new = 2021-02-03,%
    doc parameter = \marg{title},%
    doclang/environment content = problem,%
  ]%
  {%
    {%
      doc name = physicsproblem,%
      doc description = use this variant for vertical lists,%
    },%
    {%
      doc name = physicsproblem*,%
      doc description = use this variant for in-line lists,%
    },%
    {%
      doc name = parts,%
      doc description = provides problem parts,%
    },%
  }%
  Provides an environment for stating physics problems. Each problem will 
  begin on a new page. See the examples for how to handle single and 
  multiple part problems.
\end{docEnvironments}
\begin{docCommand}[doc new = 2012-02-03]{problempart}{}
  Denotes a part of a problem within a \refEnv{parts}
  environment.
\end{docCommand}
\begin{dispExample*}{sidebyside=false}
  \begin{physicsproblem}{Problem 1}
    This is a physics problem with no parts.
  \end{physicsproblem}
\end{dispExample*}
\begin{dispExample*}{sidebyside=false}
  \begin{physicsproblem}{Problem 2}
    This is a physics problem with multiple parts.
    The list is vertical.
    \begin{parts}
      \problempart This is the first part.
      \problempart This is the second part.
      \problempart This is the third part.
    \end{parts}
  \end{physicsproblem}
\end{dispExample*}
\begin{dispExample*}{sidebyside=false}
  \begin{physicsproblem*}{Problem 3}
    This is a physics problem with multiple parts.
    The list is in-line.
    \begin{parts}
      \problempart This is the first part.
      \problempart This is the second part.
      \problempart This is the third part.
    \end{parts}
  \end{physicsproblem*}
\end{dispExample*}
%\iffalse
%</example>
%\fi
%
%\iffalse
%<*example>
%\fi
\begin{docEnvironments}[%
    doc updated = 2021-02-26,%
    doc parameter = {},%
    doclang/environment content = solution steps,%
  ]%
  {%
    {%
      doc name = physicssolution,%
      doc description = use this variant for numbered steps,%
    },%
    {%
      doc name = physicssolution*,%
      doc description = use this variant for unnumbered steps,%
    },%
  }%
  This environment is only for mathematical solutions. The starred 
  variant omits numbering of steps. See the examples.
\end{docEnvironments}
\begin{dispExample}
  \begin{physicssolution}
    x &= y + z \\
    z &= x - y \\
    y &= x - z
  \end{physicssolution}
  \begin{physicssolution*}
    x &= y + z \\
    z &= x - y \\
    y &= x - z
  \end{physicssolution*}
\end{dispExample}
%\iffalse
%</example>
%\fi
%
%\iffalse
%<*example>
%\fi
\begin{docCommand}[doc updated = 2012-02-26]{reason}{\marg{reason}}
  Provides an annotation in a step-by-step solution.
  Keep reasons short and to the point. Wrap mathematical
  content in math mode. 
\end{docCommand}
\begin{dispExample}
  \begin{physicssolution}
    x &= y + z \reason{This is a reason.}     \\
    z &= x - y \reason{This is a reason too.} \\
    y &= x - z \reason{final answer}
  \end{physicssolution}
  \begin{physicssolution*}
    x &= y + z \reason{This is a reason.}     \\
    z &= x - y \reason{This is a reason too.} \\
    y &= x - z \reason{final answer}
  \end{physicssolution*}
\end{dispExample}
%\iffalse
%</example>
%\fi
%
% When writing solutions, remember that the \refEnv{physicssolution} 
% environment is \emph{only} for mathematical content, not textual 
% content or explanations. 
%
%\iffalse
%<*example>
%\fi
\begin{dispListing*}{sidebyside=false,listing only}
  \begin{physicsproblem}{Combined Problem and Solution}
    This is an interesting physics problem.
    \begin{physicssolution}
      The solution goes here.
    \end{physicssolution}
  \end{physicsproblem}
\end{dispListing*}
\begin{dispListing*}{sidebyside=false,listing only}
  \begin{physicsproblem}{Combined Multipart Problem with Solutions}
    This is a physics problem with multiple parts.
    \begin{parts}
      \problempart This is the first part.
        \begin{physicssolution}
          The solution goes here.
        \end{physicssolution}
      \problempart This is the second part.
        \begin{physicssolution}
          The solution goes here.
        \end{physicssolution}
      \problempart This is the third part.
        \begin{physicssolution}
          The solution goes here.
        \end{physicssolution}
    \end{parts}
  \end{physicsproblem}
\end{dispListing*}
%\iffalse
%</example>
%\fi
%
%\iffalse
%<*example>
%\fi
\begin{docCommand}[doc new=2021-02-06]{hilite}{%
    \oarg{color}\marg{target}\oarg{shape}
  }%
  Hilites the desired target, which can be an entire mathematical expression 
  or a part thereof. The default color is magenta and the default shape is a
  rectangle.
\end{docCommand} 
\begin{dispListing*}{sidebyside=false,listing only}
  \begin{align*}
    (\Delta s)^2 &= -(\Delta t)^2 + (\Delta x)^2 + (\Delta y)^2 + 
                     (\Delta z)^2 \\
    (\Delta s)^2 &= \hilite{-(\Delta t)^2 + (\Delta x)^2}[rounded rectangle] + 
                     (\Delta y)^2 + (\Delta z)^2 \\
    (\Delta s)^2 &= \hilite{-(\Delta t)^2 + (\Delta x)^2}[rectangle] + 
                     (\Delta y)^2 + (\Delta z)^2 \\
    (\Delta s)^2 &= \hilite{-(\Delta t)^2 + (\Delta x)^2}[ellipse] + 
                     (\Delta y)^2 + (\Delta z)^2 \\
    (\Delta s)^{\hilite{2}[circle]} &= \hilite[green]{-}[circle]
                 (\Delta t)^{\hilite[cyan]{2}[circle]}+
                 (\Delta x)^{\hilite[orange]{2}[circle]} + 
                 (\Delta y)^{\hilite[blue!50]{2}[circle]} +
                 (\Delta z)^{\hilite[violet!45]{2}[circle]}
  \end{align*}
\end{dispListing*}
  \begin{align*}
    (\Delta s)^2 &= -(\Delta t)^2 + (\Delta x)^2 + (\Delta y)^2 + 
                     (\Delta z)^2 \\
    (\Delta s)^2 &= \hilite{-(\Delta t)^2 + (\Delta x)^2}[rounded rectangle] + 
                    (\Delta y)^2 + (\Delta z)^2 \\
    (\Delta s)^2 &= \hilite{-(\Delta t)^2 + (\Delta x)^2}[rectangle] + 
                    (\Delta y)^2 + (\Delta z)^2 \\
    (\Delta s)^2 &= \hilite{-(\Delta t)^2 + (\Delta x)^2}[ellipse] + 
                    (\Delta y)^2 + (\Delta z)^2 \\
    (\Delta s)^{\hilite{2}[circle]} &= \hilite[green]{-}[circle]
                    (\Delta t)^{\hilite[cyan]{2}[circle]}+
      (\Delta x)^{\hilite[orange]{2}[circle]} + 
                    (\Delta y)^{\hilite[blue!50]{2}[circle]} +
                    (\Delta z)^{\hilite[violet!45]{2}[circle]}
  \end{align*}
\begin{dispListing*}{sidebyside=false,listing only}
  \begin{align*}
    \Delta\vec{p} &= \vec{F}_{\sumup{net}}\,\Delta t \\
    \hilite[orange]{\Delta\vec{p}}[circle] &= \vec{F}_{\symup{net}}\,\Delta t \\
    \Delta\vec{p} &= \hilite[yellow!50]{\vec{F}_{\symup{net}}}
                     [rounded rectangle]\,\Delta t \\
    \Delta\vec{p} &= \vec{F}_{\symup{net}}\,\hilite[olive!50]
                     {\Delta t}[rectangle] \\
    \Delta\vec{p} &= \hilite[cyan!50]{\vec{F}_{\symup{net}}\,\Delta t}
                     [ellipse] \\
    \hilite{\Delta\vec{p}}[rectangle] &= \vec{F}_{\symup{net}}\,\Delta t
  \end{align*}
\end{dispListing*}
  \begin{align*}
    \Delta\vec{p} &= \vec{F}_{\symup{net}}\,\Delta t \\
    \hilite[orange]{\Delta\vec{p}}[circle] &= \vec{F}_{\symup{net}}
                     \,\Delta t \\
    \Delta\vec{p} &= \hilite[yellow!50]{\vec{F}_{\symup{net}}}
                     [rounded rectangle]\,\Delta t \\
    \Delta\vec{p} &= \vec{F}_{\symup{net}}\,\hilite[olive!50]{\Delta t}
                     [rectangle] \\
    \Delta\vec{p} &= \hilite[cyan!50]{\vec{F}_{\symup{net}}\,\Delta t}
                     [ellipse] \\
    \hilite{\Delta\vec{p}}[rectangle] &= \vec{F}_{\symup{net}}\,\Delta t
  \end{align*}
%\iffalse
%</example>
%\fi
%
%\iffalse
%<*example>
%\fi
\begin{docCommand}[doc updated = 2021-02-26]{image}{%
    \oarg{options}\marg{caption}\marg{label}\marg{image}
  }%
  Simplified interface for importing an image. The images are treated 
  as floats, so they may not appear at the most logically intuitive 
  place.
\end{docCommand}
\begin{dispListing*}{sidebyside=false,listing only,verbatim ignore percent}
  \image[scale=0.20]{example-image-1x1}
    {Image shown 20 percent actual size.}{reffig1}
\end{dispListing*}
\image[scale=0.20]{example-image-1x1}
    {Image shown 20 percent actual size.}{reffig1}
\begin{dispExample*}{sidebyside=false}
  Figure \ref{reffig1} is nice. 
  It's captioned \nameref{reffig1} and is on page \pageref{reffig1}.
\end{dispExample*}
\begin{dispListing*}{sidebyside=false,listing only,verbatim ignore percent}
  \image[scale=0.20,angle=45]{example-image-1x1}
  {Image shown 20 percent actual size and rotated.}{reffig1}
\end{dispListing*}
\image[scale=0.20,angle=45]{example-image-1x1}
{Image shown 20 percent actual size and rotated.}{reffig2}
\begin{dispExample*}{sidebyside=false}
  Figure \ref{reffig2} is nice. 
  It's captioned \nameref{reffig2} and is on page \pageref{reffig2}.
\end{dispExample*}
%\iffalse
%</example>
%\fi
%
% \newpage
% \section{Commands Specific to \emph{Matter \& Interactions}}
% \mandi\ comes with an accessory package \mandiexp\  
% which includes commands specific to 
% \emph{Matter \& Interactions}\footnote{See 
% \href{https://www.wiley.com/en-us/Matter+and+Interactions%2C+4th+Edition-p-9781118875865}
% {\emph{Matter \& Interactions}} and
% \href{https://matterandinteractions.org/}{https://matterandinteractions.org/}
% for details}. The commands are primarily for typesetting 
% mathematical expressions used in the text. Use of \mandiexp\ is 
% optional and so must be manually loaded by including the line 
% |\usepackage{mandiexp}| in your document's preamble.
% \subsection{The Momentum Principle}
%
%\iffalse
%<*example>
%\fi
\begin{docCommands}[%
    doc parameter = {},%
  ]%
  {%
    {%
      doc name = lhsmomentumprinciple,%
      doc description = {LHS of delta form, bold vectors},%
    },%
    {%
      doc name = rhsmomentumprinciple,%
      doc description = {RHS of delta form, bold vectors},%
    },%
    {%
      doc name = lhsmomentumprincipleupdate,%
      doc description = {LHS of update form, bold vectors},%
    },%
    {%
      doc name = rhsmomentumprincipleupdate,%
      doc description = {RHS of update form, bold vectors},%
    },%
    {%
      doc name = momentumprinciple,%
      doc description = {delta form, bold vectors},%
    },%
    {%
      doc name = momentumprincipleupdate,%
      doc description = {update form, bold vectors},%
    },%
    {%
      doc name = lhsmomentumprinciple*,%
      doc description = {LHS of delta form, arrow vectors},%
    },%
    {%
      doc name = rhsmomentumprinciple*,%
      doc description = {RHS of delta form, arrow vectors},%
    },%
    {%
      doc name = lhsmomentumprincipleupdate*,%
      doc description = {LHS of update form, arrow vectors},%
    },%
    {%
      doc name = rhsmomentumprincipleupdate*,%
      doc description = {RHS of update form, arrow vectors},%
    },%
    {%
      doc name = momentumprinciple*,%
      doc description = {delta form, arrow vectors},%
    },%
    {%
      doc name = momentumprincipleupdate*,%
      doc description = {update form, arrow vectors},%
    },%
  }%
  Variants of command for typesetting the momentum principle.
  Use starred variants to get arrow notation for vectors.
\end{docCommands}
\begin{dispExample}
  \( \lhsmomentumprinciple \)        \\
  \( \rhsmomentumprinciple \)        \\
  \( \lhsmomentumprincipleupdate \)  \\
  \( \rhsmomentumprincipleupdate \)  \\
  \( \momentumprinciple \)           \\
  \( \momentumprincipleupdate \)     \\
  \( \lhsmomentumprinciple* \)       \\
  \( \rhsmomentumprinciple* \)       \\
  \( \lhsmomentumprincipleupdate* \) \\
  \( \rhsmomentumprincipleupdate* \) \\
  \( \momentumprinciple* \)          \\
  \( \momentumprincipleupdate* \)
\end{dispExample}
%\iffalse
%</example>
%\fi
%
% \subsection{The Energy Principle}
%
%\iffalse
%<*example>
%\fi
\begin{docCommands}[%
    doc parameter = {},%
  ]%
  {%
    {%
      doc name = lhsenergyprinciple,%
      doc description = {LHS of delta form},%
    },%
    {%
      doc name = rhsenergyprinciple,%
      doc parameter = \oarg{\(+\)process...},%
      doc description = {RHS of delta form},%
    },%
    {%
      doc name = lhsenergyprincipleupdate,%
      doc description = {LHS of update form},%
    },%
    {%
      doc name = rhsenergyprincipleupdate,%
      doc parameter = \oarg{\(+\)process...},%
      doc description = {RHS of update form},%
    },%
    {%
      doc name = energyprinciple,%
      doc parameter = \oarg{\(+\)process...},%
      doc description = {delta form},%
    },%
    {%
      doc name = energyprincipleupdate,%
      doc parameter = \oarg{\(+\)process...},%
      doc description = {update form},%
    },%
  }%
  Variants of command for typesetting the energy principle.
\end{docCommands}
\begin{dispExample}
  \( \lhsenergyprinciple \)           \\
  \( \rhsenergyprinciple \)           \\
  \( \rhsenergyprinciple[+Q] \)       \\
  \( \energyprinciple \)              \\
  \( \energyprinciple[+Q] \)          \\
  \( \lhsenergyprincipleupdate \)     \\
  \( \rhsenergyprincipleupdate \)     \\
  \( \rhsenergyprincipleupdate[+Q] \) \\
  \( \energyprincipleupdate \)        \\
  \( \energyprincipleupdate[+Q] \)
\end{dispExample}
%\iffalse
%</example>
%\fi
%
% \subsection{The Angular Momentum Principle}
%
%\iffalse
%<*example>
%\fi
\begin{docCommands}[%
    doc parameter = {},%
  ]%
  {%
    {%
      doc name = lhsangularmomentumprinciple,%
      doc description = {LHS of delta form, bold vectors},%
    },%
    {%
      doc name = rhsangularmomentumprinciple,%
      doc description = {RHS of delta form, bold vectors},%
    },%
    {%
      doc name = lhsangularmomentumprincipleupdate,%
      doc description = {LHS of update form, bold vectors},%
    },%
    {%
      doc name = rhsangularmomentumprincipleupdate,%
      doc description = {RHS of update form, bold vectors},%
    },%
    {%
      doc name = angularmomentumprinciple,%
      doc description = {delta form, bold vectors},%
    },%
    {%
      doc name = angularmomentumprincipleupdate,%
      doc description = {update form, bold vectors},%
    },%
    {%
      doc name = lhsangularmomentumprinciple*,%
      doc description = {LHS of delta form, arrow vectors},%
    },%
    {%
      doc name = rhsangularmomentumprinciple*,%
      doc description = {RHS of delta form, arrow vectors},%
    },%
    {%
      doc name = lhsangularmomentumprincipleupdate*,%
      doc description = {LHS of update form, arrow vectors},%
    },%
    {%
      doc name = rhsangularmomentumprincipleupdate*,%
      doc description = {RHS of update form, arrow vectors},%
    },%
    {%
      doc name = angularmomentumprinciple*,%
      doc description = {delta form, arrow vectors},%
    },%
    {%
      doc name = angularmomentumprincipleupdate*,%
      doc description = {update form, arrow vectors},%
    },%
  }%
  Variants of command for typesetting the angularmomentum principle.
  Use starred variants to get arrow notation for vectors.
\end{docCommands}
\begin{dispExample}
  \( \lhsangularmomentumprinciple \)        \\
  \( \rhsangularmomentumprinciple \)        \\
  \( \lhsangularmomentumprincipleupdate \)  \\
  \( \rhsangularmomentumprincipleupdate \)  \\
  \( \angularmomentumprinciple \)           \\
  \( \angularmomentumprincipleupdate \)     \\
  \( \lhsangularmomentumprinciple* \)       \\
  \( \rhsangularmomentumprinciple* \)       \\
  \( \lhsangularmomentumprincipleupdate* \) \\
  \( \rhsangularmomentumprincipleupdate* \) \\
  \( \angularmomentumprinciple* \)          \\
  \( \angularmomentumprincipleupdate* \)
\end{dispExample}
%\iffalse
%</example>
%\fi
%
% \subsection{Other Expressions}
%
%\iffalse
%<*example>
%\fi
\begin{docCommand}[doc new=2021-02-13]{energyof}{\marg{label}\oarg{label}}
    Generic symbol for the energy of some entity.
\end{docCommand}
\begin{dispExample*}{lefthand ratio=0.6}
  \( \energyof{\symup{electron}} \) \\
  \( \energyof{\symup{electron}}[\symup{final}] \)
\end{dispExample*}
%\iffalse
%</example>
%\fi
%
%\iffalse
%<*example>
%\fi
\begin{docCommand}[doc new=2021-02-13]{systemenergy}{\oarg{label}}
    Symbol for system energy.
\end{docCommand}
\begin{dispExample}
  \( \systemenergy \) \\
  \( \systemenergy[\symup{final}] \)
\end{dispExample}
%\iffalse
%</example>
%\fi
%
%\iffalse
%<*example>
%\fi
\begin{docCommand}[doc new=2021-02-13]{particleenergy}{\oarg{label}}
    Symbol for particle energy.
\end{docCommand}
\begin{dispExample}
  \( \particleenergy \) \\
  \( \particleenergy[\symup{final}] \)
\end{dispExample}
%\iffalse
%</example>
%\fi
%
%\iffalse
%<*example>
%\fi
\begin{docCommand}[doc new=2021-02-13]{restenergy}{\oarg{label}}
    Symbol for rest energy.
\end{docCommand}
\begin{dispExample}
  \( \restenergy \) \\
  \( \restenergy[\symup{final}] \)
\end{dispExample}
%\iffalse
%</example>
%\fi
%
%\iffalse
%<*example>
%\fi
\begin{docCommand}[doc new=2021-02-13]{internalenergy}{\oarg{label}}
    Symbol for internal energy.
\end{docCommand}
\begin{dispExample}
  \( \internalenergy \) \\
  \( \internalenergy[\symup{final}] \)
\end{dispExample}
%\iffalse
%</example>
%\fi
%
%\iffalse
%<*example>
%\fi
\begin{docCommand}[doc new=2021-02-13]{chemicalenergy}{\oarg{label}}
    Symbol for chemical energy.
\end{docCommand}
\begin{dispExample}
  \( \chemicalenergy \) \\
  \( \chemicalenergy[\symup{final}] \)
\end{dispExample}
%\iffalse
%</example>
%\fi
%
%\iffalse
%<*example>
%\fi
\begin{docCommand}[doc new=2021-02-13]{thermalenergy}{\oarg{label}}
    Symbol for thermal energy.
\end{docCommand}
\begin{dispExample}
  \( \thermalenergy \) \\
  \( \thermalenergy[\symup{final}] \)
\end{dispExample}
%\iffalse
%</example>
%\fi
%
%\iffalse
%<*example>
%\fi
\begin{docCommand}[doc new=2021-02-13]{photonenergy}{\oarg{label}}
    Symbol for photon energy.
\end{docCommand}
\begin{dispExample}
  \( \photonenergy \) \\
  \( \photonenergy[\symup{final}] \)
\end{dispExample}
%\iffalse
%</example>
%\fi
%
%\iffalse
%<*example>
%\fi
\begin{docCommands}[%
  doc new=2021-02-13,%
  doc parameter = \oarg{label},%
  ]%
  {%
    {%
      doc name = translationalkineticenergy,%
    },%
    {%
      doc name = translationalkineticenergy*,%
    },%
  }%
  Symbol for translational kinetic energy.
  The starred variant gives \(E\) notation.
\end{docCommands}
\begin{dispExample*}{lefthand ratio=0.6}
  \( \translationalkineticenergy \) \\
  \( \translationalkineticenergy[\symup{initial}] \) \\
  \( \translationalkineticenergy* \) \\
  \( \translationalkineticenergy*[\symup{initial}] \)
\end{dispExample*}
%\iffalse
%</example>
%\fi
%
%\iffalse
%<*example>
%\fi
\begin{docCommands}[%
  doc new=2021-02-13,%
  doc parameter = \oarg{label},%
  ]%
  {%
    {%
      doc name = rotationalkineticenergy,%
    },%
    {%
      doc name = rotationalkineticenergy*,%
    },%
  }%
  Symbol for rotational kinetic energy.
  The starred variant gives \(E\) notation.
\end{docCommands}
\begin{dispExample*}{lefthand ratio=0.6}
  \( \rotationalkineticenergy \) \\
  \( \rotationalkineticenergy[\symup{initial}] \) \\
  \( \rotationalkineticenergy* \) \\
  \( \rotationalkineticenergy*[\symup{initial}] \)
\end{dispExample*}
%\iffalse
%</example>
%\fi
%
%\iffalse
%<*example>
%\fi
\begin{docCommands}[%
  doc new=2021-02-13,%
  doc parameter = \oarg{label},%
  ]%
  {%
    {%
      doc name = vibrationalkineticenergy,%
    },%
    {%
      doc name = vibrationalkineticenergy*,%
    },%
  }%
  Symbol for vibrational kinetic energy.
  The starred variant gives \(E\) notation.
\end{docCommands}
\begin{dispExample*}{lefthand ratio=0.6}
  \( \vibrationalkineticenergy \) \\
  \( \vibrationalkineticenergy[\symup{initial}] \) \\
  \( \vibrationalkineticenergy* \) \\
  \( \vibrationalkineticenergy*[\symup{initial}] \)
\end{dispExample*}
%\iffalse
%</example>
%\fi
%
%\iffalse
%<*example>
%\fi
\begin{docCommand}[doc new=2021-02-13]{gravitationalpotentialenergy}
  {\oarg{label}}
    Symbol for gravitational potential energy.
\end{docCommand}
\begin{dispExample*}{lefthand ratio=0.6}
  \( \gravitationalpotentialenergy \) \\
  \( \gravitationalpotentialenergy[\symup{final}] \)
\end{dispExample*}
%\iffalse
%</example>
%\fi
%
%\iffalse
%<*example>
%\fi
\begin{docCommand}[doc new=2021-02-13]{electricpotentialenergy}{\oarg{label}}
    Symbol for electric potential energy.
\end{docCommand}
\begin{dispExample*}{lefthand ratio=0.6}
  \( \electricpotentialenergy \) \\
  \( \electricpotentialenergy[\symup{final}] \)
\end{dispExample*}
%\iffalse
%</example>
%\fi
%
%\iffalse
%<*example>
%\fi
\begin{docCommand}[doc new=2021-02-13]{springpotentialenergy}{\oarg{label}}
    Symbol for spring potential energy.
\end{docCommand}
\begin{dispExample*}{lefthand ratio=0.6}
  \( \springpotentialenergy \) \\
  \( \springpotentialenergy[\symup{final}] \)
\end{dispExample*}
%\iffalse
%</example>
%\fi
%
% \StopEventually{}
%
% \newgeometry{left=0.50in,right=0.50in,top=1.00in,bottom=1.00in}
% \section{Source Code}
%
% \iffalse
%<*package>
% \fi
% Definine the package version and date for global use, exploiting the fact
% that in a \pkg{.sty} file there is now no need for |\makeatletter| and
% |\makeatother|. This simplifies defining internal commands, with |@| 
% in the name, that are not for the user to know about.
%
%    \begin{macrocode}
\def\mandi@Version{3.0.0h}
\def\mandi@Date{2021-03-17}
\NeedsTeXFormat{LaTeX2e}[1999/12/01]
\providecommand\DeclareRelease[3]{}
\providecommand\DeclareCurrentRelease[2]{}
\DeclareRelease{v3.0.0h}{2021-03-17}{mandi.sty}
\DeclareCurrentRelease{v\mandi@Version}{\mandi@Date}
\ProvidesPackage{mandi}
  [\mandi@Date\space v\mandi@Version\space Macros for introductory physics]
%    \end{macrocode}
%
% Define a convenient package version command.
%
%    \begin{macrocode}
\newcommand*{\mandiversion}{v\mandi@Version\space dated \mandi@Date}
%    \end{macrocode}
%
% Set up the fonts to be consistent with ISO 80000-2 notation.
% The \href{https://www.ctan.org/pkg/unicode-math}{\pkg{unicode-math}} package 
% loads the \href{https://www.ctan.org/pkg/fontspec}{\pkg{fontspec}} and 
% \href{https://www.ctan.org/pkg/xparse}{\pkg{xparse}}
% packages. Note that \pkg{xparse} is now part of the \LaTeX\ kernel.
% Because \pkg{unicode-math} is required, all documents using \mandi\ must
% be compiled with an engine that supports Unicode. We recommend \lualatex.
%
%    \begin{macrocode}
\RequirePackage{unicode-math}
\unimathsetup{math-style=ISO}
\unimathsetup{warnings-off={mathtools-colon,mathtools-overbracket}}
\setmathfont[Scale=MatchLowercase]{TeX Gyre DejaVu Math} % single-storey g.
%    \end{macrocode}
%
% Use normal math letters from Latin Modern Math for familiarity with 
% textbooks.
%
%    \begin{macrocode}
\setmathfont[Scale=MatchLowercase,range=it/]{Latin Modern Math}
%    \end{macrocode}
%
% Borrow |mathscr| and |mathbfscr| from XITS Math.\newline
% See \href{https://tex.stackexchange.com/a/120073/218142}
%  {https://tex.stackexchange.com/a/120073/218142}.
%
%    \begin{macrocode}
\setmathfont[Scale=MatchLowercase,range={\mathscr,\mathbfscr}]{XITS Math}
%    \end{macrocode}
%
% Get original and bold |mathcal| fonts.\newline
% See \href{https://tex.stackexchange.com/a/21742/218142}
%  {https://tex.stackexchange.com/a/21742/218142}.
%
%    \begin{macrocode}
\setmathfont[Scale=MatchLowercase,range={\mathcal,\mathbfcal},StylisticSet=1]{XITS Math}
%    \end{macrocode}
%
% Borrow Greek letters from Latin Modern Math.
%
%    \begin{macrocode}
\setmathfont[Scale=MatchLowercase,range=    it/{greek,Greek}]{Latin Modern Math}
\setmathfont[Scale=MatchLowercase,range=  bfit/{greek,Greek}]{Latin Modern Math}
\setmathfont[Scale=MatchLowercase,range=    up/{greek,Greek}]{Latin Modern Math}
\setmathfont[Scale=MatchLowercase,range=  bfup/{greek,Greek}]{Latin Modern Math}
\setmathfont[Scale=MatchLowercase,range=bfsfup/{greek,Greek}]{Latin Modern Math}
%    \end{macrocode}
%
% Load third party packages, documenting why each one is needed.
%
%    \begin{macrocode}
\RequirePackage{amsmath}             % AMS goodness (don't load amssymb or amsfonts)
\RequirePackage[inline]{enumitem}    % needed for physicsproblem environment
\RequirePackage{eso-pic}             % needed for \hilite
\RequirePackage[g]{esvect}           % needed for nice vector arrow, style g
\RequirePackage{pgfopts}             % needed for key-value interface
\RequirePackage{array}               % needed for \checkquantity and \checkconstant
\RequirePackage{iftex}               % needed for requiring LuaLaTeX
\RequirePackage{makebox}             % needed for consistent \dirvect; \makebox
\RequirePackage{mathtools}           % needed for paired delimiters; extends amsmath
\RequirePackage{nicematrix}          % needed for column and row vectors
\RequirePackage[most]{tcolorbox}     % needed for program listings
\RequirePackage{tensor}              % needed for index notation
\RequirePackage{tikz}                % needed for \hilite
\usetikzlibrary{shapes,fit,tikzmark} % needed for \hilite
\RequirePackage{hyperref}            % load last
\RequireLuaTeX                       % require this engine
%    \end{macrocode}
%
% Need to tweak the \href{https://www.ctan.org/pkg/esvect}{\pkg{esvect}} package 
% fonts to get the correct font size. Code provided by |@egreg|.\newline
% See \href{https://tex.stackexchange.com/a/566676}
%  {https://tex.stackexchange.com/a/566676}.
%
%    \begin{macrocode}
\DeclareFontFamily{U}{esvect}{}
\DeclareFontShape{U}{esvect}{m}{n}{%
  <-5.5> vect5
  <5.5-6.5> vect6
  <6.5-7.5> vect7
  <7.5-8.5> vect8
  <8.5-9.5> vect9
  <9.5-> vect10
}{}%
%    \end{macrocode}
%
%    \begin{macrocode}
\directlua{%
 luaotfload.add_colorscheme("colordigits",
   {["8000FF"] = {"one","two","three","four","five","six","seven","eight","nine","zero"}})
}%
\newfontfamily\colordigits{DejaVuSansMono}[RawFeature={color=colordigits}]
%    \end{macrocode}
%
% Set up a color scheme and a new code environment for listings. The new colors 
% are more restful on the eye. All listing commands now use 
% \href{https://www.ctan.org/pkg/tcolorbox}{\pkg{tcolorbox}}.\newline
% See \href{https://tex.stackexchange.com/a/529421/218142}
%  {https://tex.stackexchange.com/a/529421/218142}.
%
%    \begin{macrocode}
\newfontfamily{\gsfontfamily}{DejaVuSansMono}    % new font for listings
\definecolor{gsbggray}     {rgb}{0.90,0.90,0.90} % background gray
\definecolor{gsgray}       {rgb}{0.30,0.30,0.30} % gray
\definecolor{gsgreen}      {rgb}{0.00,0.60,0.00} % green
\definecolor{gsorange}     {rgb}{0.80,0.45,0.12} % orange
\definecolor{gspeach}      {rgb}{1.00,0.90,0.71} % peach
\definecolor{gspearl}      {rgb}{0.94,0.92,0.84} % pearl
\definecolor{gsplum}       {rgb}{0.74,0.46,0.70} % plum
\lstdefinestyle{vpython}{%                       % style for listings
  backgroundcolor=\color{gsbggray},%             % background color
  basicstyle=\colordigits\footnotesize,%         % default style
  breakatwhitespace=true%                        % break at whitespace
  breaklines=true,%                              % break long lines
  captionpos=b,%                                 % position caption
  classoffset=1,%                                % STILL DON'T UNDERSTAND THIS
  commentstyle=\color{gsgray},%                  % font for comments
  deletekeywords={print},%                       % delete keywords from the given language
  emph={self,cls,@classmethod,@property},%       % words to emphasize
  emphstyle=\color{gsorange}\itshape,%           % font for emphasis
  escapeinside={(*@}{@*)},%                      % add LaTeX within your code
  frame=tb,%                                     % frame style
  framerule=2.0pt,%                              % frame thickness
  framexleftmargin=5pt,%                         % extra frame left margin
  %identifierstyle=\sffamily,%                    % style for identifiers
  keywordstyle=\gsfontfamily\color{gsplum},%     % color for keywords
  language=Python,%                              % select language
  linewidth=\linewidth,%                         % width of listings
  morekeywords={%                                % VPython/GlowScript specific keywords
    __future__,abs,acos,align,ambient,angle,append,append_to_caption,%
    append_to_title,arange,arrow,asin,astuple,atan,atan2,attach_arrow,%
    attach_trail,autoscale,axis,background,billboard,bind,black,blue,border,%
    bounding_box,box,bumpaxis,bumpmap,bumpmaps,camera,canvas,caption,capture,%
    ceil,center,clear,clear_trail,click,clone,CoffeeScript,coils,color,combin,%
    comp,compound,cone,convex,cos,cross,curve,cyan,cylinder,data,degrees,del,%
    delete,depth,descender,diff_angle,digits,division,dot,draw_complete,%
    ellipsoid,emissive,end_face_color,equals,explog,extrusion,faces,factorial,%
    False,floor,follow,font,format,forward,fov,frame,gcurve,gdisplay,gdots,%
    get_library,get_selected,ghbars,global,GlowScript,graph,graphs,green,gvbars,%
    hat,headlength,headwidth,height,helix,hsv_to_rgb,index,interval,keydown,%
    keyup,label,length,lights,line,linecolor,linewidth,logx,logy,lower_left,%
    lower_right,mag,mag2,magenta,make_trail,marker_color,markers,material,%
    max,min,mouse,mousedown,mousemove,mouseup,newball,norm,normal,objects,%
    offset,one,opacity,orange,origin,path,pause,pi,pixel_to_world,pixels,plot,%
    points,pos,pow,pps,print,print_function,print_options,proj,purple,pyramid,%
    quad,radians,radius,random,rate,ray,read_local_file,readonly,red,redraw,%
    retain,rgb_to_hsv,ring,rotate,round,scene,scroll,shaftwidth,shape,shapes,%
    shininess,show_end_face,show_start_face,sign,sin,size,size_units,sleep,%
    smooth,space,sphere,sqrt,start,start_face_color,stop,tan,text,textpos,%
    texture,textures,thickness,title,trail_color,trail_object,trail_radius,%
    trail_type,triangle,trigger,True,twist,unbind,up,upper_left,upper_right,%
    userpan,userspin,userzoom,vec,vector,vertex,vertical_spacing,visible,%
    visual,vpython,VPython,waitfor,white,width,world,xtitle,yellow,yoffset,%
    ytitle%
  },%
  morekeywords={print,None,TypeError},%      % additional keywords
  morestring=[b]{"""},%                      % treat triple quotes as strings
  numbers=left,%                             % where to put line numbers
  numbersep=10pt,%                           % how far line numbers are from code
  numberstyle=\bfseries\tiny,%               % set to 'none' for no line numbers
  showstringspaces=false,%                   % show spaces in strings
  showtabs=false,%                           % show tabs within strings
  stringstyle=\gsfontfamily\color{gsgreen},% % color for strings
  upquote=true,%                             % how to typeset quotes
}%
%    \end{macrocode}
%
% Introduce a new, more intelligent \refEnv{glowscriptblock} environment.
%
%    \begin{macrocode}
\NewTCBListing[auto counter,list inside=gsprogs]{glowscriptblock}
  { O{} D(){glowscript.org} m }{%
  breakable,%
  center,%
  code = \newpage,%
  %derivpeach,%
  enhanced,%
  hyperurl interior = https://#2,%
  label = {gs:\thetcbcounter},%
  left = 8mm,%
  list entry = \thetcbcounter~~~~~#3,%
  listing only,%
  listing style = vpython,%
  nameref = {#3},%
  title = \texttt{GlowScript} Program \thetcbcounter: #3,%
  width = 0.9\textwidth,%
  {#1},
}%
%    \end{macrocode}
%
% A new command for generating a list of \GlowScript\ programs.
%
%    \begin{macrocode}
\NewDocumentCommand{\listofglowscriptprograms}{}{\tcblistof[\section*]{gsprogs}
  {List of \texttt{GlowScript} Programs}}%
%    \end{macrocode}
%
% Introduce a new, more intelligent \refCom{vpythonfile} command.
%
%    \begin{macrocode}
\NewTCBInputListing[auto counter,list inside=vpprogs]{\vpythonfile}
  { O{} m m }{%
  breakable,%
  center,%
  code = \newpage,%
  %derivgray,%
  enhanced,%
  hyperurl interior = https://,%
  label = {vp:\thetcbcounter},%
  left = 8mm,%
  list entry = \thetcbcounter~~~~~#3,%
  listing file = {#2},%
  listing only,%
  listing style = vpython,%
  nameref = {#3},%
  title = \texttt{VPython} Program \thetcbcounter: #3,%
  width = 0.9\textwidth,%
  {#1},%
}%
%    \end{macrocode}
%
%  A new command for generating a list of \VPython\ programs.
%
%    \begin{macrocode}
\NewDocumentCommand{\listofvpythonprograms}{}{\tcblistof[\section*]{vpprogs}
  {List of \texttt{VPython} Programs}}%
%    \end{macrocode}
%
% Introduce a new \refCom{glowscriptinline} command.
%
%    \begin{macrocode}
\DeclareTotalTCBox{\glowscriptinline}{ m }{%
  bottom = 0pt,%
  bottomrule = 0.0mm,%
  boxsep = 1.0mm,%
  colback = gsbggray,%
  colframe = gsbggray,%
  left = 0pt,%
  leftrule = 0.0mm,%
  nobeforeafter,%
  right = 0pt,%
  rightrule = 0.0mm,%
  sharp corners,%
  tcbox raise base,%
  top = 0pt,%
  toprule = 0.0mm,%
}{\lstinline[style = vpython]{#1}}%
%    \end{macrocode}
%
% Define \refCom{vpythoninline}, a semantic alias for \VPython\ 
% in-line listings.
%
%    \begin{macrocode}
\NewDocumentCommand{\vpythoninline}{}{\glowscriptinline}%
%    \end{macrocode}
%
% Define units to be used with the unit engine. All single letter macros 
% are now gone. We basically absorbed and adapted the now outdated 
% \href{https://ctan.org/pkg/siunits}{\pkg{SIunits}} package. 
% We make use of |\symup{...}| from the \pkg{unicode-math} package.
%
%    \begin{macrocode}
\NewDocumentCommand{\per}{}{\ensuremath{\,/\,}}
\NewDocumentCommand{\usk}{}{\ensuremath{\,\cdot\,}}
\NewDocumentCommand{\unit}{ m m }{\ensuremath{{#1}\;{#2}}}
\NewDocumentCommand{\ampere}{}{\ensuremath{\symup{A}}}
\NewDocumentCommand{\atomicmassunit}{}{\ensuremath{\symup{u}}}
\NewDocumentCommand{\candela}{}{\ensuremath{\symup{cd}}}
\NewDocumentCommand{\coulomb}{}{\ensuremath{\symup{C}}}
\NewDocumentCommand{\degree}{}{\ensuremath{^{\circ}}}
\NewDocumentCommand{\electronvolt}{}{\ensuremath{\symup{eV}}}
\NewDocumentCommand{\farad}{}{\ensuremath{\symup{F}}}
\NewDocumentCommand{\henry}{}{\ensuremath{\symup{H}}}
\NewDocumentCommand{\hertz}{}{\ensuremath{\symup{Hz}}}
\NewDocumentCommand{\joule}{}{\ensuremath{\symup{J}}}
\NewDocumentCommand{\kelvin}{}{\ensuremath{\symup{K}}}
\NewDocumentCommand{\kilogram}{}{\ensuremath{\symup{kg}}}
\NewDocumentCommand{\lightspeed}{}{\ensuremath{\symup{c}}}
\NewDocumentCommand{\meter}{}{\ensuremath{\symup{m}}}
\NewDocumentCommand{\metre}{}{\meter}
\NewDocumentCommand{\mole}{}{\ensuremath{\symup{mol}}}
\NewDocumentCommand{\newton}{}{\ensuremath{\symup{N}}}
\NewDocumentCommand{\ohm}{}{\ensuremath{\symup\Omega}}
\NewDocumentCommand{\pascal}{}{\ensuremath{\symup{Pa}}}
\NewDocumentCommand{\radian}{}{\ensuremath{\symup{rad}}}
\NewDocumentCommand{\second}{}{\ensuremath{\symup{s}}}
\NewDocumentCommand{\siemens}{}{\ensuremath{\symup{S}}}
\NewDocumentCommand{\steradian}{}{\ensuremath{\symup{sr}}}
\NewDocumentCommand{\tesla}{}{\ensuremath{\symup{T}}}
\NewDocumentCommand{\volt}{}{\ensuremath{\symup{V}}}
\NewDocumentCommand{\watt}{}{\ensuremath{\symup{W}}}
\NewDocumentCommand{\weber}{}{\ensuremath{\symup{Wb}}}
\NewDocumentCommand{\tothetwo}{}{\ensuremath{^2}}             % postfix  2
\NewDocumentCommand{\tothethree}{}{\ensuremath{^3}}           % postfix  3
\NewDocumentCommand{\tothefour}{}{\ensuremath{^4}}            % postfix  4
\NewDocumentCommand{\inverse}{}{\ensuremath{^{-1}}}           % postfix -1
\NewDocumentCommand{\totheinversetwo}{}{\ensuremath{^{-2}}}   % postfix -2
\NewDocumentCommand{\totheinversethree}{}{\ensuremath{^{-3}}} % postfix -3
\NewDocumentCommand{\totheinversefour}{}{\ensuremath{^{-4}}}  % postfix -4
\NewDocumentCommand{\emptyunit}{}{\ensuremath{\mdlgwhtsquare}}
%    \end{macrocode}
%
% The core unit engine has been completely rewritten in 
% \href{https://www.ctan.org/pkg/expl3}{\pkg{expl3}} for both clarity and power.
%
% Generic internal selectors.
%
%    \begin{macrocode}
\newcommand*{\mandi@selectunits}{}
\newcommand*{\mandi@selectprecision}{}
%    \end{macrocode}
%
% Specific internal selectors.
%
%    \begin{macrocode}
\newcommand*{\mandi@selectapproximate}[2]{#1}    % really \@firstoftwo
\newcommand*{\mandi@selectprecise}[2]{#2}        % really \@secondoftwo
\newcommand*{\mandi@selectbaseunits}[3]{#1}      % really \@firstofthree
\newcommand*{\mandi@selectderivedunits}[3]{#2}   % really \@secondofthree
\newcommand*{\mandi@selectalternateunits}[3]{#3} % really \@thirdofthree
%    \end{macrocode}
%
% Document level global switches.
%
%    \begin{macrocode}
\NewDocumentCommand{\alwaysusebaseunits}{}
  {\renewcommand*{\mandi@selectunits}{\mandi@selectbaseunits}}%
\NewDocumentCommand{\alwaysusederivedunits}{}
  {\renewcommand*{\mandi@selectunits}{\mandi@selectderivedunits}}%
\NewDocumentCommand{\alwaysusealternateunits}{}
  {\renewcommand*{\mandi@selectunits}{\mandi@selectalternateunits}}%
\NewDocumentCommand{\alwaysuseapproximateconstants}{}
  {\renewcommand*{\mandi@selectprecision}{\mandi@selectapproximate}}%
\NewDocumentCommand{\alwaysusepreciseconstants}{}
  {\renewcommand*{\mandi@selectprecision}{\mandi@selectprecise}}%
%    \end{macrocode}
%
% Document level localized variants.
%
%    \begin{macrocode}
\NewDocumentCommand{\hereusebaseunits}{ m }{\begingroup\alwaysusebaseunits#1\endgroup}%
\NewDocumentCommand{\hereusederivedunits}{ m }{\begingroup\alwaysusederivedunits#1\endgroup}%
\NewDocumentCommand{\hereusealternateunits}{ m }{\begingroup\alwaysusealternateunits#1\endgroup}%
\NewDocumentCommand{\hereuseapproximateconstants}{ m }{\begingroup\alwaysuseapproximateconstants#1\endgroup}%
\NewDocumentCommand{\hereusepreciseconstants}{ m }{\begingroup\alwaysusepreciseconstants#1\endgroup}%
%    \end{macrocode}
%
% Document level environments.
%
%    \begin{macrocode}
\NewDocumentEnvironment{usebaseunits}{}{\alwaysusebaseunits}{}%
\NewDocumentEnvironment{usederivedunits}{}{\alwaysusederivedunits}{}%
\NewDocumentEnvironment{usealternateunits}{}{\alwaysusealternateunits}{}%
\NewDocumentEnvironment{useapproximateconstants}{}{\alwaysuseapproximateconstants}{}%
\NewDocumentEnvironment{usepreciseconstants}{}{\alwaysusepreciseconstants}{}%
%    \end{macrocode}
%
% Defining a new scalar quantity. I am very much aware that this family of commands
% doesn't yet correctly abide by the \LaTeX3 concept of separating document commands
% from the programming layer. The problem is that current documentation is not
% completely understandable to me and getting help is difficult for non-experts.
%
%    \begin{macrocode}
\NewDocumentCommand{\newscalarquantity}{ m m O{#2} O{#2} }{%
  \expandafter\newcommand\csname #1\endcsname[1]{##1\,\mandi@selectunits{#2}{#3}{#4}}%
  \expandafter\newcommand\csname #1value\endcsname[1]{##1}%
  \expandafter\newcommand\csname #1baseunits\endcsname[1]{##1\,\mandi@selectbaseunits{#2}{#3}{#4}}%
  \expandafter\newcommand\csname #1derivedunits\endcsname[1]{##1\,\mandi@selectderivedunits{#2}{#3}{#4}}%
  \expandafter\newcommand\csname #1alternateunits\endcsname[1]{##1\,\mandi@selectalternateunits{#2}{#3}{#4}}%
  \expandafter\newcommand\csname #1onlybaseunits\endcsname{\mandi@selectbaseunits{#2}{#3}{#4}}%
  \expandafter\newcommand\csname #1onlyderivedunits\endcsname{\mandi@selectderivedunits{#2}{#3}{#4}}%
  \expandafter\newcommand\csname #1onlyalternateunits\endcsname{\mandi@selectalternateunits{#2}{#3}{#4}}%
}%
%    \end{macrocode}
%
% Redefining a new scalar quantity.
%
%    \begin{macrocode}
\NewDocumentCommand{\renewscalarquantity}{ m m O{#2} O{#2} }{%
  \expandafter\renewcommand\csname #1\endcsname[1]{##1\,\mandi@selectunits{#2}{#3}{#4}}%
  \expandafter\renewcommand\csname #1value\endcsname[1]{##1}%
  \expandafter\renewcommand\csname #1baseunits\endcsname[1]{##1\,\mandi@selectbaseunits{#2}{#3}{#4}}%
  \expandafter\renewcommand\csname #1derivedunits\endcsname[1]{##1\,\mandi@selectderivedunits{#2}{#3}{#4}}%
  \expandafter\renewcommand\csname #1alternateunits\endcsname[1]{##1\,\mandi@selectalternateunits{#2}{#3}{#4}}%
  \expandafter\renewcommand\csname #1onlybaseunits\endcsname{\mandi@selectbaseunits{#2}{#3}{#4}}%
  \expandafter\renewcommand\csname #1onlyderivedunits\endcsname{\mandi@selectderivedunits{#2}{#3}{#4}}%
  \expandafter\renewcommand\csname #1onlyalternateunits\endcsname{\mandi@selectalternateunits{#2}{#3}{#4}}%
}%
%    \end{macrocode}
%
% Defining a new vector quantity. Note that a corresponding scalar is also defined.
%
%    \begin{macrocode}
\NewDocumentCommand{\newvectorquantity}{ m m O{#2} O{#2} }{%
  \newscalarquantity{#1}{#2}[#3][#4]
  \expandafter\newcommand\csname vector#1\endcsname[1]{\expandafter\csname #1\endcsname{\mivector{##1}}}%
  \expandafter\newcommand\csname #1vector\endcsname[1]{\expandafter\csname #1\endcsname{\mivector{##1}}}%
}%
%    \end{macrocode}
%
% Redefining a new vector quantity. Note that a corresponding scalar is also redefined.
%
%    \begin{macrocode}
\NewDocumentCommand{\renewvectorquantity}{ m m O{#2} O{#2} }{%
  \renewscalarquantity{#1}{#2}[#3][#4]
  \expandafter\renewcommand\csname vector#1\endcsname[1]{\expandafter\csname #1\endcsname{\mivector{##1}}}%
  \expandafter\renewcommand\csname #1vector\endcsname[1]{\expandafter\csname #1\endcsname{\mivector{##1}}}%
}%
%    \end{macrocode}
%
% Defining a new physical constant.
%
%    \begin{macrocode}
\NewDocumentCommand{\newphysicalconstant}{ m m m m m O{#5} O{#5} }{%
  \expandafter\newcommand\csname #1\endcsname
    {\mandi@selectprecision{#3}{#4}\,\mandi@selectunits{#5}{#6}{#7}}%
  \expandafter\newcommand\csname #1mathsymbol\endcsname{\ensuremath{#2}}%
  \expandafter\newcommand\csname #1approximatevalue\endcsname{\ensuremath{#3}}%
  \expandafter\newcommand\csname #1precisevalue\endcsname{\ensuremath{#4}}%
  \expandafter\newcommand\csname #1baseunits\endcsname
    {\mandi@selectprecision{#3}{#4}\,\mandi@selectbaseunits{#5}{#6}{#7}}%
  \expandafter\newcommand\csname #1derivedunits\endcsname
    {\mandi@selectprecision{#3}{#4}\,\mandi@selectderivedunits{#5}{#6}{#7}}%
  \expandafter\newcommand\csname #1alternateunits\endcsname
    {\mandi@selectprecision{#3}{#4}\,\mandi@selectalternateunits{#5}{#6}{#7}}%
  \expandafter\newcommand\csname #1onlybaseunits\endcsname
    {\mandi@selectbaseunits{#5}{#6}{#7}}%
  \expandafter\newcommand\csname #1onlyderivedunits\endcsname
    {\mandi@selectderivedunits{#5}{#6}{#7}}%
  \expandafter\newcommand\csname #1onlyalternateunits\endcsname
    {\mandi@selectalternateunits{#5}{#6}{#7}}%
}%
%    \end{macrocode}
%
% Redefining a new physical constant.
%
%    \begin{macrocode}
\NewDocumentCommand{\renewphysicalconstant}{ m m m m m O{#5} O{#5} }{%
  \expandafter\renewcommand\csname #1\endcsname
    {\mandi@selectprecision{#3}{#4}\,\mandi@selectunits{#5}{#6}{#7}}%
  \expandafter\renewcommand\csname #1mathsymbol\endcsname{\ensuremath{#2}}%
  \expandafter\renewcommand\csname #1approximatevalue\endcsname{\ensuremath{#3}}%
  \expandafter\renewcommand\csname #1precisevalue\endcsname{\ensuremath{#4}}%
  \expandafter\renewcommand\csname #1baseunits\endcsname
    {\mandi@selectprecision{#3}{#4}\,\mandi@selectbaseunits{#5}{#6}{#7}}%
  \expandafter\renewcommand\csname #1derivedunits\endcsname
    {\mandi@selectprecision{#3}{#4}\,\mandi@selectderivedunits{#5}{#6}{#7}}%
  \expandafter\renewcommand\csname #1alternateunits\endcsname
    {\mandi@selectprecision{#3}{#4}\,\mandi@selectalternateunits{#5}{#6}{#7}}%
  \expandafter\renewcommand\csname #1onlybaseunits\endcsname
    {\mandi@selectbaseunits{#5}{#6}{#7}}%
  \expandafter\renewcommand\csname #1onlyderivedunits\endcsname
    {\mandi@selectderivedunits{#5}{#6}{#7}}%
  \expandafter\renewcommand\csname #1onlyalternateunits\endcsname
    {\mandi@selectalternateunits{#5}{#6}{#7}}%
}%
%    \end{macrocode}
%
% \mandi\ now has a key-value interface, implemented with 
% \href{https://www.ctan.org/pkg/pgfopts}{\pkg{pgfopts}} and 
% \href{https://www.ctan.org/pkg/pgfkeys}{\pkg{pgfkeys}}.
% There are two options:\newline
% \refKey{units}, with values \docValue{base}, \docValue{derived}, or 
% \docValue{alternate} selects the default form of units\newline
% \refKey{preciseconstants}, with values \docValue{true} and 
% \docValue{false}, selects precise numerical values for constants 
% rather than approximate values. 
%
% First, define the keys. The key handlers require certain commands defined 
% by the unit engine, and thus must be defined and processed after the unit 
% engine code.
%
%    \begin{macrocode}
\newif\ifusingpreciseconstants
\pgfkeys{%
  /mandi/options/.cd,
  initial@setup/.style={%
    /mandi/options/buffered@units/.initial=alternate,%
  },%
  initial@setup,%
  preciseconstants/.is if=usingpreciseconstants,%
  units/.is choice,%
  units/.default=derived,%
  units/alternate/.style={/mandi/options/buffered@units=alternate},%
  units/base/.style={/mandi/options/buffered@units=base},%
  units/derived/.style={/mandi/options/buffered@units=derived},%
}%
%    \end{macrocode}
%
% Process the options.
%
%    \begin{macrocode}
\ProcessPgfPackageOptions{/mandi/options}
%    \end{macrocode}
%
% Write a banner to the console showing the options in use.
% The value of the \refKey{units} key is used in situ to set
% the default units.
%
%    \begin{macrocode}
\newcommand*{\mandi@linetwo}{\typeout{mandi: Loadtime options...}}
\newcommand*{\mandi@do@setup}{%
  \typeout{}%
  \typeout{mandi: You are using mandi \mandiversion.}%
  \mandi@linetwo
  \csname alwaysuse\pgfkeysvalueof{/mandi/options/buffered@units}units\endcsname%
  \typeout{mandi: You will get \pgfkeysvalueof{/mandi/options/buffered@units}\space units.}%
  \ifusingpreciseconstants
    \alwaysusepreciseconstants
    \typeout{mandi: You will get precise constants.}%
  \else
    \alwaysuseapproximateconstants
    \typeout{mandi: You will get approximate constants.}%
  \fi
  \typeout{}%
}%
\mandi@do@setup
%    \end{macrocode}
%
% Define a setup command that overrides the loadtime options
% when called with new options. A new banner is written to the console.
%
%    \begin{macrocode}
\NewDocumentCommand{\mandisetup}{ m }{%
  \IfValueT{#1}{%
    \pgfqkeys{/mandi/options}{#1}
    \renewcommand*{\mandi@linetwo}{\typeout{mandi: mandisetup options...}}
    \mandi@do@setup
  }%
}%
%    \end{macrocode}
%
% Define every quantity we need in introductory physics, alphabetically
% for convenience. This is really the core feature of \mandi\ that no other
% package offers. There are commands for quantities that have no dimensions 
% or units, and these quantities are defined for semantic completeness.
%
%    \begin{macrocode}
\newvectorquantity{acceleration}%
  {\meter\usk\second\totheinversetwo}%
  [\newton\per\kilogram]%
  [\meter\per\second\tothetwo]%
\newscalarquantity{amount}%
  {\mole}%
\newvectorquantity{angularacceleration}%
  {\radian\usk\second\totheinversetwo}%
  [\radian\per\second\tothetwo]%
  [\radian\per\second\tothetwo]%
\newscalarquantity{angularfrequency}%
  {\radian\usk\second\inverse}%
  [\radian\per\second]%
  [\radian\per\second]%
%\ifmandi@rotradians
%  \newphysicalquantity{angularimpulse}%
%    {\meter\tothetwo\usk\kilogram\usk\second\inverse\usk\radian\inverse}%
%    [\joule\usk\second\per\radian]%
%    [\newton\usk\meter\usk\second\per\radian]%
%  \newphysicalquantity{angularmomentum}%
%    {\meter\tothetwo\usk\kilogram\usk\second\inverse\usk\radian\inverse}%
%    [\kilogram\usk\meter\tothetwo\per(\second\usk\radian)]%
%    [\newton\usk\meter\usk\second\per\radian]%
%\else
  \newvectorquantity{angularimpulse}%
    {\meter\tothetwo\usk\kilogram\usk\second\inverse}%
    [\kilogram\usk\meter\tothetwo\per\second]% % also \joule\usk\second
    [\kilogram\usk\meter\tothetwo\per\second]% % also \newton\usk\meter\usk\second
  \newvectorquantity{angularmomentum}%
    {\meter\tothetwo\usk\kilogram\usk\second\inverse}%
    [\kilogram\usk\meter\tothetwo\per\second]% % also \joule\usk\second
    [\kilogram\usk\meter\tothetwo\per\second]% % also \newton\usk\meter\usk\second
%\fi
\newvectorquantity{angularvelocity}%
  {\radian\usk\second\inverse}%
  [\radian\per\second]%
  [\radian\per\second]%
\newscalarquantity{area}%
  {\meter\tothetwo}%
\newscalarquantity{areamassdensity}%
  {\meter\totheinversetwo\usk\kilogram}%
  [\kilogram\per\meter\tothetwo]%
  [\kilogram\per\meter\tothetwo]%
\newscalarquantity{areachargedensity}%
  {\meter\totheinversetwo\usk\second\usk\ampere}%
  [\coulomb\per\meter\tothetwo]%
  [\coulomb\per\meter\tothetwo]%
\newscalarquantity{capacitance}%
  {\meter\totheinversetwo\usk\kilogram\inverse\usk\second\tothefour\usk\ampere\tothetwo}%
  [\farad]%
  [\coulomb\per\volt]% % also \coulomb\tothetwo\per\newton\usk\meter, \second\per\ohm
\newscalarquantity{charge}%
  {\ampere\usk\second}%
  [\coulomb]%
  [\coulomb]% % also \farad\usk\volt
\newvectorquantity{cmagneticfield}%
  {\meter\usk\kilogram\usk\second\totheinversethree\usk\ampere\inverse}%
  [\volt\per\meter]%
  [\newton\per\coulomb]%
\newscalarquantity{conductance}%
  {\meter\totheinversetwo\usk\kilogram\inverse\usk\second\tothethree\usk\ampere\tothetwo}%
  [\siemens]%
  [\ampere\per\volt]%
\newscalarquantity{conductivity}%
  {\meter\totheinversethree\usk\kilogram\inverse\usk\second\tothethree\usk\ampere\tothetwo}%
  [\siemens\per\meter]%
  [(\ampere\per\meter\tothetwo)\per(\volt\per\meter)]%
\newscalarquantity{conventionalcurrent}%
  {\ampere}%
  [\coulomb\per\second]%
  [\ampere]%
\newscalarquantity{current}%
  {\ampere}%
\newscalarquantity{currentdensity}%
  {\meter\totheinversetwo\usk\ampere}%
  [\coulomb\usk\second\per\meter\tothetwo]%
  [\ampere\per\meter\tothetwo]%
\newscalarquantity{dielectricconstant}%
  {}%
\newvectorquantity{displacement}%
  {\meter}
\newscalarquantity{duration}%
  {\second}%
\newvectorquantity{electricdipolemoment}%
  {\meter\usk\second\usk\ampere}%
  [\coulomb\usk\meter]%
  [\coulomb\usk\meter]%
\newvectorquantity{electricfield}%
  {\meter\usk\kilogram\usk\second\totheinversethree\usk\ampere\inverse}%
  [\volt\per\meter]%
  [\newton\per\coulomb]%
\newscalarquantity{electricflux}%
  {\meter\tothethree\usk\kilogram\usk\second\totheinversethree\usk\ampere\inverse}%
  [\volt\usk\meter]%
  [\newton\usk\meter\tothetwo\per\coulomb]%
\newscalarquantity{electricpotential}%
  {\meter\tothetwo\usk\kilogram\usk\second\totheinversethree\usk\ampere\inverse}%
  [\volt]%
  [\joule\per\coulomb]%
\newscalarquantity{electroncurrent}%
  {\second\inverse}%
  [\ensuremath{\symup{e}}\per\second]%
  [\ensuremath{\symup{e}}\per\second]%
\newscalarquantity{emf}%
  {\meter\tothetwo\usk\kilogram\usk\second\totheinversethree\usk\ampere\inverse}%
  [\volt]%
  [\joule\per\coulomb]%
\newscalarquantity{energy}%
  {\meter\tothetwo\usk\kilogram\usk\second\totheinversetwo}%
  [\joule]% % also \newton\usk\meter
  [\joule]%
\newscalarquantity{energydensity}%
  {\meter\inverse\usk\kilogram\usk\second\totheinversetwo}%
  [\joule\per\meter\tothethree]%
  [\joule\per\meter\tothethree]%
\newscalarquantity{energyflux}%
  {\kilogram\usk\second\totheinversethree}%
  [\watt\per\meter\tothetwo]%
  [\watt\per\meter\tothetwo]%
\newscalarquantity{entropy}%
  {\meter\tothetwo\usk\kilogram\usk\second\totheinversetwo\usk\kelvin\inverse}%
  [\joule\per\kelvin]%
  [\joule\per\kelvin]%
\newvectorquantity{force}%
  {\meter\usk\kilogram\usk\second\totheinversetwo}%
  [\newton]%
  [\newton]% % also \kilogram\usk\meter\per\second\tothetwo
\newscalarquantity{frequency}%
  {\second\inverse}%
  [\hertz]%
  [\hertz]%
\newvectorquantity{gravitationalfield}%
  {\meter\usk\second\totheinversetwo}%
  [\newton\per\kilogram]%
  [\newton\per\kilogram]%
\newscalarquantity{gravitationalpotential}%
  {\meter\tothetwo\usk\second\totheinversetwo}%
  [\joule\per\kilogram]%
  [\joule\per\kilogram]%
\newvectorquantity{impulse}%
  {\meter\usk\kilogram\usk\second\inverse}%
  [\newton\usk\second]%
  [\newton\usk\second]%
\newscalarquantity{indexofrefraction}%
  {}%
\newscalarquantity{inductance}%
  {\meter\tothetwo\usk\kilogram\usk\second\totheinversetwo\usk\ampere\totheinversetwo}%
  [\henry]%
  [\volt\usk\second\per\ampere]% % also \square\meter\usk\kilogram\per\coulomb\tothetwo, \Wb\per\ampere
\newscalarquantity{linearchargedensity}%
  {\meter\inverse\usk\second\usk\ampere}%
  [\coulomb\per\meter]%
  [\coulomb\per\meter]%
\newscalarquantity{linearmassdensity}%
  {\meter\inverse\usk\kilogram}%
  [\kilogram\per\meter]%
  [\kilogram\per\meter]%
\newscalarquantity{luminous}%
  {\candela}%
\newscalarquantity{magneticcharge}%
  {\meter\usk\ampere}%
\newvectorquantity{magneticdipolemoment}%
  {\meter\tothetwo\usk\ampere}%
  [\ampere\usk\meter\tothetwo]%
  [\joule\per\tesla]%
\newvectorquantity{magneticfield}%
  {\kilogram\usk\second\totheinversetwo\usk\ampere\inverse}%
  [\tesla]%
  [\newton\per\coulomb\usk(\meter\per\second)]% % also \Wb\per\meter\tothetwo
\newscalarquantity{magneticflux}%
  {\meter\tothetwo\usk\kilogram\usk\second\totheinversetwo\usk\ampere\inverse}%
  [\tesla\usk\meter\tothetwo]%
  [\volt\usk\second]% % also \Wb and \joule\per\ampere
\newscalarquantity{mass}%
  {\kilogram}%
\newscalarquantity{mobility}%
  {\meter\tothetwo\usk\kilogram\usk\second\totheinversefour\usk\ampere\inverse}%
  [\meter\tothetwo\per\volt\usk\second]%
  [(\meter\per\second)\per(\newton\per\coulomb)]%
\newscalarquantity{momentofinertia}%
  {\meter\tothetwo\usk\kilogram}%
  [\joule\usk\second\tothetwo]%
  [\kilogram\usk\meter\tothetwo]%
\newvectorquantity{momentum}%
  {\meter\usk\kilogram\usk\second\inverse}%
  [\newton\usk\second]%
  [\kilogram\usk\meter\per\second]%
\newvectorquantity{momentumflux}%
  {\meter\inverse\usk\kilogram\usk\second\totheinversetwo}%
  [\newton\per\meter\tothetwo]%
  [\newton\per\meter\tothetwo]%
\newscalarquantity{numberdensity}%
  {\meter\totheinversethree}%
  [\per\meter\tothethree]%
  [\per\meter\tothethree]%
\newscalarquantity{permeability}%
  {\meter\usk\kilogram\usk\second\totheinversetwo\usk\ampere\totheinversetwo}%
  [\tesla\usk\meter\per\ampere]%
  [\henry\per\meter]%
\newscalarquantity{permittivity}%
  {\meter\totheinversethree\usk\kilogram\inverse\usk\second\totheinversefour\usk\ampere\tothetwo}%
  [\farad\per\meter]%
  [\coulomb\tothetwo\per\newton\usk\meter\tothetwo]%
\newscalarquantity{planeangle}%
  {\meter\usk\meter\inverse}%
  [\radian]%
  [\radian]%
\newscalarquantity{polarizability}%
  {\kilogram\inverse\usk\second\tothefour\usk\ampere\tothetwo}%
  [\coulomb\usk\meter\tothetwo\per\volt]%
  [\coulomb\usk\meter\per(\newton\per\coulomb)]%
\newscalarquantity{power}%
  {\meter\tothetwo\usk\kilogram\usk\second\totheinversethree}%
  [\watt]%
  [\joule\per\second]%
\newvectorquantity{poynting}%
  {\kilogram\usk\second\totheinversethree}%
  [\watt\per\meter\tothetwo]%
  [\watt\per\meter\tothetwo]%
\newscalarquantity{pressure}%
  {\meter\inverse\usk\kilogram\usk\second\totheinversetwo}%
  [\pascal]%
  [\newton\per\meter\tothetwo]%
\newscalarquantity{relativepermeability}
  {}%
\newscalarquantity{relativepermittivity}%
  {}%
\newscalarquantity{resistance}%
  {\meter\tothetwo\usk\kilogram\usk\second\totheinversethree\usk\ampere\totheinversetwo}%
  [\volt\per\ampere]%
  [\ohm]%
\newscalarquantity{resistivity}%
  {\meter\tothethree\usk\kilogram\usk\second\totheinversethree\usk\ampere\totheinversetwo}%
  [\ohm\usk\meter]%
  [(\volt\per\meter)\per(\ampere\per\meter\tothetwo)]%
\newscalarquantity{solidangle}%
  {\meter\tothetwo\usk\meter\totheinversetwo}%
  [\steradian]%
  [\steradian]%
\newscalarquantity{specificheatcapacity}%
  {\meter\tothetwo\usk\second\totheinversetwo\usk\kelvin\inverse}%
  [\joule\per\kelvin\usk\kilogram]%
  [\joule\per\kelvin\usk\kilogram]
\newscalarquantity{springstiffness}%
  {\kilogram\usk\second\totheinversetwo}%
  [\newton\per\meter]%
  [\newton\per\meter]%
\newscalarquantity{springstretch}% % This is really just a displacement.
  {\meter}%
\newscalarquantity{stress}%
  {\meter\inverse\usk\kilogram\usk\second\totheinversetwo}%
  [\pascal]%
  [\newton\per\meter\tothetwo]%
\newscalarquantity{strain}%
  {}%
\newscalarquantity{temperature}%
  {\kelvin}%
%\ifmandi@rotradians
%  \newphysicalquantity{torque}%
%    {\meter\tothetwo\usk\kilogram\usk\second\totheinversetwo\usk\radian\inverse}%
%    [\newton\usk\meter\per\radian]%
%    [\newton\usk\meter\per\radian]%
%\else
  \newvectorquantity{torque}%
    {\meter\tothetwo\usk\kilogram\usk\second\totheinversetwo}%
    [\newton\usk\meter]%
    [\newton\usk\meter]%
%\fi
\newvectorquantity{velocity}%
  {\meter\usk\second\inverse}%
  [\meter\per\second]%
  [\meter\per\second]%
\newvectorquantity{velocityc}%
  {\lightspeed}%
  []%
  [\lightspeed]%
\newscalarquantity{volume}%
  {\meter\tothethree}%
\newscalarquantity{volumechargedensity}%
  {\meter\totheinversethree\usk\second\usk\ampere}%
  [\coulomb\per\meter\tothethree]%
  [\coulomb\per\meter\tothethree]%
\newscalarquantity{volumemassdensity}%
  {\meter\totheinversethree\usk\kilogram}%
  [\kilogram\per\meter\tothethree]%
  [\kilogram\per\meter\tothethree]%
\newscalarquantity{wavelength}% % This is really just a displacement.
  {\meter}%
\newvectorquantity{wavenumber}%
  {\meter\inverse}%
  [\per\meter]%
  [\per\meter]%
\newscalarquantity{work}%
  {\meter\tothetwo\usk\kilogram\usk\second\totheinversetwo}%
  [\joule]%
  [\newton\usk\meter]%
\newscalarquantity{youngsmodulus}% % This is really just a stress.
  {\meter\inverse\usk\kilogram\usk\second\totheinversetwo}%
  [\pascal]%
  [\newton\per\meter\tothetwo]%
%    \end{macrocode}
%
% Define physical constants for introductory physics, again alphabetically
% for convenience.
%
%    \begin{macrocode}
\newphysicalconstant{avogadro}%
  {\symup{N_A}}%
  {6\timestento{23}}{6.02214076\timestento{23}}%
  {\mole\inverse}%
  [\per\mole]%
  [\per\mole]%
\newphysicalconstant{biotsavartconstant}% % alias for \mzofp
  {\symup{\frac{\mu_o}{4\pi}}}%
  {\tento{-7}}{\tento{-7}}%
  {\meter\usk\kilogram\usk\second\totheinversetwo\usk\ampere\totheinversetwo}%
  [\henry\per\meter]%
  [\tesla\usk\meter\per\ampere]%
\newphysicalconstant{bohrradius}%
  {\symup{a_o}}%
  {5.3\timestento{-11}}{5.2917721067\timestento{-11}}%
  {\meter}%
\newphysicalconstant{boltzmann}%
  {\symup{k_B}}%
  {1.4\timestento{-23}}{1.380649\timestento{-23}}%
  {\meter\tothetwo\usk\kilogram\usk\second\totheinversetwo\usk\kelvin\inverse}%
  [\joule\per\kelvin]%
  [\joule\per\kelvin]%
\newphysicalconstant{coulombconstant}% % alias for \oofpez
  {\symup{\frac{1}{4\pi\epsilon_o}}}%
  {9\timestento{9}}{8.9875517873681764\timestento{9}}%
  {\meter\tothethree\usk\kilogram\usk\second\totheinversefour\usk\ampere\totheinversetwo}%
  [\meter\per\farad]%
  [\newton\usk\meter\tothetwo\per\coulomb\tothetwo]%
\newphysicalconstant{earthmass}%
  {\symup{M_{Earth}}}%
  {6.0\timestento{24}}{5.97237\timestento{24}}%
  {\kilogram}%
\newphysicalconstant{earthmoondistance}%
  {\symup{d_{EM}}}%
  {3.8\timestento{8}}{3.81550\timestento{8}}%
  {\meter}%
\newphysicalconstant{earthradius}%
  {\symup{R_{Earth}}}%
  {6.4\timestento{6}}{6.371\timestento{6}}%
  {\meter}%
\newphysicalconstant{earthsundistance}%
  {\symup{d_{ES}}}%
  {1.5\timestento{11}}{1.496\timestento{11}}%
  {\meter}%
\newphysicalconstant{electroncharge}%
  {\symup{q_e}}%
  {-\elementarychargeapproximatevalue}{-\elementarychargeprecisevalue}%
  {\ampere\usk\second}%
  [\coulomb]%
  [\coulomb]%
\newphysicalconstant{electronCharge}%
  {\symup{Q_e}}%
  {-\elementarychargeapproximatevalue}{-\elementarychargeprecisevalue}%
  {\ampere\usk\second}%
  [\coulomb]%
  [\coulomb]%
\newphysicalconstant{electronmass}%
  {\symup{m_e}}%
  {9.1\timestento{-31}}{9.10938356\timestento{-31}}%
  {\kilogram}%
\newphysicalconstant{elementarycharge}%
  {\symup{e}}%
  {1.6\timestento{-19}}{1.602176634\timestento{-19}}%
  {\ampere\usk\second}%
  [\coulomb]%
  [\coulomb]%
\newphysicalconstant{finestructure}%
  {\symup{\alpha}}%
  {\frac{1}{137}}{7.2973525664\timestento{-3}}%
  {}%
\newphysicalconstant{hydrogenmass}%
  {\symup{m_H}}%
  {1.7\timestento{-27}}{1.6737236\timestento{-27}}%
  {\kilogram}%
\newphysicalconstant{moonearthdistance}%
  {\symup{d_{ME}}}%
  {3.8\timestento{8}}{3.81550\timestento{8}}%
  {\meter}%
\newphysicalconstant{moonmass}%
  {\symup{M_{Moon}}}%
  {7.3\timestento{22}}{7.342\timestento{22}}%
  {\kilogram}%
\newphysicalconstant{moonradius}%
  {\symup{R_{Moon}}}%
  {1.7\timestento{6}}{1.7371\timestento{6}}%
  {\meter}%
\newphysicalconstant{mzofp}%
  {\symup{\frac{\mu_o}{4\pi}}}%
  {\tento{-7}}{\tento{-7}}%
  {\meter\usk\kilogram\usk\second\totheinversetwo\usk\ampere\totheinversetwo}%
  [\henry\per\meter]%
  [\tesla\usk\meter\per\ampere]%
\newphysicalconstant{neutronmass}%
  {\symup{m_n}}%
  {1.7\timestento{-27}}{1.674927471\timestento{-27}}%
  {\kilogram}%
\newphysicalconstant{oofpez}%
  {\symup{\frac{1}{4\pi\epsilon_o}}}%
  {9\timestento{9}}{8.987551787\timestento{9}}%
  {\meter\tothethree\usk\kilogram\usk\second\totheinversefour\usk\ampere\totheinversetwo}%
  [\meter\per\farad]%
  [\newton\usk\meter\tothetwo\per\coulomb\tothetwo]%
\newphysicalconstant{oofpezcs}%
  {\symup{\frac{1}{4\pi\epsilon_o c^2}}}%
  {\tento{-7}}{\tento{-7}}%
  {\meter\usk\kilogram\usk\second\totheinversetwo\usk\ampere\totheinversetwo}%
  [\tesla\usk\meter\tothetwo]%
  [\newton\usk\second\tothetwo\per\coulomb\tothetwo]%
\newphysicalconstant{planck}%
  {\symup{h}}%
  {6.6\timestento{-34}}{6.62607015\timestento{-34}}%
  {\meter\tothetwo\usk\kilogram\usk\second\inverse}%
  [\joule\usk\second]%
  [\joule\usk\second]%
%    \end{macrocode}
%
% See \href{https://tex.stackexchange.com/a/448565/218142}
%  {https://tex.stackexchange.com/a/448565/218142}.
%
%    \begin{macrocode}
\newphysicalconstant{planckbar}%
  {\symup{\lower0.18ex\hbox{\mathchar"AF}\mkern-7mu h}}%
  {1.1\timestento{-34}}{1.054571817\timestento{-34}}%
  {\meter\tothetwo\usk\kilogram\usk\second\inverse}%
  [\joule\usk\second]%
  [\joule\usk\second]
\newphysicalconstant{planckc}%
  {\symup{hc}}%
  {2.0\timestento{-25}}{1.98644586\timestento{-25}}%
  {\meter\tothethree\usk\kilogram\usk\second\totheinversetwo}%
  [\joule\usk\meter]%
  [\joule\usk\meter]%
\newphysicalconstant{protoncharge}%
  {\symup{q_p}}%
  {+\elementarychargeapproximatevalue}{+\elementarychargeprecisevalue}%
  {\ampere\usk\second}%
  [\coulomb]%
  [\coulomb]%
\newphysicalconstant{protonCharge}%
  {\symup{Q_p}}%
  {+\elementarychargeapproximatevalue}{+\elementarychargeprecisevalue}%
  {\ampere\usk\second}%
  [\coulomb]%
  [\coulomb]%
\newphysicalconstant{protonmass}%
  {\symup{m_p}}%
  {1.7\timestento{-27}}{1.672621898\timestento{-27}}%
  {\kilogram}%
\newphysicalconstant{rydberg}%
  {\symup{R_{\infty}}}%
  {1.1\timestento{7}}{1.0973731568508\timestento{7}}%
  {\meter\inverse}%
\newphysicalconstant{speedoflight}%
  {\symup{c}}%
  {3\timestento{8}}{2.99792458\timestento{8}}%
  {\meter\usk\second\inverse}%
  [\meter\per\second]%
  [\meter\per\second]
\newphysicalconstant{stefanboltzmann}%
  {\symup{\sigma}}%
  {5.7\timestento{-8}}{5.670367\timestento{-8}}%
  {\kilogram\usk\second\totheinversethree\usk\kelvin\totheinversefour}%
  [\watt\per\meter\tothetwo\usk\kelvin\tothefour]%
  [\watt\per\meter\tothetwo\usk\kelvin\tothefour]
\newphysicalconstant{sunearthdistance}%
  {\symup{d_{SE}}}%
  {1.5\timestento{11}}{1.496\timestento{11}}%
  {\meter}%
\newphysicalconstant{sunmass}%
  {\symup{M_{Sun}}}%
  {2.0\timestento{30}}{1.98855\timestento{30}}%
  {\kilogram}%
\newphysicalconstant{sunradius}%
  {\symup{R_{Sun}}}%
  {7.0\timestento{8}}{6.957\timestento{8}}%
  {\meter}%
\newphysicalconstant{surfacegravfield}%
  {\symup{g}}%
  {9.8}{9.807}%
  {\meter\usk\second\totheinversetwo}%
  [\newton\per\kilogram]%
  [\newton\per\kilogram]%
\newphysicalconstant{universalgrav}%
  {\symup{G}}%
  {6.7\timestento{-11}}{6.67408\timestento{-11}}%
  {\meter\tothethree\usk\kilogram\inverse\usk\second\totheinversetwo}%
  [\newton\usk\meter\tothetwo\per\kilogram\tothetwo]% % also \joule\usk\meter\per\kilogram\tothetwo
  [\newton\usk\meter\tothetwo\per\kilogram\tothetwo]%
\newphysicalconstant{vacuumpermeability}%
  {\symup{\mu_o}}%
  {4\pi\timestento{-7}}{4\pi\timestento{-7}}%
  {\meter\usk\kilogram\usk\second\totheinversetwo\usk\ampere\totheinversetwo}%
  [\henry\per\meter]%
  [\tesla\usk\meter\per\ampere]%
\newphysicalconstant{vacuumpermittivity}%
  {\symup{\epsilon_o}}%
  {9\timestento{-12}}{8.854187817\timestento{-12}}%
  {\meter\totheinversethree\usk\kilogram\inverse\usk\second\tothefour\usk\ampere\tothetwo}%
  [\farad\per\meter]%
  [\coulomb\tothetwo\per\newton\usk\meter\tothetwo]%
%    \end{macrocode}
%
% A better, intelligent coordinate-free \refCom{vec} command. Note the use of 
% the |e{_^}| type of optional argument. This accounts for much of the 
% flexibility and power of this command. Also note the use of the \TeX\ 
% primitives |\sb{}| and |\sp{}|. Why doesn't it work when I put spaces 
% around |#3| or |#4|? Because outside of |\ExplSyntaxOn...\ExplSyntaxOff|, 
% the |_| character has a different catcode and is treated as a mathematical 
% entity.\newline
% See \href{https://tex.stackexchange.com/q/554706/218142}
%  {https://tex.stackexchange.com/q/554706/218142}.\newline
% See also \href{https://tex.stackexchange.com/a/531037/218142}
%  {https://tex.stackexchange.com/a/531037/218142}.
%
%    \begin{macrocode}
\RenewDocumentCommand{\vec}{ s m e{_^} }{%
  \ensuremath{%
    % Note the \, used to make superscript look better.
    \IfBooleanTF {#1}
      {\vv{#2}%      % * gives an arrow
         % Use \sp{} primitive for superscript.
         % Adjust superscript for the arrow.
         \sp{\IfValueT{#4}{\,#4}\vphantom{\smash[t]{\big|}}}
      }%         
      {\symbfit{#2}  % no * gives us bold
         % Use \sp{} primitive for superscript.
         % No superscript adjustment needed.
         \sp{\IfValueT{#4}{#4}\vphantom{\smash[t]{\big|}}}
      }% 
    % Use \sb{} primitive for subscript.
    \sb{\IfValueT{#3}{#3}\vphantom{\smash[b]{|}}}
  }%
}%
%    \end{macrocode}
%
% A command for the direction of a vector.
% We use a slight tweak is needed to get uniform hats that 
% requires the \href{https://www.ctan.org/pkg/makebox}{\pkg{makebox}} 
% package.\newline
% See \href{https://tex.stackexchange.com/a/391204/218142}
%  {https://tex.stackexchange.com/a/391204/218142}.
%
%    \begin{macrocode}
\NewDocumentCommand{\dirvec}{ s m e{_^} }{%
  \ensuremath{%
    \widehat{\makebox*{\(w\)}{\ensuremath{%
      \IfBooleanTF {#1}
        {%
          #2
        }%
        {%
          \symbfit{#2}
        }%
       }%
      }%
     }%
    \sb{\IfValueT{#3}{#3}\vphantom{\smash[b]{|}}}
    \sp{\IfValueT{#4}{\,#4}\vphantom{\smash[t]{\big|}}}      
  }%
}%
%    \end{macrocode}
%
% The zero vector.
%
%    \begin{macrocode}
\NewDocumentCommand{\zerovec}{ s }{%
  \IfBooleanTF {#1}
    {\vv{0}}%
    {\symbfup{0}}%
}%
%    \end{macrocode}
%
% Notation for column and row vectors. \refCom{mivector} is a workhorse 
% command.\newline
% Orginal code provided by |@egreg|.\newline
% See \href{https://tex.stackexchange.com/a/39054/218142}
%  {https://tex.stackexchange.com/a/39054/218142}.
%
%    \begin{macrocode}
\ExplSyntaxOn
\NewDocumentCommand{\mivector}{ O{,} m o }%
 {%
   \mi_vector:nn { #1 } { #2 }
   \IfValueT{#3}{\;{#3}}
 }%
\seq_new:N \l__mi_list_seq
\cs_new_protected:Npn \mi_vector:nn #1 #2
{%
  \ensuremath{%
    \seq_set_split:Nnn \l__mi_list_seq { , } { #2 }
    \int_compare:nF { \seq_count:N \l__mi_list_seq = 1 } { \left\langle }
    \seq_use:Nnnn \l__mi_list_seq { #1 } { #1 } { #1 }
    \int_compare:nF { \seq_count:N \l__mi_list_seq = 1 } { \right\rangle }
  }%
}%
\NewDocumentCommand{\colvec}{ O{,} m }{%
  \vector_main:nnnn { p } { \\ } { #1 } { #2 }
}%
\NewDocumentCommand{\rowvec}{ O{,} m }{%
  \vector_main:nnnn { p } { & } { #1 } { #2 }
}%
\seq_new:N \l__vector_arg_seq
\cs_new_protected:Npn \vector_main:nnnn #1 #2 #3 #4 {%
  \seq_set_split:Nnn \l__vector_arg_seq { #3 } { #4 }
  \begin{#1NiceMatrix}[r]
    \seq_use:Nnnn \l__vector_arg_seq { #2 } { #2 } { #2 }
  \end{#1NiceMatrix}
}%
\ExplSyntaxOff
%    \end{macrocode}
%
%    \begin{macrocode}
\NewDocumentCommand{\changein}{}{\Delta}
%    \end{macrocode}
%
% Intelligent delimiters provided via the 
% \href{https://www.ctan.org/pkg/mathtools}{\pkg{mathtools}} package.
% Use the starred variants for fractions. You can supply optional sizes. Note that
% default placeholders are used when the argument is empty.
%
%    \begin{macrocode}
\DeclarePairedDelimiterX{\doublebars}[1]{\lVert}{\rVert}{\ifblank{#1}{\:\cdot\:}{#1}}
\DeclarePairedDelimiterX{\singlebars}[1]{\lvert}{\rvert}{\ifblank{#1}{\:\cdot\:}{#1}}
\DeclarePairedDelimiterX{\anglebrackets}[1]{\langle}{\rangle}{\ifblank{#1}{\:\cdot\:}{#1}}
\DeclarePairedDelimiterX{\parentheses}[1]{(}{)}{\ifblank{#1}{\:\cdot\:}{#1}}
\DeclarePairedDelimiterX{\squarebrackets}[1]{\lbrack}{\rbrack}{\ifblank{#1}{\:\cdot\:}{#1}}
\DeclarePairedDelimiterX{\curlybraces}[1]{\lbrace}{\rbrace}{\ifblank{#1}{\:\cdot\:}{#1}}
%    \end{macrocode}
%
% Some semantic aliases. Because of the way \refCom{vec} and
% \refCom{dirvec} are defined, I reluctantly decided not to
% implement a |\magvec| command. It would require accounting
% for too mamy options. So \refCom{magnitude} is the new
% solution.
%
%    \begin{macrocode}
\NewDocumentCommand{\magnitude}{}{\doublebars}
\NewDocumentCommand{\norm}{}{\doublebars}
\NewDocumentCommand{\absolutevalue}{}{\singlebars}
\NewDocumentCommand{\direction}{}{\mivector}
\NewDocumentCommand{\unitvector}{}{\mivector}
%    \end{macrocode}
%
% Intelligent commands for typesetting vector and tensor symbols and 
% components suitable for use with both coordinate-free and index 
% notations. Use starred form for index notation, unstarred form for 
% coordinate-free.
%
%    \begin{macrocode}
\NewDocumentCommand{\veccomp}{ s m }{%
  % Consider renaming this to \vectorsym.
  \IfBooleanTF{#1}
  {%
    \ensuremath{\symnormal{#2}}%
  }%
  {%
    \ensuremath{\symbfit{#2}}%
  }%
}%
\NewDocumentCommand{\tencomp}{ s m }{%
  % Consider renaming this to \tensororsym.
  \IfBooleanTF{#1}
  {%
    \ensuremath{\symsfit{#2}}%
  }%
  {%
    \ensuremath{\symbfsfit{#2}}%
  }%
}%
%    \end{macrocode}
%
% An environment for problem statements. The starred variant gives 
% in-line lists.
%
%    \begin{macrocode}
\NewDocumentEnvironment{physicsproblem}{ m }{%
  \newpage%
  \section*{#1}%
  \newlist{parts}{enumerate}{2}%
  \setlist[parts]{label=\bfseries(\alph*)}}%
  {}%
\NewDocumentEnvironment{physicsproblem*}{ m }{%
  \newpage%
  \section*{#1}%
  \newlist{parts}{enumerate*}{2}%
  \setlist[parts]{label=\bfseries(\alph*)}}%
  {}%
\NewDocumentCommand{\problempart}{}{\item}%
%    \end{macrocode}
%
% An environment for problem solutions.
%
%    \begin{macrocode}
\NewDocumentEnvironment{physicssolution}{ +b }{%
  % Make equation numbering consecutive through the document.
  \begin{align}
    #1
  \end{align}
}{}%
\NewDocumentEnvironment{physicssolution*}{ +b }{%
  % Make equation numbering consecutive through the document.
  \begin{align*}
    #1
  \end{align*}
}{}%
%    \end{macrocode}
%
% A simplified command for importing images.
%
%    \begin{macrocode}
\NewDocumentCommand{\image}{ O{scale=1} m m m }{%
  \begin{figure}[ht!]
    \begin{center}%
      \includegraphics[#1]{#2}%
    \end{center}%
    \caption{#3}%
    \label{#4}%
  \end{figure}%
}%
%    \end{macrocode}
%
% See \href{https://tex.stackexchange.com/q/570223/218142}
%  {https://tex.stackexchange.com/q/570223/218142}.
%
%    \begin{macrocode}
\NewDocumentCommand{\reason}{ O{4cm} m }
  {&&\begin{minipage}{#1}\raggedright\small #2\end{minipage}}
%    \end{macrocode}
%
% Commands for scientific notation.
%
%    \begin{macrocode}
\NewDocumentCommand{\tento}{ m }{\ensuremath{10^{#1}}}
\NewDocumentCommand{\timestento}{ m }{\ensuremath{\;\times\;\tento{#1}}}
\NewDocumentCommand{\xtento}{ m }{\ensuremath{\;\times\;\tento{#1}}}
%    \end{macrocode}
%
% Command for highlighting parts of, or entire, mathematical expressions.\newline
% Original code by anonymous user |@abcdefg|, modified by me.\newline
%  See \href{https://texample.net/tikz/examples/beamer-arrows/}
%   {https://texample.net/tikz/examples/beamer-arrows/}.\newline
%  See also \href{https://tex.stackexchange.com/a/406084/218142}
%   {https://tex.stackexchange.com/a/406084/218142}.\newline
%  See also \href{https://tex.stackexchange.com/a/570858/218142}
%   {https://tex.stackexchange.com/a/570858/218142}.\newline
%  See also \href{https://tex.stackexchange.com/a/570789/218142}
%   {https://tex.stackexchange.com/a/570789/218142}.\newline
%  See also \href{https://tex.stackexchange.com/a/79659/218142}
%   {https://tex.stackexchange.com/a/79659/218142}.\newline
%  See also \href{https://tex.stackexchange.com/q/375032/218142}
%   {https://tex.stackexchange.com/q/375032/218142}.\newline
%  See also \href{https://tex.stackexchange.com/a/571744/218142}%
%   {https://tex.stackexchange.com/a/571744/218142}.
%
%    \begin{macrocode}
\newcounter{tikzhighlightnode}
\NewDocumentCommand{\hilite}{ O{magenta!60} m O{rectangle} }{%
  \stepcounter{tikzhighlightnode}%
  \tikzmarknode{highlighted-node-\number\value{tikzhighlightnode}}{#2}%
  \edef\temp{%
    \noexpand\AddToShipoutPictureBG{%
      \noexpand\begin{tikzpicture}[overlay,remember picture]%
      \noexpand\iftikzmarkoncurrentpage{highlighted-node-\number\value{tikzhighlightnode}}%
       \noexpand\node[inner sep=1.0pt,fill=#1,#3,fit=(highlighted-node-\number\value{tikzhighlightnode})]{};%
      \noexpand\fi
      \noexpand\end{tikzpicture}%
    }%
  }%
  \temp%
}%
%    \end{macrocode}
%
% Intelligent slot command for coordinate-free tensor notation.
%
%    \begin{macrocode}
\NewDocumentCommand{\slot}{ s d[] }{%
  % d[] must be used because of the way consecutive optional
  %  arguments are handled. See xparse docs for details.
  \IfBooleanTF{#1}
  {%
    \IfValueTF{#2}
    {% Insert a vector, but don't show the slot.
      \smash{\makebox[1.5em]{\ensuremath{#2}}}
    }%
    {% No vector, no slot.
      \smash{\makebox[1.5em]{\ensuremath{}}}
    }%
  }%
  {%
    \IfValueTF{#2}
    {% Insert a vector and show the slot.
      \underline{\smash{\makebox[1.5em]{\ensuremath{#2}}}}
    }%
    {% No vector; just show the slot.
      \underline{\smash{\makebox[1.5em]{\ensuremath{}}}}
    }%
  }%
}%
%    \end{macrocode}
%
% Intelligent notation for contraction on pairs of slots.
%
%    \begin{macrocode}
\NewDocumentCommand{\contraction}{ s m }{%
  \IfBooleanTF{#1}
  {\mathsf{C}}%
  {\symbb{C}}%
  _{#2}
}%
%    \end{macrocode}
%
% Intelligent differential (exterior derivative) operator.
%
%    \begin{macrocode}
\NewDocumentCommand{\dd}{ s }{%
  \mathop{}\!
  \IfBooleanTF{#1}
  {\symbfsfup{d}}%
  {\symsfup{d}}%
}%
%    \end{macrocode}
%
% Command to typeset tensor valence.
%
%    \begin{macrocode}
\NewDocumentCommand{\valence}{ s m m }{%
  \IfBooleanTF{#1}
    {(#2,#3)}
    {\binom{#2}{#3}}
}%
%    \end{macrocode}
%
% Diagnostic commands to provide sanity checks on commands that 
% represent physical quantities and constants.
%
%    \begin{macrocode}
\NewDocumentCommand{\checkquantity}{ m }{%
  % Works for both scalar and vector quantities.
  \begin{center}
    \begin{tabular}{>{\centering}p{4cm} >{\centering}p{3cm} >{\centering}p{4cm} >{\centering}p{3cm}}
      name & base & derived & alternate \tabularnewline
      \ttfamily\small{\expandafter\string\csname #1\endcsname} & 
      \small{\csname #1onlybaseunits\endcsname} & 
      \small{\csname #1onlyderivedunits\endcsname} &
      \small{\csname #1onlyalternateunits\endcsname}
    \end{tabular}
  \end{center}
}%
\NewDocumentCommand{\checkconstant}{ m }{%
  \begin{center}
    \begin{tabular}{>{\centering}p{4cm} >{\centering}p{3cm} >{\centering}p{4cm} >{\centering}p{3cm}}
      name & base & derived & alternate \tabularnewline
      \ttfamily\small{\expandafter\string\csname #1\endcsname} & 
      \small{\csname #1onlybaseunits\endcsname} & 
      \small{\csname #1onlyderivedunits\endcsname} &
      \small{\csname #1onlyalternateunits\endcsname} \tabularnewline
      symbol & approximate & precise \tabularnewline
      \small{\csname #1mathsymbol\endcsname} & 
      \small{\csname #1approximatevalue\endcsname} &
      \small{\csname #1precisevalue\endcsname} \tabularnewline
    \end{tabular}
  \end{center}
}%
%    \end{macrocode}
% \restoregeometry
%
% \iffalse
%</package>
% \fi
%
% \Finale
