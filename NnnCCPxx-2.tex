% !TEX program = lualatexmk
% !TEX encoding = UTF-8 Unicode
%
%%%%%%%%%%%%%%%%%%%%%%%%%%%%%%%%%% 80 columns %%%%%%%%%%%%%%%%%%%%%%%%%%%%%%%%%%
% INSTRUCTIONS FOR OVERLEAF USERS:                                              
%                                                                               
% 1.  Create a new, empty Overleaf project. Name it "mandi template" or         
%     something similar.                                                        
% 2a. mandi version 3.22 is included with TeX Live 2024 and is available on     
%     Overleaf. If you want to use this version, all you need are the files     
%     NnnCCPxx-1.tex and NnnCCPxx-2.tex. If you have a free Overleaf account,   
%     choose NnnCCPxx-1.tex. If you have a paid account, choose NnnCCPxx-2.tex. 
%     The reason is that free accounts are limited to a fixed number of compiler
%     passes per document before timing out. A free account should be all a     
%     typical introductory physics student needs.                               
% 2b. Overleaf only updates TeX Live once per year and mandi may have changed   
%     to a version not included in TeX Live. In that case, you will need the    
%     file(s) from part a) and additionally the files mandi.sty,                
%     mandistudent.sty, and mandiexp.sty.                                       
% 2c. The file mandi.pdf is the official bundle documentation, but Overleaf     
%     will not let you view it so you need not upload it. Instead, open it in a 
%     tab in your browser. Alternatively, you can view the documentation online 
%     by going to https://texdoc.org/ and searching for mandi and opening the   
%     resulting PDF file in your browser.                                       
% 2d. Upload the files you need to your new empty project folder on Overleaf.   
% 3.  From inside the project folder, click the Menu button in the upper        
%     left corner and set Compiler to LuaLaTeX, TeX Live Version to the         
%     latest available, and Main document to NnnCCPxx-1.tex (for free           
%     accounts) or NnnCCPxx-2 (for paid accounts). Other options may be set     
%     as desired. Dismiss the configuration window by clicking anywhere on its  
%     exterior.                                                                 
% 4.  Click the NnnCCPxx-1.tex (or NnnCCPxx-2.tex) file so that it is selected. 
%     Now click Recompile and a newly created PDF document should appear.       
% 5.  At this point you have a clean, original project to use as a template.    
% 6.  Make a new copy of your newly created "mandi template" (or whatever you   
%     named it) project. Name it NnnCCPxx by replacing Nnn with your last name  
%     (e.g. Heafner, Rodriguez, Nguyen, Jackson, Lee, etc.), replacing CC with a
%     two digit chapter number (e.g. 03), leaving P unchanged, and xx is a two  
%     or three digit problem number (e.g. 25, 025, 101, etc.). With modern      
%     filesystems practically any names can be accommodated.                    
% 7.  From inside this newly created project, rename the NnnCCPxx-1.tex (or     
%     NnnCCPxx-2.tex) file using the same convention as in the previous step.   
% 8.  The file is now ready for editing and writing up a problem solution.      
%     Your instructor will provide details.                                     
% 9.  You can copy the new files to your projects from within your projects if  
%     you want to. Doing so may necessitate editing your existing documents.    
%%%%%%%%%%%%%%%%%%%%%%%%%%%%%%%%%% 80 columns %%%%%%%%%%%%%%%%%%%%%%%%%%%%%%%%%%

\documentclass[10pt]{article}
\usepackage{mandi}          % load mandi
\usepackage{mandistudent}   % load mandistudent    
\usepackage{mandiexp}       % load mandiexp
\usepackage{numerica}       % for computation
\usepackage{geometry}       % for changing page layout
\usepackage{doi}            % for DOIs in references
\usepackage{mwe}            % for demo images
\usepackage{pgfplots}       % for plots
\pgfplotsset{compat=newest} % for plots
\hypersetup{colorlinks}     % color hyperlinks

%---------- DO NOT EDIT THESE LINES!
%  See https://tex.stackexchange.com/q/156383/218142
\newcommand*{\pkg}[1]{\textsf{#1}}                    % typeset package names
\newcommand*{\mandi}{\textsf{mandi}}                  % typeset mandi
\newcommand*{\mandistudent}{\textsf{mandistudent}}    % typeset mandistudent
\newcommand*{\mandiexp}{\textsf{mandiexp}}            % typeset mandiexp
\newcommand*{\WebVPython}{\texttt{Web VPython}}       % typeset WebVPython
\newcommand*{\WebVPythonorg}{\texttt{WebVPython.org}} % typeset WebVPython.org
\newcommand*{\VPython}{\texttt{VPython}}              % typeset VPython
\newcommand*{\VPythonorg}{\texttt{VPython.org}}       % typeset VPython.org
\newcommand*{\gsurl}{webvpython.org}                  % WebVPython URL
\newcommand*{\vpurl}{vpython.org}                     % VPython URL
\newcommand*{\lualatex}{Lua\LaTeX}                    % typeset LuaLaTeX
%---------- END

%---------- DOCUMENT MARGINS (EDIT TO SUIT)
\geometry{margin=0.5in,top=1.5in,bottom=1.5in}
%---------- END DOCUMENT MARGINS

%---------- STUDENT INFORMATION (EDIT TO SUIT)
\newcommand{\assignment}{ASSIGNMENT TITLE} % Edit this appropriately.
\newcommand{\authorname}{Your Name}        % Edit this to use your name.
%---------- END STUDENT INFORMATION

\begin{document}

\title{\assignment}
\author{\authorname}
\maketitle
\thispagestyle{empty}
\centerline{Using \mandi{} version \mandiversion}
\centerline{Using \mandistudent{} version \mandistudentversion}
\centerline{Using \mandiexp{} version \mandiexpversion}
\listofwebvpythonprograms
\listoffigures
\newpage

%---------- YOUR WORK (EDIT TO SUIT)
\begin{physicsproblem}{PROBLEM TITLE HERE}
Every document must have only ONE problem. You can have as many problem parts as you need. 
Each part will need a \normalfont{\ttfamily{\small{physicssolution}}} environment \emph{only} 
if extended mathematical content is necessary. You can cite a reference\cite{Kap86} if 
appropriate. This sample problem is about the Lorentz factor \( \gamma \).

\begin{parts}
  \problempart
  This is an example of how you would typeset a response that does not require a multistep
  mathematical solution. You can include inline mathematics such as 
  \( \vec{p}=m\gamma\vec{v} \) and you can cite a reference\cite{Cha19}
  if appropriate. If you were asked to state whether the Lorentz factor \( \gamma \)
  is a scalar or a vector,  you would simply state: The Lorentz factor is a scalar.
  
  \problempart
  Here is a sample mathematical solution requiring multiple steps. Let's calculate the
  Lorentz factor for a speed of \( \velocityc{0.986} \).

  % Set up constants for numerica to use
  \nmcConstants{v=0.986}
  
  \begin{physicssolution}
    \gamma &= \frac{1}{\sqrt{1-\frac{\vec{v}\cdot\vec{v}}{\lightspeed^{2}}}}
      \reason{definition of \(\gamma\)} \\
    \vec{v}\cdot\vec{v} &= \magnitude{\vec{v}}^{2}
      \reason{dot product and magnitude squared} \label{step:2} \\
    \vec{v}\cdot\vec{v} &= (\velocityc{0.986})^{2}
      \reason{need to use given value} \\
    \gamma &= \frac{1}{\sqrt{1-\frac{(\velocityc{0.986})^{2}}{\lightspeed^{2}}}}\label{step:4}
      \reason{substitute numerical values} \\
    \gamma &= \frac{1}{\sqrt{1-(0.986)^{2}}}
      \reason{factors of \( \lightspeed \) divide out} \\
    \therefore \gamma &\approx \fpeval{round(1/sqrt(1-0.986^2),3)}\label{step:6}
      \reason{final answer using \LaTeX's \texttt{fpeval}} \\
    \therefore \gamma &\approx \eval{1/\sqrt{1-v^{2}}}[3]\label{step:7}
      \reason{final answer using \pkg{numerica}'s \texttt{eval}}
  \end{physicssolution}

  We gave some steps labels so we can refer back to them. Step \ref{step:2} should be 
  familiar to you. Step \ref{step:4} illustrates why setting \( \lightspeed = 1 \) is 
  handy. Steps \ref{step:6} and \ref{step:7} show how to get \LaTeX{} to do computation 
  for you two different ways. Remember that great minds\cite{Cha15}\cite{Jac99} 
  think alike. Blogs\cite{Hea19} are cool too. Other journals\cite{Cro15} can be cited.
  
  \problempart
  Now, in this part, you are asked to make a plot showing \( \gamma \) as a function
  of speed and to highlight the value of \( \gamma \) for a speed of
  \( \velocityc{0.866} \).

  Here is the requested plot. Let \LaTeX{} take care of its placement for you!
  \begin{figure}[ht!]
  \centering
  \begin{tikzpicture}
    \begin{axis}[%
        declare function = {%
          gamma(\x) = 1/sqrt(1-\x^2);%
          myx = 0.866;% pick a particular value
        },%
        width=0.7\textwidth,%
        title style={font=\sffamily},%
          title=Lorentz Factor vs. Speed,%
        label style={font=\sffamily},%
          xlabel=speed \(\frac{v}{c}\),%
          ylabel={Lorentz factor \( \gamma \)},%
        tick label style={%
          font=\scriptsize,%
          /pgf/number format/precision=1,%
          /pgf/number format/fixed,%
          /pgf/number format/fixed zerofill,%
        },%
        xtick={0,0.2,...,1},%
        ytick={1,2,...,10},%
          extra x ticks={myx},%
          extra tick style={%
            tick label style={%
              font=\tiny,%
              /pgf/number format/precision=3,%
              /pgf/number format/fixed,%
              /pgf/number format/fixed zerofill,%
              color=red,%
            },%
            xticklabel shift={8pt},%
            yticklabel shift={12pt},%
          },%
          extra y ticks={gamma(myx)},%
          extra y tick style={%
          tick label style={rotate=90},%
          },%
        xmin=0,%
        xmax=1.05,%
        ymin=0,%
        ymax=10,%
      ]%
      %-- main plot
      \addplot[blue, very thin, smooth, domain=0:0.999, samples=200] {gamma(x)};
      \addplot[blue, mark=*, mark size=0.5pt] ({myx},{gamma(myx)});
      %-- vertical asymptote
      \draw[red, very thin, densely dashed] (1,0) -- (1,\pgfkeysvalueof{/pgfplots/ymax});
      %-- horizontal asymptote
      \draw[red, very thin, densely dashed] (0,1) -- (\pgfkeysvalueof{/pgfplots/xmax},1);
      %-- particular value
      \draw[red, very thin] (0,{gamma(myx)}) -- (myx,{gamma(myx)});
      \draw[red, very thin] (myx,0) -- (myx,{gamma(myx)});
    \end{axis}
  \end{tikzpicture}
  \caption{Plot of Lorentz factor as a function of speed.}
  \label{plot:1}
  \end{figure}
 
  Figure \ref{plot:1} is a great graph. It's captioned \nameref{plot:1} and is on page 
  \pageref{plot:1}. Notice the asymptotic behavior as speed approaches \(1\) 
  (light's speed). 
  
  Traditional figures can be included with one line of \LaTeX{} code. Again,
  let \LaTeX{} handle the placement for you. The placement may not seem logical
  to you, but that's okay. You can always refer directly to the figure.
  
  \image[scale=1]{example-image-1x1}{A traditional diagram.}{fig:1}

  Figure \ref{fig:1} is nice. It's captioned \nameref{fig:1} and is on 
  page \pageref{fig:1}.

  \problempart
  Finally, suppose you are asked to write and include a \WebVPython{} program. 
  It is simple! Once again, let \LaTeX{} handle the placement for you. The 
  placement may not seem logical to you, but that's okay. Unlike a traditional 
  figure, the program's caption will always be at the top of the program. 
  \textbf{Do not include the \texttt{https://} in the URL because that is 
  added for you!}
  
  \begin{webvpythonblock}(glowscript.org/#/user/heafnerj/folder/mandidemo/program/placeholder)
    {A Placeholder Program}
    Web VPython 3.2
    # This is just a placeholder.
    # It is part of the mandi documentation.
    
    sphere()
  \end{webvpythonblock}

  \WebVPython{} program \ref{gs:1} is nice. It's called \nameref{gs:1} and is on 
  page \pageref{gs:1}. Scanning the QR code, clicking the URL, or clicking anywhere
  in the gray box (between the two thick horizontal black lines) will take you to
  the program's source code.
\end{parts}
\end{physicsproblem}

That's the end of the physics problem itself. If you need to include a bibliography, 
and you really should if you consulted extra resources, that is relatively simple too. 
Study the code to see how to handle various sources. These examples are taken from
physics literature, specifically the 
\href{https://pubs.aip.org/aapt/ajp}{American Journal of Physics}.

\begin{thebibliography}{100} % 100 is a random guess of the total number of references
% Am. J. Phys. one author
\bibitem{Kap86} H. Kaplan, ``The Runge-Lenz vector as an `extra' constant of the motion,'' 
  Am. J. Phys. \textbf{54}, 157-161 (1986); \doi{10.1119/1.14713}
% Am. J. Phys. multiple authors
\bibitem{Cha19} Ruth Chabay, Bruce Sherwood, and Aaron Titus, ``A unified, contemporary
  approach to teaching energy in introductory physics,'' Am. J. Phys. \textbf{87}, 504-509 (2019);
  \doi{10.1119/1.5109519}
% textbook multiple authors
\bibitem{Cha15} R. Chabay and B. Sherwood, \emph{Matter \& Interactions}, 4th ed.
  (John Wiley \& Sons, Hoboken, NJ, 2015).
% textbook one author
\bibitem{Jac99} John D. Jackson, \emph{Classical Electrodynamics}, 3rd ed. 
  (Wiley, Hoboken, NJ, 1999), pp. 464-468.
% blog post
\bibitem{Hea19} J. Heafner, ``Vector Formalism in Introductory Physics VI: A Unified Solution
  for Simple Dot Product and Cross Product Equations.'' (2019) 
  <\url{https://tensortime.sticksandshadows.net/archives/4924}>.
% Phys. Teach. one author
\bibitem{Cro15} R. Cross, ``Precession of a spinning ball rolling down an inclined plane,''
  Phys. Teach. \textbf{53}, 217-219 (2015); \doi{10.1119/1.4914559}
% software portal
\bibitem{Wvp} \WebVPython, <\url{https://www.webvpython.org}>.
% Stack Exchange site
\bibitem{Mse18} ``Mathematica stack exchange,'' <\url{https://mathematica.stackexchange.com/q/105298}>, 
  accessed on August 18, 2018.
% Wikipedia
\bibitem{Wik01} <\url{https://wikipedia.org/wiki/List_of_refractive_indices}>.
\end{thebibliography}
%---------- END YOUR WORK

\end{document}\
