% !TEX program = lualatexmk
% !TEX encoding = UTF-8 Unicode

\documentclass{article}
\usepackage[units=derived]{mandi}
\usepackage{mandistudent}
%\setmainfont{Helvetica Neue Light}
%\setmainfont{GFS Neohellenic}
%\setmainfont{TeX Gyre Heros}
%\setmainfont{TeX Gyre Termes}
%\setmainfont{TeX Gyre Schola}
%\setmainfont{TeX Gyre Bonum}
%\setmainfont{TeX Gyre Cursor}
%\setmainfont{TeX Gyre Chorus}
%\setmainfont{TeX Gyre Adventor}

\begin{document}

Momentum is defined as \( \vec{p} = \gamma m\vec{v}\).

\begin{physicssolution}
  \vec{p} &= \gamma m \vec{v} \reason{definition of momentum} \\
  \vec{v} &= \frac{\vec{p}}{\gamma m} \reason{solve for velocity}
\end{physicssolution}

The dot product is really a contraction on two slots, and can be notated 
as \( \contraction*{1,2} \).

\newpage
%\itshape\bfseries
The momentum can be expressed in all the following ways:

\begin{gather*}
 \momentum{4}               \\
 \momentumbaseunits{4}      \\
 \momentumderivedunits{4}   \\
 \momentumalternateunits{4} \\
 \vectormomentum{3,2,-4}    \\
 \momentumvector{3,2,-4}
\end{gather*}
\begin{gather*}
  \force{3}         \\
  \energy{3}        \\
  \magneticfield{3}
\end{gather*}

\mandisetup{units=base}
The capacitance can be expressed in all the following ways:

\begin{gather*}
 \capacitance{4}               \\
 \capacitancebaseunits{4}      \\
 \capacitancederivedunits{4}   \\
 \capacitancealternateunits{4}
\end{gather*}
\begin{gather*}
  \force{3}         \\
  \energy{3}        \\
  \magneticfield{3}
\end{gather*}

\mandisetup{units=alternate}
The resistance can be expressed in all the following ways:

\begin{gather*}
 \resistance{4}               \\
 \resistancebaseunits{4}      \\
 \resistancederivedunits{4}   \\
 \resistancealternateunits{4}
\end{gather*}
\begin{gather*}
  \force{3}         \\
  \energy{3}        \\
  \magneticfield{3}
\end{gather*}

A current of \( \current{2} \) and a resistance of \( \resistance{3} \) gives a potential
difference of \( \electricpotentialdifference{6} \).

\newpage
\checkquantity{electricpotentialdifference}
\checkquantity{energy}
\checkquantity{angularmomentum}
\checkquantity{momentum}
\checkconstant{oofpez}
\checkconstant{vacuumpermeability}
\checkconstant{vacuumpermittivity}

\end{document}
